\section{Session Four}\label{sec:session_four}

\subsection{Control}\label{sub:control}

Control allows you to chose which path to follow based on conditions.
A branch is somewhere the code can follow \(>1\) path.

\subsubsection{The if statement}\label{ssub:the-if-statement}

This is the very simplest type of branching.
\begin{highlight}{Python if statements}
    \begin{code}{python}
        if x:
        branch_one()
        if y:
        branch_two()
    \end{code}
\end{highlight}

\subsubsection{The else if statement}\label{ssub:the-else-if-statement}

One of the two paths are always followed.
\begin{highlight}{The Python else if statement}
    \begin{code}{python}
        if x:
        branch_one()
        else:
        branch_two()
    \end{code}
\end{highlight}

\subsubsection{The elif statement}\label{ssub:the-elif-statement}

\begin{itemize}
    \item Can handle more than just two outcomes.
    \item Could also be done with just lots of \mintinline{python}{if} statements, this is nasty and hard to read.
\end{itemize}

\begin{note}
    Switch and case statements can also handle more than one branch too.
\end{note}

\subsection{Iteration}\label{sub:iteration}

Repeating code is what makes computers powerful.
The key thing about looping is you do the same thing on different data.

\subsubsection{Definite Iteration}\label{ssub:definite-iteration}

\begin{itemize}
    \item Loop a certain number of times.
    \item Usually done with a \mintinline{python}{for} keyword.
\end{itemize}

\subsubsection{Nested Loops}\label{ssub:nested-loops}

\begin{itemize}
    \item One loop inside of another.
    \item Too many levels of looping is hard to read and slow.
\end{itemize}

\subsubsection{Indefinite Iteration}\label{ssub:indefinite-iteration}

\begin{itemize}
    \item Continues until some condition is met.
    \item Usually done with the \mintinline{python}{while} keyword.
    \item If the condition is \mintinline{java}{true}, the loop continues, otherwise it stops.
    \item Generally worse than definite loops.
    \item Use indefinite loops only when you don't know how many iterations there will be.
\end{itemize}
