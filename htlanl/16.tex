\section{Session 15}\label{sec:session_15}

\subsection{Abstract Classes}\label{sub:abstract_classes}

Use the \mintinline{python}{abstract} keyword.
An abstract class cannot be instantiated, only extended.
A subclass must implement all of the methods, but if it doesn't, it must be an abstract class too.

An abstract class can have defined methods and abstract methods (which are just signitures to be implemented in a child).

\subsubsection{Abstract interface or an interface?}\label{ssub:abstract_interface_or_an_interface_}

\begin{itemize}
	\item Neither can be instantiated.
	\item Both can have implemented methods:
	      \begin{itemize}
		      \item In an interface, they cannot be overwritten.
		      \item In an abstract class they can be overwritten.
	      \end{itemize}
	\item Abstract classes can have non-static non-final variables (because concrete methods have to access them).
	\item A class can only extend \(1\) (abstract) class, but a class can implement many interfaces (\emph{multiple inheritance}).
	\item Interfaces are generally more common.
	\item An abstract class can have a constructor (only callable from a child).
	\item Use abstract classes if several closely related classes need to share code.
	\item Interfaces have no state, abstract classes do.
	\item Interfaces can be completely unrelated, abstract classes usually extend similarly.
\end{itemize}
