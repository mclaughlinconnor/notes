\section{Session Ten}\label{sec:session_ten}

\subsection{Dictionaries cont.}\label{sub:dictionaries_cont_}

\begin{itemize}
	\item \mintinline{python}{len()} can be used to get the number of values in a dictionary.
	\item \mintinline{python}{items.keys()} and \mintinline{python}{items.values()} can be used to get keys and values.
	\item \mintinline{python}{"one" in items} can be used to test whether key "one" is in \mintinline{python}{items}.
	      This is mostly used for keeping track of values that have been seen before.
	      Time to calculate this does not vary with the length of the dictionary.
	\item \mintinline{python}{items.update()} will merge two dictionaries overwriting the values in the first from the second.
	\item \mintinline{python}{items.copy()} works the same as with lists where just assigning creates a reference only.
	      \mintinline{python}{items.copy()} only creates a new instance of the dictionary itself, but the values themselves are still references.
	      \mintinline{python}{copy.deepcopy()} should be used to create new instances of the values too.
	\item Dictionaries can be iterated over just like lists.
	      \mintinline{python}{for key in items} loops over the keys.
	      \mintinline{python}{for value in items.values()} loops over the values of the dictionary.
	      \mintinline{python}{for key, value in items.items()} loops over the keys and values at the same time.
\end{itemize}

\subsection{JavaScript Object Notation (JSON)}\label{sub:javascript_object_notation_json_}

JSON is not in any special format - it is just text that looks like a dictionary.

