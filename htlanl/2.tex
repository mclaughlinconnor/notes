\section{Session Two}\label{sub:session_2_types}

\subsection{Types}\label{ssub:types}

\emph{Primitive types} are ones that cannot be broken down (\mintinline{c}{int}, \mintinline{c}{char}, \mintinline{c}{long}).
\emph{Composite types} can be broken down (array, class and sometimes strings are \mintinline{c}{char[]}s)

\begin{note}
    In most old languages (like C, C++), booleans are often represented by just a \(1\) or  \(0\).
    Also, strings are often just stored as an array of ASCII codes, Python is an exception where \mintinline{c}{chars} are just strings with length 1
\end{note}

\subsection{Learning Logs}\label{ssub:learning_logs}

A final workplace report worth \(25\%\) is due 2021-03-01.
\begin{itemize}
    \item Reflect and show what was learned at work.
    \item Should be about \(1500\) words in length.
    \item Should include a critical evaluation of the programming practices encountered.
    \item Should include examples of programming concepts covered in HTLANL (inheritance, scope, etc.).
    \item \textbf{What have I learned?}
\end{itemize}

\subsection{Academic Writing}\label{ssub:academic_writing}

On Moodle there is a basic introduction, phrase bank and a like to LEADS (a university program to help with academic writing).

Reflective writing is evidence of reflective thinking.
Reflective thinking helps to reinforce what we have learned.

Reflective writing is generally a more relaxed form of writing where you can write in first person (I, me, we).

Reflective writing should aim to look back at an event (\emph{description}), analyse and explain the event (\emph{interpretation}) and decide what this means for me in the future (\emph{outcome}).

\begin{note}
    When describing a theory or model, use present tense since the theory or model is still the same.
    When talking about events, use past tense since the events happened in the past.
\end{note}

