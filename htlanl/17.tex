\section{Session 17}\label{sec:session_17}

\subsection{Enumerations}\label{sub:enumerations}

An enumeration allows a variable to be one of a set on constants.
Java has a special \mintinline{python}{enum} class which is allowed to have methods.
The names of values are generally all upper case (just like regular constants).

\begin{itemize}
    \item \mintinline{python}{.values} gives an array of values
    \item \mintinline{python}{.valueOf(str)} gives the enum constant of the string
    \item \mintinline{python}{.equals()}, \mintinline{python}{.compareTo()}, \mintinline{python}{toString()} work too.
\end{itemize}

\begin{highlight}{Enumerations}
    \begin{code}{java}
        public enum Planet {
            MERCURY (...),
            MARS (...);

            private final double mass;
            private final double radius;

            Planet (double m, double r) {
                this.mass = m;
                this.radius = r;
                }
            }
    \end{code}
\end{highlight}

\subsection{Paths/Files}\label{sub:paths_files}

A path identifies the path to a file.
A \emph{delimiter} separates the directory names (is ``\\'' on windows and ``/'' everywhere else).

Java uses \mintinline{python}{java.nio.file.Path} which must be imported to represent a path.
This class lets you examine, locate, etc., paths.
\mintinline{python}{java.nio.file.Files} has methods for copying, moving, etc., files.

\subsection{Exceptions}\label{sub:exceptions}

Exceptions are thrown when an error occurs, unless caught using a \mintinline{java}{try} \mintinline{java}{catch} block, the program will crash.

You should keep error handling code separate from all other code preferably.
Error handling can avoid having lots of conditional statements if you assume everything just works and worry about exceptions later on.
