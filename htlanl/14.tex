\section{Session 14}\label{sec:session_14}

\subsection{Method overriding}\label{sub:method_overriding}

\emph{Overriding} is used to change the behaviour of a parent class in the child class.
Any overridden method must have the same method signature (name, types, parameters) as the parent.
An overridden method can still call \mintinline{python}{super.method()}.

\subsubsection{Why override?}\label{ssub:why_override_}

\begin{itemize}
    \item Allows for runtime polymorphism where the compiler runs at runtime and decides what code to run.
    \item Allows for a general class that specifies methods that will be common to all subclasses while allowing subclasses to define the implementation.
    \item ``One interface, multiple methods''.
    \item Gives a consistent interface with a flexible implementation.
\end{itemize}


\subsection{Method Overloading}\label{sub:method_overloading}

Two functions have the same name, but accept different parameters/types (this is also polymorphism).
When called the compiler uses the types of the parameters provided to figure out which implementation to call.
The return types of methods can be different, but this doesn't change how the method is resolved.

The main benefit of method overloading is avoiding having lots of similar methods with slightly different names.

\mintinline{python}{final} prevents a method from being overridden.

