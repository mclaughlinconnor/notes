\section{Session 18}\label{sec:session_18}

\subsection{SOLID}\label{sub:solid}

\begin{itemize}
	\item \textbf{S}ingle responsibility principle.
	\item \textbf{O}pen closed principle.
	\item \textbf{L}iskov substitution principle
	\item \textbf{I}ntegration segregation principle
	\item \textbf{D}ependency inversion principle.
\end{itemize}

\subsubsection{Single Responsibility Principle}\label{ssub:single_responsibility_principle}

A class should don only one thing or job (there's only one readon to update a class's code).

\subsubsection{Open Closed Principle}\label{ssub:open_closed_prinicple}

Objects should be open for extension, but closed for modification (modifying the original will cause breakages).

\subsubsection{Liskov Substitution Principle}\label{ssub:liskov_substitution_principle}

\begin{quote}
	``Let \(q(x)\) be a property provable about objects of \(x\) of type \(T\). Then \(q(y)\) should be provable for objects of \(y\) of type \(S\) where \(S\) is a subtype of \(T\).''
\end{quote}
%
Basically a subclass should be able to be replaced with its parent.

\subsubsection{Integration Segregation Principle}\label{ssub:integration_segregation_principle}

A client should never be office to implement an interface it doesn't need or clients should not depend on methods it doesn't need (make many small interfaces).

\subsubsection{Dependency Inversion Principle}\label{ssub:dependency_inversion_principle}

Entities must depend on abstraction not concretions (depend on interfaces, not specific implementations).

\subsection{DRY}\label{sub:dry}

\begin{itemize}
	\item \textbf{D}on't
	\item \textbf{R}epeat
	\item \textbf{Y}ourself
\end{itemize}

\begin{itemize}
	\item Create modules (modulisation)
	\item Create small reusable chunks of code to be reused in other places or codebases.
	\item Also KISS (keep it simple, stupid) which means that simple is better.
\end{itemize}

\subsection{Code Smells}\label{sub:code_smells}

Code smells are a surface indication that points to underlying problems.
Eg.\ too many parameters, too long methods, magic numbers.
