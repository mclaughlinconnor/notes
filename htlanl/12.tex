\section{Session 11}\label{sec:session_11}

\subsection{Programming paradigms}\label{sub:programming_paradigms}

We have procedural programming which focuses on what is happening (the code).
We also have object oriented programming which focuses on what is being affected (the data).

\subsection{Abstraction}\label{sub:abstraction}

\emph{Abstraction} is where you break down big problems into smaller general ones.
\emph{Parametrisation} and \emph{isolation} can help us achieve this.

\paragraph{Parametrisation}\label{par:parametrisation}

This is where we have several parts that can all be varied controllably using parameters.

\paragraph{Isolation}\label{par:isolation_one}

This is where we separate (isolate) variables from each other.

\medskip
\noindent
A car has many parts (steering, heating, brakes), but is considered only one object.
As a user of a car, I don't need to know how the brakes work, only how to use them.

\subsection{Encapsulation}\label{sub:encapsulation}

\emph{Encapsulation} keeps code and data sage from outside interference or misuse (you cannot change gear manually, you must use a gear shifter).
Encapsulation lets us provide a well defined interface for the object that other pieces of code can use.

\subsection{Classes}\label{sub:classes}

The basis of encapsulation is classes.
A class defines the structure and the behaviour defined by an object.
Objects are sometimes called instances of classes.

The code and data of a class are called members.
The data are member variables or instance variables.
The behaviour and interface of the class are defined by methods that operate on its instance data.

Encapsulation is where we hide the complexity of a class.
\mintinline{python}{public} methods can be accessed from outside of the class and form the public interface of a class.
\mintinline{python}{private} methods can only be accessed from inside of the class and are usually altered with getters and setters.
Encapsulation is used to prevent improper use of you code (a car cannot just go to \(100\)mph, it must accelerate first).

\subsection{Inheritance}\label{sub:inheritance}

``One object acquires properties from its parent'': A Labrador is a dog, mammal, and an animal.

Inheritance means that an object only has to define what makes it unique which means that there is only ever one implementation for each method.

\subsection{Polymorphism}\label{sub:polymorphism}

Polymorphism allows a single interface to be used as a single class of actions: ``one class, multiple methods''.
It means you can design a generic interface to several related activities to reduce complexity and let the computer decide which action to perform.
%
\begin{minted}[linenos,numbersep=5pt,frame=lines,framesep=2mm]{python}
class Dog:
    def look(h: Human):
        love(h)

    def look(s: Squirell):
        hate(s)
\end{minted}
%
Both functions have the same name, but perform different actions based on the parameter passed in.
