\section{Session Six}\label{sec:session_six}

\subsection{Work-based Task}\label{sub:work_based_task}

Ask someone how they go about learning a new programming language or learning something new about a language they already know.

\subsection{Scope}\label{sub:scope}

Creating new variables inside of functions creates ones that are only available inside of that function named \emph{local variables}.
``Each function has its own workspace.''
Local variables are not persisted across calls.

\subsubsection{Global Scope}\label{ssub:global_scope}

These are variables outside of a function which are visible to all code.
Some languages don't really have global variables (C\# and Java just have \mintinline{java}{static}) and in Python, globals are read only by default to protect from accidental writes in local scopes (use \mintinline{python}{global} to write to them).

\subsubsection{Shadowing}\label{ssub:shadowing}

Local variables can shadow a global making it inaccessible.
In Python \mintinline{python}{global} must be used to edit globals in local scopes.

\subsubsection{Evils of Globalisation}\label{ssub:evils_of_glovalisation}

\emph{Globalisation} is generally considered bad practice because it makes functions rely on each other and variables can be changed accidentally leading to unintended behaviour.

\subsection{Recursion}\label{sub:recursion}

\emph{Recursion} is when a function calls itself.
This is an alternative to \mintinline{python}{while} and \mintinline{python}{for} loops.
Some problems can be solved by calling the same function on the data returned from last time.
eg. Fibonacci sequences, fractals, binary trees, sorts.

