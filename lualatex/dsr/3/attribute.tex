%! TeX program = lualatex

\documentclass{standalone}
\usepackage{tikz}
\usepackage{tikz-er2}
\usetikzlibrary{arrows.meta, decorations.markings, decorations.pathmorphing, graphdrawing}
\usegdlibrary{force}
\usepackage{ulem}

\begin{document}
\tikz[
    % orient=right,
    spring electrical layout,
    % component sep=2cm,
    node distance=10mm,
    electric charge=2,
    % electric force order=2,
    thick,
    one/.style={
            decoration={
                    markings,
                    mark = at position 0.2 with {\node[fill=black!20]{\(\mathbf{1}\)};}
                },
            postaction={decorate}
        },
    nany/.style={
            decoration={
                    markings,
                    mark = at position 0.2 with {\node[fill=black!20]{\(\mathbf{n}\)};}
                },
            postaction={decorate}
        },
    many/.style={
            decoration={
                    markings,
                    mark = at position 0.2 with {\node[fill=black!20]{\(\mathbf{m}\)};}
                },
            postaction={decorate}
        },
    subentity/.style={
            decoration={
                    markings,
                    mark = at position 0.25 with {\pgftransformscale{3}\arrow{Arc Barb}}
                },
            postaction={decorate}
        }
] {
    \node[entity] (entity) {Entity};
    \node[attribute, anchor=north] (atrone) {Attribute One};
    \node[attribute, anchor=south] (atrtwo) {Attribute Two};
    \node[attribute, anchor=west] (atrthree) {Attribute Three};
    \node[attribute, anchor=east] (atrfour) {Attribute Four};

    \draw (entity) edge (atrone);
    \draw (entity) edge (atrtwo);
    \draw (entity) edge (atrthree);
    \draw (entity) edge (atrfour);
}
\end{document}

