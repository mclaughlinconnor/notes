\section{Process Scheduling Introduction}\label{sec:process_scheduling_introduction}

\subsection{How does the OS provide a multiprogramming environment}\label{sub:how_does_the_os_provide_a_multiprogramming_environment}

The process abstraction is used to structure this service.
The OS gives a system to access services to manage processes (an API).
This API decides which program should run first, for how long and when it should run (a policy).

\subsection{Process}\label{sub:process}

A process is a program in execution.
A program is a passive entity, whereas a process is an active entity.

A process needs certain resources like CPU time, memory, files and IO devices.
The three main components of a process are data for a program, the executable itself and the context (registers, files, etc.)

\subsubsection{How Does The OS Deal With Processes?}\label{ssub:how_does_the_os_deal_with_processes}

The OS is responsible for three process based tasks:
\begin{itemize}
    \item Process creation and deletion (done in a hierarchical fashion -- a process can create a process)
    \item Process suspension and resumption
    \item Provisions for process synchronisation and communication
\end{itemize}

\subsubsection{Process States and Transitions}\label{ssub:process_states_and_transitions}

There are four primary process states:
\begin{itemize}
    \item NEW
    \item READY
    \item RUNNING
    \item BLOCKED
\end{itemize}
The transitions between these states are as follows:
\begin{description}
    \item[Admission] Moving from NEW to READY
    \item[Dispatch] Moving from READY to RUNNING when a program gets a shot on the CPU
    \item[Timeout] Moving from RUNNING to READY to let another process have a turn on the CPU
    \item[Blocking] Moving from RUNNING to BLOCKING when waiting for data from an IO device
    \item[Wakeup] Moving from BLOCKED to READY when the data a process was waiting from arrives
    \item[Exit] Process is complete (ie., it was killed, terminated or destroyed)
\end{description}

\begin{highlight}{CPU running state transitions}
    \begin{tikzpicture}[node distance=20mm]
        \draw
        node (n) {New}
        node[right=of n] (r) {READY}
        node[right=of r] (ru) {RUNNING}
        node[right=of ru] (t) {TERMINATE};
        \coordinate (btop) at ($(ru)!.5!(r)$);
        \draw node[below of=btop] (b) {BLOCKED};
        \draw[->] (n) -- node[midway, above] {Admission} (r);

        \draw[->] ($(r.east) + (0,0.2)$) -- node[midway, above] {Dispatch} ($(ru.west) + (0,0.2)$);
        \draw[<-] ($(r.east) - (0,0.2)$) -- node[midway, below] {Timeout} ($(ru.west) - (0,0.2)$);

        \draw[->] (ru) -- node[midway, above] {Exit} (t);

        \draw[->] (ru) -- node[midway, right, xshift=2mm] {Blocking} (b);
        \draw[<-] (r) -- node[midway, left, xshift=-2mm] {Wakeup} (b);
    \end{tikzpicture}
\end{highlight}

\subsection{Process Scheduling}\label{sub:process_scheduling}

The main aim of a process scheduler is to maximize the usage of the CPU.
In order to do so, it must maintain several queues:
\begin{description}
    \item[Job Queue] Stores all processes
    \item[Ready Queue] Processes currently in RAM and ready to execute
    \item[Device Queues] There is a queue for each device containing a list of processes waiting on it
\end{description}

\begin{note}
    Processes can move between queues when their states change, etc.
\end{note}

\subsection{Schedulers}\label{sub:schedulers}

The OS can make scheduling decisions at any of four events:
\begin{itemize}
    \item When a process switches from RUNNING to BLOCKED (eg., on an IO request or sleep)
    \item Moving from RUNNING to READY (when a timout interrupt occurs)
    \item Moving from BLOCKED to READY (when an IO request has completed)
    \item When the program terminates
\end{itemize}
A scheduler can be one of two types:
\begin{description}
    \item[Non-Preemptive] The process gets to decide when to give up (``yield'') the CPU
    \item[Preemptive] The process scheduler decides when a process yields the CPU
\end{description}

\subsection{Criteria for comparing CPU schedulers}\label{sub:criteria_for_comparing_cpu_shcedulers}

\begin{itemize}
    \item CPU utilisation (more is better): how much a scheduler keeps the CPU busy.
    \item Throughput (more is better): number of processes computed per time unit.
    \item Waiting time (less is better): the total time a process spends in the ready queue (ie., loaded but not executing on the cpu)
    \item Turnaround time (less is better): time from submission of a process until the process completes
    \item Response time (less is better): time from submission until first response
\end{itemize}

\subsection{Some Process Scheduling Policies}\label{sub:some_process_scheduling_policies}

\begin{itemize}
    \item First Come First Served (FCFS or FIFO)
    \item Priority based (prementive and non-premtive)
    \item Shortest Job First (SJF)
    \item Shortest Remaining Time First (SRTF)
    \item Round Robin(RR)
\end{itemize}
%
\begin{note}
    Switching between processes is quite slow, but we will ignore this here.
\end{note}
%
In the examples below, we will be using the following processes for each.
\begin{highlight}{Process scheduling processes}
    \begin{tabular}{cccc}
        \toprule
        Process ID & Arrival Time & CPU Time & Priority \\
        \midrule
        1          & 0            & 4        & 2        \\
        2          & 4            & 16       & 3        \\
        3          & 6            & 10       & 1        \\
        \bottomrule
    \end{tabular}
\end{highlight}

\subsubsection{First Come First Served}\label{ssub:first_come_first_served}

The First Come First Serve scheduler is a non-preemptive, simple to implement scheduler than has some of the highest average wait times.
The algorithm:
\begin{enumerate}
    \item Add each process to the queue in order of their arrival
    \item Execute the first process
    \item If the process needs more time than the scheduler wants to give it, put it back to the end of the queue
\end{enumerate}

\paragraph{Time Diagram}\label{par:time_diagram}

\begin{highlight}{First come first served CPU policy: time diagram}
    \begin{tikzpicture}[scale=0.4]
        \def\myNotes{1 / 0 / 6, 2 / 6 / 14, 3 / 20 / 10}
        \draw[-latex] (0, 0) -- (32,0);
        \foreach \x in {0, 5, 10, 15, 20, 25, 30}
        \draw[shift={(\x,0)},color=black] (0pt,3pt) -- (0pt,-3pt);
        \foreach \x in {0, 5, 10, 15, 20, 25, 30}
        \draw[shift={(\x,0)},color=black] (0pt,0pt) -- (0pt,-3pt) node[below] {$\x$};
        \foreach \pid/\start/\length [count=\i] in \myNotes
        \draw[->] ($ (\start, -1) - (0, \i) $) -- node[midway, below] {Process \(\pid\)} ++(\length, 0);
    \end{tikzpicture}
\end{highlight}

\paragraph{Wait Times}\label{par:wait_times}

For calculating the start time of a process we can use the general formula:
\[
    \text{wait time} = \text{start time} - \text{arrival time at CPU}
\]
Here are some examples using the data from our examples:
\begin{highlight}{First come first served CPU policy: wait times}
    \begin{align*}
        \text{Process \(1\)} & = 0 - 0 = 0   \\
        \text{Process \(2\)} & = 6 - 4 = 2   \\
        \text{Process \(3\)} & = 20 - 5 = 15
    \end{align*}
\end{highlight}

\paragraph{Average Wait Time}\label{par:average_wait_time}

The average wait time can be calculated by summing and dividing the wait times:
\begin{highlight}{First come first served CPU policy: average wait time}
    \begin{align*}
        \Sigma t     & = 0 + 2 + 15         \\
        \Sigma t     & = 17                 \\
        \overline{t} & = \frac{\Sigma t}{n} \\
        \overline{t} & = \frac{17}{3}       \\
        \overline{t} & = 3
    \end{align*}
\end{highlight}

\paragraph{Turnaround Time}\label{par:turnaround_time}

The turnaround time of a process can be calculated using the general formula:
\[
    \text{turnaround time} = \text{completion time} - \text{arrival time}
\]
Here are some examples using the data from our examples:
\begin{highlight}{First come first served CPU policy: turnaround time}
    \begin{align*}
        \text{Process \(1\)} & = 6 - 0 = 6   \\
        \text{Process \(2\)} & = 20 - 4 = 16 \\
        \text{Process \(3\)} & = 30 - 5 = 25
    \end{align*}
\end{highlight}

\subsection{Shortest job first}\label{sub:shortest_job_first}

Shortest job first is a simple to implement, non-preemptive scheduler that processes jobs in order of how much CPU time they still need.
SJF has the provable minimum average wait time, but many short jobs one after each other will increase the time a long job has to wait.

\paragraph{Time Diagram}\label{par:time_diagram_2}

\begin{highlight}{Shortest job first: time diagram}
    \begin{tikzpicture}[scale=0.4]
        \def\myNotes{1 / 0 / 6, 3 / 6 / 10, 2 / 16 / 14}
        \draw[-latex] (0, 0) -- (32,0);
        \foreach \x in {0, 5, 10, 15, 20, 25, 30}
        \draw[shift={(\x,0)},color=black] (0pt,3pt) -- (0pt,-3pt);
        \foreach \x in {0, 5, 10, 15, 20, 25, 30}
        \draw[shift={(\x,0)},color=black] (0pt,0pt) -- (0pt,-3pt) node[below] {$\x$};
        \foreach \pid/\start/\length [count=\i] in \myNotes
        \draw[->] ($ (\start, -1) - (0, \i) $) -- node[midway, below] {Process \(\pid\)} ++(\length, 0);
    \end{tikzpicture}
\end{highlight}

\paragraph{Wait Times}\label{par:wait_times_1}

For calculating the start time of a process we can use the general formula:
\[
    \text{wait time} = \text{start time} - \text{arrival time at CPU}
\]
Here are some examples using the data from our examples:
\begin{highlight}{Shortest job first: wait times}
    \begin{align*}
        \text{Process \(1\)} & = 0 - 0 = 0   \\
        \text{Process \(2\)} & = 16 - 4 = 12 \\
        \text{Process \(3\)} & = 6 - 5 = 1
    \end{align*}
\end{highlight}

\paragraph{Average Wait Time}\label{par:average_wait_time_1}

The average wait time can be calculated by summing and dividing the wait times:
\begin{highlight}{Shortest job first: average wait times}
    \begin{align*}
        \Sigma t     & = 0 + 12 + 1         \\
        \Sigma t     & = 13                 \\
        \overline{t} & = \frac{\Sigma t}{n} \\
        \overline{t} & = \frac{13}{3}       \\
        \overline{t} & = 4.3
    \end{align*}
\end{highlight}

\paragraph{Turnaround Time}\label{par:turnaround_time_1}

The turnaround time of a process can be calculated using the general formula:
\[
    \text{turnaround time} = \text{completion time} - \text{arrival time}
\]
Here are some examples using the data from our examples:
\begin{highlight}{Shortest job first: turnaround times}
    \begin{align*}
        \text{Process \(1\)} & = 6 - 0 = 6   \\
        \text{Process \(2\)} & = 30 - 4 = 26 \\
        \text{Process \(3\)} & = 16 - 5 = 11
    \end{align*}
\end{highlight}

\subsection{Shortest Time Remaining First}\label{sub:shortest_time_remaining_first}

This is effectively a preemptive version of Shortest Job First scheduling which also has a provably minimum average wait time.
Processes are added to the queue in order of the total time they still need to execute.
Where two processes need the same time, a First Come First Serve approach will be used.
Where a new processes has less time remaining than the current one, it gets priority.

\begin{note}
    The time diagram and calculations have been omitted here for brevity since they are \emph{exactly} as for Shortest Job First.
\end{note}

\subsection{Non-Preemptive Priority Scheduling}\label{sub:non_preemptive_priority_scheduling}

Non-Preemptive Priority Scheduling is an easy to implement scheduling system where each process is given a priority (a larger priority value means a lower actual priority) which is used to handle processes in order.
Where more than one process has the same priority as another, a First Come First Serve approach will be used.
This system can lead to lower priorities blocking indefinitely (``starvation'')

\begin{note}
    The time diagram and calculations have been omitted here for brevity since they are \emph{exactly} as for Shortest Job First.
    The reason they are the same an not instead executed in order \(3\), \(1\), \(2\) is: when the CPU starts, Process \(3\) is not available yet, so Process \(1\) is started.
\end{note}

\subsection{Round robin}\label{sub:round_robin}

The Round Robin system a preemptive version of the First Come First Serve system where processes are added to the queue in order of arrival.
The difference lies in that, if a process takes longer than a specified time quantum/time slice, they are suspended and moved to the end of the queue

Round Robin is also easy to implement, but here there may be very long wait times since each process will need to repeatedly wait for its time slice to come back -- performance depends heavily on the time slice value (think of extreme numbers).
We also see a slight performance hit since processes have to switch far more often.

\paragraph{Time Diagram}\label{par:time_diagram_3}

\begin{highlight}{Round robin: time diagram}
    \begin{tikzpicture}[scale=0.4]
        \def\myNotes{1/0/2, 1/2/2, 2/4/2, 1/6/2, 3/8/2, 2/10/2, 3/12/2, 2/14/2, 3/16/2, 2/18/2, 3/20/2, 2/22/2, 3/24/2, 2/26/2, 3/28/2}
        \draw[-latex] (0, 0) -- (32,0);
        \foreach \x in {0, 5, 10, 15, 20, 25, 30}
        \draw[shift={(\x,0)},color=black] (0pt,3pt) -- (0pt,-3pt);
        \foreach \x in {0, 5, 10, 15, 20, 25, 30}
        \draw[shift={(\x,0)},color=black] (0pt,0pt) -- (0pt,-3pt) node[below] {$\x$};
        \foreach \pid/\start/\length [count=\i] in \myNotes
        \draw[->] ($ (\start, -1) - (0, \i) $) -- node[midway, right, rotate=-90] {Process \(\pid\)} ++(\length, 0);
    \end{tikzpicture}
\end{highlight}

\paragraph{Wait Times}\label{par:wait_times_2}

For calculating the wait time with Round Robin, just count the time between when the process was received by the processor and when it finishes executing where that process is not being handled.
\begin{highlight}{Round robin: wait times}
    \begin{align*}
        \text{Process \(1\)} & = 2  \\
        \text{Process \(2\)} & = 12 \\
        \text{Process \(3\)} & = 11
    \end{align*}
\end{highlight}

\paragraph{Average Wait Time}\label{par:average_wait_time_2}

The average wait time can be calculated by summing and dividing the wait times:

\begin{highlight}{Round robin: average wait times}
    \begin{align*}
        \Sigma t     & = 2 + 12 + 11        \\
        \Sigma t     & = 25                 \\
        \overline{t} & = \frac{\Sigma t}{n} \\
        \overline{t} & = \frac{25}{3}       \\
        \overline{t} & = 8.3
    \end{align*}
\end{highlight}

\paragraph{Turnaround Time}\label{par:turnaround_time_2}

The turnaround time of a process can be calculated using the general formula:
\[
    \text{turnaround time} = \text{completion time} - \text{arrival time}
\]
Here are some examples using the data from our examples:
\begin{highlight}{Round robin: turnaround times}
    \begin{align*}
        \text{Process \(1\)} & = 8 - 0 = 8   \\
        \text{Process \(2\)} & = 30 - 4 = 26 \\
        \text{Process \(3\)} & = 26 - 5 = 21
    \end{align*}
\end{highlight}
