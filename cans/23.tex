\section{Memory and Storage Management}\label{sec:memory_and_storage_management}

\subsection{Memory Management}\label{sub:memory_management}

Use the external storage (HDD, SSD) as ``add-on memory'' and keep only a \emph{subset} of the data of each process in the main memory and the rest on external storage.
Then, swap data in and out of memory to give each process the illusion that it has the entire system memory to work with.
Each process has its own \emph{virtual memory}.

The concept of a logical address space and a physical address space is central to proper memory management.
\begin{description}
    \item[Logical address] Also called a virtual address.
        This is generated by the CPU and is where the process thinks the data is.
    \item[Physical address] This is where the data actually is.
        The requests to the logical address are translated to physical addresses by the CPU.
\end{description}
When we transfer data between virtual and physical memory, we transfer fixed sized partitions called \emph{pages} instead of just words in order to save time.

\subsection{Storage Management}\label{sub:storage_management}

Here we are managing external storage to allow us to have data persistence.
The OS allows us to create, edit, delete files and directories (also backing up files and mapping files and permissions).

\subsubsection{OS and File Permissions}\label{ssub:os_and_file_permissions}

There are system calls to create files, delete files, move files, rename files, \dots.
\begin{itemize}
    \item When we \mintinline{python}{open(f)} files we search the directory structure on disk for \(f\) and move the contents to memory.
    \item When we \mintinline{python}{close(f)} we move the content of \(f\) in memory back to the disk again.
\end{itemize}

\subsection{File Protection}\label{sub:file_protection}

File owner/creator should be able to control what can be done by whom.
Types of access: read, write, execute, append, delete.

\section{More on OS Services}\label{sec:more_on_os_services}

\subsection{System Calls}\label{sub:system_calls}

\emph{System Calls} are programming interfaces to the services provided by the operating system.
We generally use a library that is build upon the direct system calls instead of using them directly.
A system call changes the processor from running in \emph{user mode} into \emph{kernel mode} where the operating system controls data on the CPU directly.
The CPU can do very low level operations with data directly on the CPU.

\subsection{Communication}\label{sub:communication}

We can use \emph{Inter-process Communication (IPC)} to allow processes to be either independent of cooperative.
Cooperative processes can be affected by other processes including sharing data.
There are some advantages to cooperative processes:
\begin{itemize}
    \item Information sharing
    \item Computation speedup (you don't need to write to external storage to transfer data)
    \item Modularity (can break down a large process)
    \item Convenience
\end{itemize}

\begin{note}
    Processors do much more:
    \begin{itemize}
        \item Multi-user services
        \item Real-time Operating Systems
        \item Managing multi-processor systems
        \item Managing distributed storage
        \item Virtualisation and hypervisors.
    \end{itemize}
\end{note}

