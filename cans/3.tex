\section{Binary Addition}\label{sec:binary_addition}

Simply add the \(0\)s and the \(1\)s together as in the example below.

\begin{highlight}{Binary addition example for \(3\) bits}
    \begin{tabular}{ccc|cc|c}
        \toprule
        \(x\) & \(y\) & \(z\) & Carry & Sum & Decimal \\
        \midrule
        0     & 0     & 0     & 0     & 0   & 0       \\
        0     & 0     & 1     & 0     & 1   & 1       \\
        0     & 1     & 0     & 0     & 1   & 1       \\
        0     & 1     & 1     & 1     & 0   & 2       \\
        1     & 0     & 0     & 0     & 1   & 1       \\
        1     & 0     & 1     & 1     & 0   & 2       \\
        1     & 1     & 0     & 1     & 0   & 2       \\
        1     & 1     & 1     & 1     & 1   & 3       \\
        \bottomrule
    \end{tabular}
\end{highlight}

To add longer \(8\) bit numbers simply repeat the process many more times.

\begin{highlight}{Binary addition example for \(8\) bits}
    \begin{figure}[H]
        \centering
        \begin{addtable}
            \binone & 0 & 0 & 1 & 0 & 1 & 1 & 0 & 1 \\
            \bintwo & 0 & 1 & 0 & 0 & 1 & 1 & 1 & 0 \\
            \divrule
            \bincar & 0 & 0 & 0 & 0 & 1 & 1 & 0 & 0 \\
            \rowstyle{\bfseries}
            \binsum & 0 & 1 & 1 & 1 & 1 & 0 & 1 & 1 \\
        \end{addtable}
        \label{tbl:eightbit_addition}
    \end{figure}
\end{highlight}

\section{Two's Complement}\label{sec:two_s_complement}

When we add and subtract in the decimal system, the logic used for both operations is different; this would be impractical for computers.

For two's complement to work, the number of bits used \emph{must} be specified.
Because half of the numbers used are negative, the possible range of numbers is half of that possible in the binary numbering system.
This range is defined as \(-2^{n-1} \le x \le 2^{k-1} - 1\).

To subtract two numbers, simply negate (flip the bits then add \(1\)) the subtracting number then add using the method above, ignore any overflow.

To find the positive representation of a negative number, simply negate.
An alternative method is to treat the most significant bit as a negative version of what you would expect then just add normally.

\section{Hexadecimal Notation}\label{sec:hexadecimal_notation}

Hexadecimal values can be used to represent bytes and words in a easier to understand way than a long string of \(0\)s or \(1\)s since each \(4\) bits directly corresponds to \(1\) hexadecimal character.

\begin{highlight}{Hexadecimal notation}
    \begin{tabular}{c|c|c|c}
        0001 & 1010 & 1111 & 0000 \\
        1    & A    & F    & 0
    \end{tabular}
\end{highlight}
