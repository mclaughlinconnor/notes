\section{Introduction to the Network Layer (L3)}\label{sec:introduction_to_the_network_layer}

The network layer addresses the problem of building networks of networks.
Previously, we have had point-to-point and multiple-access networks with hubs and switches to create a local network.

The Network Layer is the first end to end layer in the OSI layer meaning it is responsible for the end to end delivery of data across many link-layer hops and autonomous systems.
We can create an inter-network network (or internet).

\subsection{Components}\label{sub:components_}

\subsubsection{Common end-to-end network protocol}\label{ssub:common_end_to_end_network_protocol}

A common end-to-end protocol provides a single seamless service to interface with the transport layer above it by delivering data packets (or by provisioning circuits in a circuit switched network) and addressing each end system.

\subsubsection{Routing Devices}\label{ssub:routing_devices}

A routing device will connect one network to another by implementing the common network protocol.
A gateway or router will hide differences in the link layers between the networks:
\begin{itemize}
	\item Framing
	\item Addressing
	\item Flow control
	\item Error detection and correction
\end{itemize}
with the least amount of effort possible.

\subsection{The Internet}\label{sub:the_internet}

\begin{description}
	\item[1965] Concept of packet switching
	\item[1969] Wide-area packet networks
	\item[1973] First non-US ARPANET sites
	\item[1974] Initial version of the internet protocol
	\item[1981] Access to ARPANET broadened to non-DARPA funded sites.
	\item[1983] Switched to IPv4
	\item[1992] Development of IPv6 starts
\end{description}

\section{The Internet Protocol}\label{sec:the_internet_protocol}

The internet protocol is a common abstraction layer to interface with the transport protocols above it and the data link (and physical) layers below it.
\begin{itemize}
	\item Simple
	\item Best effort
	\item Connectionless
	\item Packet delivery
\end{itemize}
Providing:
\begin{itemize}
	\item End-to-end global addressing
	\item Routing
	\item Fragmentation and reassembly (breaking up packets)
\end{itemize}
Key concepts:
% TODO: make subsections
\begin{itemize}
	\item Global inter-network protocol
	\item Hour glass protocol stack:
	      \begin{itemize}
		      \item Many transport and application layers
		      \item A single network layer protocol:
		            \begin{itemize}
			            \item Packet switched, best effort service
			            \item Uniform network and host addressing
			            \item Uniform end-to-end connectivity
			            \item Range of link-layer technologies
		            \end{itemize}
	      \end{itemize}
\end{itemize}

\subsubsection{Basic Routing and Delivery}\label{ssub:basic_routing_and_delivery}

The packets are just sent, there is no permanent connection.
There are no guarantees made so packets could be delayed, dropped, corrupted or duplicated.
Any packets that cannot be delivered are just discarded.

\subsection{The Internet Protocol Packet}\label{sub:the_internet_protocol_packet}

Every network interface on every host is intended to have unique address (these are internet protocol addresses or ``logical addresses'').

\begin{description}
	\item[IPv4] These are \(32\) bit addresses (\(130.209.241.197\)) which is made up of \(4\) sets of \(8\) bit decimal numbers.
		There are not very many addresses.
	\item[IPv6] These are \(128\) bit addresses which are \(4\) sets of \(4\) hexadecimal digits.
\end{description}

\subsection{Fragmentation}\label{sub:fragmentation}

The maximum frame size of the link layer is \(1500\) bytes so IPv4 routers will split up (fragment) packets which are larger than this (MTU size).
A special bit is set (MF) is set if more fragments are still to follow.

In IPv6 there is no support for fragmentation in transit so packets must be split up on the hosts themselves.

\subsection{Loop Protection}\label{sub:loop_protection}

Some packets could loop forever and just take up processing power so packets include a forwarding limit which is a value (usually \(64\) or \(128\)) that each router reduces by \(1\) until it is discarded at \(0\).

\subsection{Header Checksum}\label{sub:header_checksum}

IPv4 header contains a checksum to detect and corruption introduced during transmission.
This is quite similar to the link-layer checksum, but uses a different algorithm.
\begin{note}
	This exclusively protects the header -- the data should be protected in an upper layer protocol.
\end{note}
\noindent
IPv6 does not contain a checksum and assumes the header and data will be protected on the link later.

\subsection{Transport Layer Protocol Identifier}\label{sub:transport_layer_protocol_identifier}

Network layer packets include the transport layer data as a payload:
\begin{itemize}
	\item TCP (\(6\))
	\item UDP (\(17\))
	\item DCCP (\(13\))
	\item ICMP (\(1\))
\end{itemize}
These are managed by the IANA.

\subsection{Differentiated Services}\label{sub:differentiated_services}

End systems can request special services from the network.
\begin{itemize}
	\item Emergency traffic should have priority
	\item Gaming or telephony prefer low latency, but don't need high bandwidth
	\item Background software might ask for low priori
\end{itemize}
You can signal your preference using a \emph{differentiated service code point (DSCP)} field in the header.
These are only a hint and can often be stripped away at network boundaries.
There can be difficult issues like ``who is allowed to set the DSCP?'' and ``what are people charged for setting the DSCP''?

\subsection{IPv4 or IPv6}\label{sub:ipv4_or_ipv6}

IPv4 has reached its end of life since there are no addresses left.

IPv6 is supposed to be a long-term replacement for IPv4.
The primary goal is to increase the address size to allow more hosts on the network.
An additional benefit is that the protocol is simple which makes high speed implementations easier.

It is not yet clear if IPv6 will be widely deployed, but it is very easy to build applications that use both.
A DNS query will return a IPv6 address if possible.

\section{IP Addressing}\label{sec:ip_addressing}

There are a few challenges:
\begin{enumerate}
	\item Is the address an identify or a location? Does it name the host or the location where it is?
	\item How should addresses be allocated? Do we use a hierarchical system or just allocated them flat?
	\item What is the address format? A number or text? Human readable? Fixed or variable length?
\end{enumerate}

\subsection{Identity and Location}\label{sub:identity_and_locaiton}

An address can denote the identity of a host.
A host should have a consistent address, irrespective of where or when they attach to the network to simplify upper-layer protocols (transport layer and applications don't care if you move about since the IP address doesn't change).
Complexity is moved to the network layer.

The network layer must know the location of the destination host so it can send data to it.
Sometimes a database is used to attach an identity to its address.

Alternatively, an address can indicate the location which a host attaches to the network.
This simplifies the network layer since it can send data directly to a host, but makes the higher layers more complex since they have to handle the changing IP addresses.

\subsection{Address Allocation}\label{sub:address_allocation}

We can have a hierarchical system where there is a common prefix or suffix for a set of machines.
This requires a rigid control of allocations.

If we just use a flat allocation, we cannot filter by an are like we can with a hierarchical system, but allocations can happen much more flexibly.

\subsection{Address Format}\label{sub:address_format}

Fixed length binary is easier for machines to read, but variable length text is better for humans.

\subsection{IP Addresses}\label{sub:ip_addresses}

IP addresses:
\begin{itemize}
	\item Specify the location of a network interface
	\item Are allocated hierarchically
	\item Are of a fixed size (either \(32\) bits or \(128\) bits).
\end{itemize}
A host with several network interfaces will have an IP address for each interface (like for Wi-Fi and Ethernet).
Both IPv4 and IPv6 addresses are hierarchical, and encode location by splitting addresses into a host and a network.
\begin{description}
	\item[Netmast] Describes the number of bits in the network part as a mask (\(1\) is a network,  \(0\) is a host).
	\item[Network] The network itself has the address wit the host part equal to \(0\).
	\item[Broadcast] The broadcast address has the address with the host section all \(1\)s.
\end{description}

\subsubsection{IP Address Management}\label{ssub:ip_address_management}

\paragraph{IPv4}\label{par:ipv4}

IPv4 only has \(2^{32}=4,294,967,296\) addresses.
The IANA administers the pool of unallocated addresses.
Addresses assigned to Regional Internet Registries (RIRs) can then assign to ISPs and large enterprises within their region (ISPs can then allocate to their customers)

\paragraph{IPv6}\label{par:ipv6}

IPv6, if deployed, will solve our address shortage issue for a \emph{long} time.
There are \(2^{128}=340,282,366,920,9938,463,463,374,607,431,768,211,456\) total addresses.

\subsection{IPv6 Deployment Issues}\label{sub:ipv6_deployment_issues}

IPv6 requires changes to be made to every host, router, firewall and application.
So far, most backbone routers support IPv6 and well as most operating systems, but there is still lots to go.

