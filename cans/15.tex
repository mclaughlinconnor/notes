\section{Working with Variables in Memory}\label{sec:working_with_variables_in_memory}

There are two ways of working with variables in memory:
\begin{description}
	\item[Statement-by-statement] Each statement is compiled independently: load then arithmetic then store. This is straightforward but inefficient (because memory transfers are slow).
		\begin{itemize}
			\item Each statement is compiled separately
			\item Each statement loads, computes, then stores
			\item A variable may be in several different registers
			\item There may be lots of redundant loading and storing code
			\item \emph{Object code} will match source code, but will be very long.
		\end{itemize}
	\item[Register variable style] Keep variables in registers access a group of statements. You don't need so many loads and stores, but you do need to keep track of which variables are in which registers (so you must use comments).
		\begin{itemize}
			\item The instructions corresponding to the statements are mixed together
			\item Some statements are executed entirely in the registers
			\item A variable is kept in the same register across many statements
			\item The use of loads and stores is minimal
			\item Object code is concise, but hard to read and match to source code
		\end{itemize}
\end{description}

\begin{note}
	It is possible to mix and match, but you are better to just use register variable style wherever possible.
\end{note}

\section{Arrays}\label{sec:arrays}

\subsection{Address Arithmetic}\label{sub:address_arithmetic}

We are going to start doing computations with addresses by treating them as variables stored in memory.
Addresses are unsigned numbers going from \(0\) to \(65535\) represented in binary format.

\subsubsection{What are the uses}\label{ssub:what_are_the_uses}

\begin{itemize}
	\item Powerful data structures
	      \begin{itemize}
		      \item Arrays
		      \item Pointers and records
		      \item Trees
		      \item Queues
		      \item Stacks
		      \item \dots
	      \end{itemize}
	\item Powerful control structures
	      \begin{itemize}
		      \item Input/output
		      \item Procedures and functions
		      \item Recursion
		      \item Methods
		      \item Case dispatch
	      \end{itemize}
\end{itemize}

\subsection{Arrays}\label{sub:arrays}

An arrays is represented in a computer by placing the elements in consecutive memory locations.
The array \mintinline{asm}{x} starts at \mintinline{asm}{$01a5} so the following elements are at \mintinline{asm}{$01a6}, \mintinline{asm}{$01a7}, etc.
To find the address of an element use:
\[
	x[i] = x+i
\]
An array is in memory along with other data (ie.\ must come after the \mintinline{asm}{trap}).

\begin{highlight}{Storing arrays in Sigma16}
	\begin{code}{asm}
		trap R0,R0,R0
		x data 13
		data 189
		data 870
		data 0
	\end{code}
\end{highlight}

\subsubsection{What about big arrays?}\label{ssub:what_about_big_arrays}

In large scale software arrays are allocated dynamically with the help from the operating system.
The user program calculates how big the arrays should be, the asks the OS for a block of memory that big at runtime.

\subsubsection{Accessing an element}\label{ssub:accessing_an_element}

We must use \emph{effective addresses}: \mintinline{asm}{result[R4]}, where the machine adds the displacement to the index register.

With effective addresses, you can access ordinary variables, or also an array element \mintinline{asm}{load R2,x[R8]}.
Suppose we want to execute \mintinline{python}{x[i] = x[i] + 50}, we must do:
\begin{highlight}{Accessing array elements in Sigma16}
	\begin{code}{asm}
		lea R1,50[R0] ; R1=50
		load R5,i[R0] ; R5=i
		laod R6,x[R5] ; R6 = x[i]
		add R6,R6,R1 ; R6 = x[i]+50
		store R6,x[R5] ; x[i] = x[i]+50
	\end{code}
\end{highlight}

\section{Array Traversal and For Loops}\label{sec:array_traversal_and_for_loops}

We want to perform a calculation on each element.

Usually we do this with a \mintinline{asm}{for} loop, but it can be done with a \mintinline{asm}{while} loop and an index too.

For loops are implemented in low level by creating a starting jump and an ending jump.
When the exit condition is truthy, we jump after the loop.

\subsection{Example Program: Array Max}\label{sub:example_program_array_max}

\begin{highlight}{Example array program in low level code}
	\begin{code}{python}
		max = x[0]
		i = 1
		forloop:
		if i >= then goto done
		if x[o] <=  max then goto skip
		max = x[i]
		skip:
		i = i+1
		goto forloop
		done:
		terminate
	\end{code}
\end{highlight}

\begin{highlight}{Example array program in Sigma16}
	\begin{code}{asm}
		lea R1,1[R0] ; R1 = constant 1
		load R2,n[R0] ; R2 = n
		load R4,x[R0] ; max = x[0]
		lea R3,1[R0] ; i = 1

		forloop ; top of loop, determine whether to stay in lopp
		; if i >= n then goto done
		cmp R3,R2 ; compare i, n
		jmpge done[R0] ; if i >= then goto done

		; if x[i] <= max then goto else
		load R5,x[R3] ; R5=x[i]
		cmp R5,R4 ; compare x[i],max
		jumple skip[R0] ; if x[i] <= max then goto skip

		add R4,R5,R0 ; max = x[i]

		skip
		add R3,R4,R1 ; i = i + 1
		jump forloop[R0] ; goto forloop

		done ; exit from for loop
		store R4,max[R0]
		trap R0,R0,R0 ; terminate
	\end{code}
\end{highlight}

\begin{note}
	To copy one register to another register, we just do \mintinline{asm}{add R4,R5,0}.
\end{note}

