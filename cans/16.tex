\section{More on compilation patterns}\label{sec:more_on_compilation_patterns}

\subsection{Why follow the patterns?}\label{sub:why_follow_the_patterns_}

\begin{itemize}
	\item Helps you understand what the high level language constructs mean.
	\item It is essentially how compilers work
	\item Saves time because it is quicker to catch errors at the highest level and it makes the code more readable (and therefore easier to debug).
	\item Leads to good comments to make more readable it.
	      The approach scales up to large programs
	\item Experienced programmers will recognize the patterns so the code will be easier to read and maintain.
\end{itemize}

\subsection{How do you know if you are using the patterns?}\label{sub:how_do_you_know_if_you_are_using_the_patterns_}

Each pattern contains:
\begin{description}
	\item[Changeable parts] Boolean and integer expressions and statements
	\item[Fixed parts] GOTO and if-then-GOTO
	\item[Labels] have to be different every time, but the structure of the fixed parts never changes.
\end{description}

\section{Records and Pointers}\label{sec:records_and_pointers}

\subsection{Records}\label{sub:records}

A record contains several fields that can be accessed with a dot operator.
Some languages call it a struct or dictionary.

\subsubsection{How do you do records in Sigma16}\label{ssub:how_do_you_do_records_in_sigma16}

To store data in a record do this:
\begin{minted}[linenos,numbersep=5pt,frame=lines,framesep=2mm]{asm}
x ; just a label to mark the starting address of x
x_fieldA data 3 ; offset 0 from x
x_fieldB data 1 ; offset 1 from x
\end{minted}
To retrieve data from a record in a bad way:
\begin{minted}[linenos,numbersep=5pt,frame=lines,framesep=2mm]{asm}
load R1,x_fieldB[R0]
load R2,x_fieldC[R0]
add R1,R1,R2
store R1,x_fieldA[R0]
\end{minted}
To do it better, we will use pointers.

\subsection{Pointers}\label{sub:pointers}

An alternative way to find data in memory is to make a pointer to it.
A pointer is an address which is available to us as a variable (an label like we used before is static -- we can't do operations on it).
We have a few new fancy symbols that we have to use now:
\begin{description}
	\item[The ampersand: \mintinline{c}{&x}] is used symbolically for the address of \mintinline{c}{x}
	      \begin{itemize}
		      \item You can apply to \mintinline{c}{&} operator to any variable (but not a complex expression).
		      \item \mintinline{c}{&x} is okay.
		            \mintinline{c}{&(3*x)} is not okay.
	      \end{itemize}
	\item[The asterisk: \mintinline{c}{*p}] means the value that \mintinline{c}{p} points to.
	      The \mintinline{c}{*} operator can be used on any pointer.
\end{description}

\subsubsection{Expressions using pointers}\label{ssub:expressions_using_pointers}

\paragraph[The & operator]{The \mintinline{c}{&} operator}\label{par:the_ampersand_operator}

The \mintinline{c}{&} operator give the address of its operand:
\mintinline{c}{p=x} puts the value of  \mintinline{c}{x} into \mintinline{c}{p}.
\mintinline{c}{p=&x} puts the address of \mintinline{c}{x} into \mintinline{c}{p} (the address of \mintinline{c}{x} is a pointer to \mintinline{c}{x} and we say ``\mintinline{c}{p} points at \mintinline{c}{x}'')

\paragraph[The * operator]{The \mintinline{c}{*} operator}\label{par:the_asterisk_operator}

The \mintinline{c}{*} operator follow a pointer and give what ever it points to.
\mintinline{c}{*p}  is an expressions whose value if what \mintinline{c}{p} points to.
\mintinline{c}{y=p} stores \mintinline{c}{p} into \mintinline{c}{y} so if \mintinline{c}{p} pointed to \mintinline{c}{x}, then now \mintinline{c}{y} will too.
\mintinline{c}{y=*p} follow the pointer to get the value into \mintinline{c}{y}.

\subsubsection{Pointers in Sigma16}\label{ssub:pointers_in_sigma16}

\paragraph{Getting the address pointers}\label{par:getting_the_address_pointers}

\begin{minted}[linenos,numbersep=5pt,frame=lines,framesep=2mm]{asm}
lea R5,x[R0] ; load effective address of &x, R5=&x
...
load R6,y[R0] ; R6=&y
store R6,p[R0] ; p = &y, p now contains a pointer to the variable for use later
\end{minted}

\paragraph{Getting data from pointers}\label{par:getting_data_from_pointers}

\begin{minted}[linenos,numbersep=5pt,frame=lines,framesep=2mm]{asm}
load R7,p[R0] ; R7=p (load the pointer to a register)
load R8,0[R7] ; R8=*p
; start at displacement 0
; offset by the amount at R7
; ie. get the data pointed to by the address at R7
\end{minted}

\subsection[The flexibility of LOAD and LEA]{The flexibility of \mintinline{asm}{load} and \mintinline{asm}{lea}}\label{sub:the_flexibility_of_load_and_lea}

\subsubsection[LEA]{\mintinline{asm}{lea}}\label{ssub:lea}

\begin{itemize}
	\item Can load a constant into a register: \mintinline{asm}{lea R1,42[R0]}
	\item Can create a pointer: \mintinline{asm}{lea R2,x[R0]}
	\item \mintinline{asm}{lea} can do even more
\end{itemize}

\subsubsection[LOAD]{\mintinline{asm}{load}}\label{ssub:load}

\begin{itemize}
	\item To load a variable into a register: \mintinline{asm}{load R3,x[R0]}
	\item To access and array element: \mintinline{asm}{load R4,a[R5]}
	\item To follow a pointer: \mintinline{asm}{load R6,0[R7]}
\end{itemize}

\subsection{Examples of pointers}\label{sub:examples_of_pointers}

\paragraph{A simple addition in a high-level language}\label{par:a_simple_addition_in_a_high_level_language}

\begin{minted}[linenos,numbersep=5pt,frame=lines,framesep=2mm]{c}
x = x + 5
\end{minted}

\paragraph{The Sigma16 equivalent code}\label{par:the_sigma16_equivalent_code}

\begin{minted}[linenos,numbersep=5pt,frame=lines,framesep=2mm]{asm}
; Put a pointer to x into R3 which contains the address of x
; R3 = &x
lea R3,x[R0]

; Add 5 to whatever word R3 points to
lea R1,5[R0] ; R1 = 5 (constant)
load R4,0[R3] ; R4 = *R3
add R4,R4,R1 ; R4 = *R3 + 5
store R4,0[R3] ; *R3 = *R3 + 5
\end{minted}
\begin{note}
	This is equivalent to \mintinline{c}{*(&x)=*(&x) + 5}.
\end{note}
The benefits of this is that there is no direct references to a label, only to a memory address that contains a pointer to a label.
This means that we can reuse this block many times.

\subsection{Why are pointers useful}\label{sub:why_are_pointers_useful}

\begin{itemize}
	\item We can write a block of code that accesses variables thorough pointers
	\item We can reuse code by executing it with a pointer that points to different data
\end{itemize}

\subsection{Accessing a record using a pointer}\label{sub:accessing_a_record_using_a_pointer}

\begin{minted}[linenos,numbersep=5pt,frame=lines,framesep=2mm]{asm}
load R1,1[R3] ; We move 1 address forward, then increase by R3
; This is the same as increasing the offset from R3 by one.
\end{minted}

