\section{Operating Systems Introduction}\label{sec:operating_systems_introduction}
\begin{note}
	We should get a few definitions out of the way to begin.
	\begin{description}
		\item[System Software] The interface between hardware and a user application.
		\item[Operating System] The most important piece of system software on a computer.
	\end{description}
\end{note}
\noindent
There are three key roles of a computer:
\begin{itemize}
	\item Virtualisation (of processor and memory)
	\item Concurrency
	\item Persistence (storage and IO)
\end{itemize}

\subsection{Virtualisation}\label{sub:virtualisation}

When we drive car, we don't have to worry about how the engine in the car actually works.

Virtualisation is similar in that we interact with a simple abstraction of the hardware and not get bogged down with all of the details.
The other main goal of Virtualisation is to manage shared resources.

\subsubsection{Resource Sharing}\label{ssub:resource_sharing}

We share CPU time between many programs and also sharing memory (and external storage, IO, networking, etc.) since there is only one main memory, but many different programs.

