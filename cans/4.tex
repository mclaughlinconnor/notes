\section{Using Binary To Encode Non-Numeric Data}\label{sec:using_binary_to_encode_non_numeric_data}

Numbers can be used numerically (eg.\ \(23 + 17\)) or as non-numeric data to encode information (eg.\ PA3 3BT).
Non-numeric data can represent far more things and have a greater flexibility because arithmetic operations don't need to be applied.

Just like with binary counting \(V=2^n\), but you don't have to use all values.

\subsection{ASCII}\label{sub:ascii}

The \textbf{A}merican \textbf{S}tandard \textbf{C}ode for \textbf{I}nformation \textbf{I}nterchange encodes text with \(7\) bits which are made up of \(94\) printable character and \(34\) non-printable characters.

\begin{note}
	ASCII also encodes numbers, but here \(\textrm{``5''} \neq 5\).
\end{note}

\subsection{Unicode}\label{sub:unicode}

Unicode extends ASCII to \(65536\) universal characters from many different languages using \(16\) bit (\(2\) byte) words.

\subsection{Bitmaps}\label{sub:bitmaps}

Each pixel is encoded as a number representing the colour of that pixel.

\subsection{Vector Graphics}\label{sub:vector_graphics}

Vectors store instructions to draw the resultant image so can be scaled up or down infinitely.

\subsection{Words}\label{sub:words}

Although all of these encodings use binary words, the words themselves have no meaning unless we are told what type of data is encoded in them.

\section{Boolean Logic and Gates}\label{sec:boolean_logic_and_gates}

There are three basic boolean logic operators: AND (\(\land\), \(\cdot\)), OR (\(\lor\), \(\parallel\)) and NOT (\(\lnot\), \(\sim\), \(!\), \(\bar{x}\)).
Eg.\ \(A \cdot B\) is the same as \(A\) AND \(B\)
sr

\begin{note}
	\(1+1=2\), but also \(1+1=1\) so context is very important.
\end{note}

A \emph{truth table} is used to define the relationship between all inputs and outputs.
For \(n\) inputs, there are \(2^{n}\) outputs.

\subsection{Logic Gates}\label{sub:logic_gates}

Switches implement specific logic functions and can be called \emph{logic gates}.
In early computers, switches were controlled by magnetic fields from coils in relays to open or close paths.
Next vacuum tubes were used.
And now, we use transistors instead.

\subsubsection{AND}\label{ssub:and}

An AND gate has an output whenever both inputs are on.

\begin{minipage}{0.45\linewidth}
	\begin{figure}[H]
		\centering
		\begin{circuitikz}
			\draw
			node[and port] (and) {}
			(and.in 1) node[anchor=east] {A}
			(and.in 2) node[anchor=east] {B}
			(and.out) node[anchor=west] {Y};
		\end{circuitikz}
		\caption{The AND logic gate}
	\end{figure}
\end{minipage}
\hfill
\begin{minipage}{0.45\linewidth}
	\begin{figure}[H]
		\centering
		\begin{circuitikz}
			\draw (0,-1) coordinate (origin) to[battery] ++(0,2) to[nos=\(A\)] ++(2,0) to[nos=\(B\)] ++(2,0) to[lamp] ++(0,-2) -- (origin);
		\end{circuitikz}
		\caption{An example of the AND gate}
	\end{figure}
\end{minipage}

\subsubsection{OR}\label{ssub:or}

The OR gate has an output whenever either of the inputs are on.

\begin{minipage}{0.45\linewidth}
	\begin{figure}[H]
		\centering
		\begin{circuitikz}
			\draw
			node[or port] (or) {}
			(or.in 1) node[anchor=east] {A}
			(or.in 2) node[anchor=east] {B}
			(or.out) node[anchor=west] {Y};
		\end{circuitikz}
		\caption{The OR logic gate}
	\end{figure}
\end{minipage}
\hfill
\begin{minipage}{0.45\linewidth}
	\begin{figure}[H]
		\centering
		\begin{circuitikz}
			\draw (0,-1) coordinate (origin) to[battery] ++(0,2) to[short,-*] ++(0.5,0) coordinate (forkpoint)%to[nos=\(A\)] ++(2,0) to[nos=\(B\)] ++(2,0) to[lamp] ++(0,-2) -- (origin);
			(forkpoint) -- ++(0,-0.5) to[nos=\(B\)] ++(2,0) -| ++(0.5,0.5) coordinate (forkjoin)
			(forkpoint) -- ++(0,0.5) to[nos=\(A\)] ++(2,0) -| (forkjoin) -- ++(0.5,0) to[lamp] ++(0,-2) -- (origin);
		\end{circuitikz}
		\caption{An example of the OR gate}
	\end{figure}
\end{minipage}

\subsubsection{NOT}\label{ssub:not}

\begin{minipage}{0.45\linewidth}
	\begin{figure}[H]
		\centering
		\begin{circuitikz}
			\draw
			node[not port] (gate) {}
			(gate.in 1) node[anchor=east] {A}
			(gate.out) node[anchor=west] {Y};
		\end{circuitikz}
		\caption{The OR logic gate}
	\end{figure}
\end{minipage}
\hfill
\begin{minipage}{0.45\linewidth}
	The output is inverted to the output.
\end{minipage}

\subsubsection{BUFFER}\label{ssub:buffer}

\begin{minipage}{0.45\linewidth}
	\begin{figure}[H]
		\centering
		\begin{circuitikz}
			\draw
			node[buffer port] (gate) {}
			(gate.in 1) node[anchor=east] {A}
			(gate.out) node[anchor=west] {Y};
		\end{circuitikz}
		\caption{The BUFFER logic gate}
	\end{figure}
\end{minipage}
\hfill
\begin{minipage}{0.45\linewidth}
	The input is exactly the same as the output.
\end{minipage}

\subsubsection{NAND}\label{ssub:nand}

\begin{minipage}{0.45\linewidth}
	\begin{figure}[H]
		\centering
		\begin{circuitikz}
			\draw
			node[nand port] (gate) {}
			(gate.in 1) node[anchor=east] {A}
			(gate.in 2) node[anchor=east] {B}
			(gate.out) node[anchor=west] {Y};
		\end{circuitikz}
		\caption{The NAND logic gate}
	\end{figure}
\end{minipage}
\hfill
\begin{minipage}{0.45\linewidth}
	The output is on, unless both inputs are on.
\end{minipage}

\subsubsection{NOR}\label{ssub:nor}

\begin{minipage}{0.45\linewidth}
	\begin{figure}[H]
		\centering
		\begin{circuitikz}
			\draw
			node[nor port] (gate) {}
			(gate.in 1) node[anchor=east] {A}
			(gate.in 2) node[anchor=east] {B}
			(gate.out) node[anchor=west] {Y};
		\end{circuitikz}
		\caption{The NOR logic gate}
	\end{figure}
\end{minipage}
\hfill
\begin{minipage}{0.45\linewidth}
	The output is on only if both inputs are off.
\end{minipage}

\subsubsection{XOR}\label{ssub:xor}

\begin{minipage}{0.45\linewidth}
	\begin{figure}[H]
		\centering
		\begin{circuitikz}
			\draw
			node[xor port] (gate) {}
			(gate.in 1) node[anchor=east] {A}
			(gate.in 2) node[anchor=east] {B}
			(gate.out) node[anchor=west] {Y};
		\end{circuitikz}
		\caption{The XOR logic gate}
	\end{figure}
\end{minipage}
\hfill
\begin{minipage}{0.45\linewidth}
	Sometimes called an exclusive OR. The output is on if only one of the inputs is on.
\end{minipage}

\section{Boolean Algebra}\label{sec:boolean_algebra}

\subsection{Operations with Constants}\label{sub:operations_with_constants}

\begin{itemize}
	\item \(\mathbf{x \cdot 0 = 0}\): ANDing any variable with \(0\) gives \(0\).
	\item \(\mathbf{x \cdot 1 = x}\): ANDing any variable with \(1\) gives the original variable.
	\item \(\mathbf{x + 0 = x}\): ORing \(x\) with \(0\) gives \(x\).
	\item \(\mathbf{x + 1 = 1}\): ORing \(x\) with \(1\) gives \(1\).
\end{itemize}

\subsection{Idempotence}\label{sub:idempotence}

\emph{Indempotence} is a characteristic where performing the same action again and again results in the same answer.
For example:

\begin{itemize}
	\item \(x + x = x\)
	\item \(x \cdot x = x\)
\end{itemize}

\subsection{Commutativity}\label{sub:commutativity}

\emph{Commutativity} is where swapping the order of the inputs results in the same output.
For example:

\begin{itemize}
	\item \(x+y=y+x\)
	\item \(x \cdot y = y \cdot x\)
\end{itemize}

\subsection{Associativity}\label{sub:associativity}

The term \emph{associativity} refers to the idea that terms can be grouped in brackets in any order.
For example:

\begin{itemize}
	\item \(x + (y + z) = (x + y) + z\)
	\item \(x \cdot (y \cdot z) = (x \cdot y) \cdot z\)
\end{itemize}

\subsection{Logical Reasoning}\label{sub:logical_reasoning}

An example question might be: ``Show two boolean expressions always have the same value''.
To answer this, you can either use a truth table to brute force the answer, or you can use boolean algebra to make the two statements equal.

\subsubsection{Example}\label{ssub:example}

\begin{minipage}{0.40\linewidth}
	\begin{circuitikz}[scale=0.75]
		\draw
		node[and port] (and) {}

		node[or port, below right=of and] (or) {}

		node at ([xshift=-5mm]and.in 1) {\(0\)}
		node at ([xshift=-5mm]and.in 2) {\(p\)}

		(and.out) |- (or.in 1)

		node at ([xshift=-5mm]or.in 2) {\(q\)}
		;
	\end{circuitikz}
\end{minipage}
\hfill
\begin{minipage}{0.57\linewidth}
	\centering
	\begin{align*}
		(0 \cdot p) + q & = (p \cdot 0) + q &  & \text{Commutative law}    \\
		(0 \cdot p) + q & = 0 + q           &  & \text{Constant operation} \\
		(0 \cdot p) + q & = q               &  & \text{Constant operation}
	\end{align*}

	\begin{circuitikz}
		\draw node {\(q\)} (0.5,0) to[short] (2, 0);

	\end{circuitikz}
\end{minipage}
