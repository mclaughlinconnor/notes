\section{Combinational Circuits}\label{sec:combinational_circuits}

A ``combinational circuit'' is one that works immediately and without any clock input and are created by combining AND, OR and NOT gates.

\subsection{Deriving Boolean Expressions}\label{sub:deriving_boolean_expressions}

\begin{highlight}{An example truth table}
	\begin{figure}[H]
		\centering
		\begin{tabular}{ccc|c|c}
			\toprule
			\(A\) & \(B\) & \(C\) & \(F\) & Product term                                         \\
			\midrule
			\(0\) & \(0\) & \(1\) & \(1\) & \(\overline{A}\cdot \overline{B}\cdot C\)            \\
			\(0\) & \(1\) & \(0\) & \(1\) & \(\overline{A}\cdot B\cdot \overline{C}\)            \\
			\(1\) & \(1\) & \(1\) & \(1\) & \(A\cdot B\cdot C\)                                  \\
			\(0\) & \(0\) & \(0\) & \(0\) & \(\overline{A}\cdot \overline{B}\cdot \overline{C}\) \\
		\end{tabular}
		\label{fig:example_truth_table}
	\end{figure}
\end{highlight}

\noindent
Collect the product terms where \(F=1\) and AND them together for \(\overline{A}\cdot \overline{B}\cdot C + \overline{A}\cdot B\cdot \overline{C} +A\cdot B\cdot C\), then simplify.

\subsection{Addition and Subtraction}\label{sub:addition_and_subtraction}

Addition, subtraction and negation can all be performed by a ``ripple add'' circuit.
To add just \(2\) bits together, use a ``half adder''.
To add \(3\) bits (two numbers and a carry), use a ''full adder`` -- this is what we generally use.
In order to add \(2\) full words together, use a ``ripple add'' circuit

\subsubsection{Half Adder}\label{ssub:half_adder}

Adds \(2\) bits together.
Since the resultant output could be \(0\), \(1\) or \(2\), we need to use \(2\) bits for the output which are named ``carry'' and ``sum''.

\begin{highlight}{A half adder}
	\begin{minipage}{0.42\linewidth}
		\begin{figure}[H]
			\centering
			\begin{tabular}{ccccc}
				\toprule
				\(x\) & \(y\) & Result & Carry & Sum   \\
				\midrule
				\(0\) & \(0\) & \(0\)  & \(0\) & \(0\) \\
				\(0\) & \(1\) & \(1\)  & \(0\) & \(1\) \\
				\(1\) & \(0\) & \(1\)  & \(0\) & \(1\) \\
				\(1\) & \(1\) & \(2\)  & \(1\) & \(0\) \\
				\bottomrule
			\end{tabular}
			\label{fig:half_adder}

			\medskip
			The truth table of a half adder
		\end{figure}
	\end{minipage}
	\hfill
	\begin{minipage}{0.48\linewidth}
		\begin{figure}[H]
			\centering
			\begin{circuitikz}
				\draw
				node[and port] (and) {}
				node[xor port, below=of and] (xor) {}

				node[left=of and.in 1] (x) {\(x\)}
				node[left=of xor.in 2] (y) {\(y\)}

				(x.east) -- (and.in 1)
				(y.east) -- (xor.in 2)

				([xshift=4mm]x.east) node[circ] {} |- (xor.in 1)
				([xshift=8mm]y.east) node[circ] {} |- (and.in 2)

				(and.out) node[right] {Carry \(x \cdot y\)}
				(xor.out) node[right] {Sum \(\overline{x}\cdot y + x \cdot \overline{y}\)}
				;
			\end{circuitikz}
			\label{fig:half_adder_circuit}

			\medskip
			A half adder circuit diagram
		\end{figure}
	\end{minipage}
	\medskip
\end{highlight}

\subsubsection{Full Adder}\label{ssub:full_adder}

A ``full adder'' is used to add two binary numbers together with a carry from the addition to the left and outputs the sum and the carry where \(c = x\cdot y + y \cdot z + x\cdot z\) and \(s = x \oplus y \oplus z\)

\subsubsection{Word Addition}\label{ssub:word_addition}

To add two words together (of any size), you must use a full adder for each bit of the word with the carry output being input to the next adder's carry input.

\subsubsection{Subtraction}\label{ssub:subtraction}

To subtract, simply negate and add \(1\) to the subtracting number, then add the two numbers together.
