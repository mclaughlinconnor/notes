\section{Routing}\label{sec:routing}

In a network, nodes learn only a subset of the network topology and runs an algorithm to decide where to forward packets destined for other hosts.

\begin{itemize}
	\item End hosts (consumer devices) usually only know what is and isn't on your local network and use only a simple algorithm.
	      If a packet is for the outside network, packets are sent to the default gateway.
	\item Gateway devices (routers) exchange information to decide on the best route based on their knowledge of the network.
\end{itemize}

\subsection{Unicast Routing}\label{sub:unicast_routing}

We route to deliver packets from one source to one destination.
The choice of algorithm used can vary:
\begin{itemize}
	\item Intra-domain routing
	\item Inter-domain routing
	\item Politics and economics
\end{itemize}

\subsubsection{Inter-domain unicast routing}\label{ssub:inter_domain_unicast_routing}

Each network/internetwork is entirely autonomous and can have different policies and technologies.
These operate potentially with distrust.

\subsubsection{Intra-domain unicast routing}\label{ssub:intra_domain_unicast_routing}

Here we are routing inside of an autonomous system.
There is a single trust domain where there are no restrictions on which links can be used or who can determine the network topology.

With access to the whole network, you can make the best use of the network you have available to give the most efficient routing.

There are two approaches to routing packets in an intra domain system:
\begin{itemize}
	\item Link state
	\item Distance
\end{itemize}

\subsection{Inter-domain Routing}\label{sub:inter_domain_routing}

Here the goal is to find the best route to the destination network.
Each network is treated as a single node and routing is done ignoring the internal topology.

The network is made up of autonomous systems where each autonomous system is an independently administered network.

\begin{note}
	Some nodes can have several exit points (``not on the edge''), so there needs to be some logic here, or a ``routing table''.
\end{note}

\subsubsection{Edge Routing}\label{ssub:edge_routing}

Where you have multiple exits to a network, you have a \emph{default free zone} where there is no default exit point.
We must route based on policies and not the shortest route.
\begin{itemize}
	\item Use autonomous system \(x\) in preference to AS \(y\).
	\item Use AS \(x\) to reach only specific addresses
	\item Use the path that crosses the fewest ASes
	\item Avoid ASes located in a country.
\end{itemize}
All of these require a complete AS-level knowledge of the network.

We use a protocol named \emph{border gate way} protocol (BGP) which runs on specific routers to exchange data between ASes.
The individual ASes choose which internal routes to expose.
These are complex and policy driven.

\section{Intra-Domain Routing}\label{sec:intra_domain_routing}

Intra-domain routing works inside of a domain (any network or internetwork operated by a single entity: an autonomous system) and operates a single routing protocol which is typically the OSPF (open shortest path first) where you exchange routes to IP address prefixes (prefixes point to networks).

\subsection{Distance Vector Protocol}\label{sub:distance_vector_protocol}

Each node has a table containing the distance to each node in the network which is periodically exchanged with neighbours so eventually each node will know the distance to all other nodes (``converges'')

\begin{note}
	Nodes only update their own tables if there has been a change on another node.
\end{note}

At each node we store the following data for each node in the network:
\begin{itemize}
	\item Cost which is the length of the path to \(x\).
	\item The next hop which is the next node on the path to \(x\).
	      A \(-\) signifies an unknown hop and a \(\cdot\) signifies that no hop is needed.
\end{itemize}

\paragraph{Time Step One}\label{par:time_step_one}

\begin{itemize}
	\item Nodes are initialised and only know their immediate neighbours.
	\item All nodes update their own local rows with their neighbours too.
\end{itemize}

\paragraph{Time Step Two}\label{par:time_step_two}

\begin{itemize}
	\item Routing data has spread by one hop so we can reach nodes two hops away (each node knows its neighbours).
	\item Nodes report their route tables to their neighbouring nodes, which are merged together to find the shortest routes.
	\item At the end of the second time step, we can reach nodes three stops away.
\end{itemize}

\paragraph{Time Step Three}\label{par:time_step_three}

\begin{itemize}
	\item (From the example) routing data has spread two hops and the routing tables are now complete.
\end{itemize}

\paragraph{Time Step Four}\label{par:time_step_four}

\begin{itemize}
	\item Nodes continue to exchange distance metrics in case the network topology changes.
\end{itemize}


