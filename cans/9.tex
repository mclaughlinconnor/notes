\section{The CPU}\label{sec:the_cpu}

A CPU has a set of requirements that it must meet in order to function correctly:
\begin{itemize}
    \item Fetch instructions from memory
    \item Interpret those instructions
    \item Fetch data from memory
    \item Process data
    \item Write data back to memory again
\end{itemize}

\subsection{The CPU and the System Bus}\label{sub:the_cpu_and_the_system_bus}

The system bus is made up of several smaller buses:
\begin{itemize}
    \item \textbf{Control bus}: transfers control signals
    \item \textbf{Address bus}: transfers memory addresses about
    \item \textbf{Data bus}: transfers data to and from memory
\end{itemize}

\subsection{Internal structure of a CPU}\label{sub:internal_structure_of_a_cpu}

\begin{highlight}{The internal structure of a CPU}
    \begin{tikzpicture}
        \draw
        node (alu) {\textbf{ALU}}
        node[below=5mm of alu] (status) {Status flags}
        node[below=0mm of status] (shift) {Shifter}
        node[below=0mm of shift] (compl) {Complementer}
        node[below=0mm of compl] (abl) {Arithmetic and boolean logic}
        node[draw,fit=(alu) (status) (shift) (compl) (abl)] (alubox) {};

        \draw node [draw,shape=rectangle, minimum height=5cm, minimum width=0cm, anchor=center,right=of shift, xshift=20mm] (cpubus) {\rotatebox{90}{Internal CPU bus}};

        \draw[->] (status.east) -- (cpubus.west |- status);
        \draw[->] (shift.east) -- (cpubus.west |- shift);
        \draw[->] (compl.east) -- (cpubus.west |- compl);
        \draw[->] (abl.east) -- (cpubus.west |- abl);

        \draw
        node[right=5cm of alu] (reg) {\textbf{Registers}}
        node[below=5mm of reg] (rone) {\(r_1\)}
        node[below=0mm of rone] (rtwo) {\(r_2\)}
        node[below=0mm of rtwo] (rthree) {\(r_3\)}
        node[below=0mm of rthree] (rfour) {\(r_4\)}
        node[below=0mm of rfour] (rdots) {\(\cdots\)}
        node[draw,fit=(reg) (rone) (rtwo) (rthree) (rfour) (rdots)] (regbox) {};

        \draw[<-] (rone.west) -- (cpubus.east |- rone.west);
        \draw[<-] (rtwo.west) -- (cpubus.east |- rtwo.west);
        \draw[<-] (rthree.west) -- (cpubus.east |- rthree.west);
        \draw[<-] (rfour.west) -- (cpubus.east |- rfour.west);
        \draw[<-] (rdots.west) -- (cpubus.east |- rdots.west);

        \draw node [draw,shape=rectangle, anchor=center,below=of regbox] (cu) {Control Unit};
        \draw[<->] (cu) -- (regbox);

        \draw[<->, thick] (cu.west) -| (alubox.south) node[midway, below right] {Control paths};
        \draw[<->, thick] (cu.west) -| ++(-6mm, 14mm);
        \draw[<->, thick] (cu.west) -| ++(-6mm, 7mm) -- ++(-5mm, 0);
    \end{tikzpicture}
\end{highlight}

\subsection{What does an n-bit architecture mean?}\label{sub:what_does_an_n_bit_architecture_mean_}

It means that the word size of a processor is size \(n\).
Generally this means that instructions; internal registers; internal interconnects (register bus, data bus, control bus); and the ALU can handle numbers of that size.

\section{Registers}\label{sec:registers}

\subsection{New top down perspective}\label{sub:new_top_down_perspective}

CPU must have working space -- named registers
The number and functions of registers varies between processor designs -- this is a major design decision
Registers are at the very top of the memory hierarchy (and are the fastest)

\subsection{Register Organisation}\label{sub:register_organisation}

Registers can be ``user visible'' (general purpose or specialized) and store data the user knows about.
Registers can also be for control or status which only the CPU knows about.

\subsection{User Visible}\label{sub:user_visible}

Accessible through machine language and can be general purpose or more specialised:
\begin{itemize}
    \item data(accumulator stores output of accumulated output of alu)
    \item Address (segment pointer, index regigser, stack pointer)
    \item Condition codes(flags) which are user visible, but not writeable.
\end{itemize}
There is a trade-off between general purpose and specialised because a general purpose one cannot be optimized like a specialised one can be.

\subsection{How wide?}\label{sub:how_wide_}

Large enough to hold full memory addresses and full data words.
Often is its possible to combine two registers and store one piece of large data

\subsection{Control and status registers}\label{sub:control_and_status_registers}

Generally not user visible
\begin{itemize}
    \item program counter
    \item instruction register
    \item  memory address register
    \item memory bugger register
    \item program status word
\end{itemize}

\subsubsection{Program status word}\label{ssub:program_status_word}

Stores many separate bits together which are flags (like the result of the last program)
Can be read implicitly by programs (eg. jump if \(0\))
Cannot (usually be set by programs)
Can be considered both control/status and user visible.

\section{Fetch execute cycle}\label{sec:fetch_execute_cycle}

\emph{Functional view of processor} -- what sequence of processes need to be done to do something.

\subsection{Fetch cycle}\label{sub:fetch_cycle}

\begin{itemize}
    \item Program counter stores address of next instruction to fetch
    \item Processor fetches from memory at program counter
    \item Increment program counter(unless told not to)
    \item Memory loaded into instruction register
\end{itemize}
