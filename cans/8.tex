\section{Computer Organisation Introduction}\label{sec:computer_organisation_introduction}

\emph{Organisation} is the term used to refer to the physical circuits.
\emph{Architecture} is the term used to refer to the interface programmers can use (like instructions).

All programs have memory and data, each with their own addresses.

IO devices must use an IO module to speak to IO devices since the CPU is a very general device and is kept as simple as possible.
IO devices are \emph{memory mapped} meaning that the CPU can access certain memory addresses to tell the IO module to perform a task.

\section{Von-Neuman Machines (IAS)}\label{sec:von_neuman_machines_ias_}

Hardwired systems are inflexible (an ADDer will always ADD).
If we were to create a general system that uses control signals to do many different things we could extend a system with a new set of control signals to provide new functionality.

\subsection{IAS Machine}\label{sub:ias_machine}

There were four components that separated the IAS from computers before it:
\begin{enumerate}
    \item Memory stored both programs and data
    \item An ALU operated using binary data
    \item A control unit interprets and executes instructions from memory
    \item IO is controlled by a control unit
\end{enumerate}
%
Most computers have this architecture.
The von Neuman machine is a model saying:
\begin{itemize}
    \item There should be four main subsystems (above)
    \item Stored program concept: store programs in memory
    \item Sequential execution of instructions (fetch, decode, execute)
\end{itemize}
%
The von Neuman machines have a selection of special registers to use also:
\begin{itemize}
    \item \(PC\): Program Counter - stores the index of the current instruction
    \item \(IR\): Instruction Register - stores the current instruction
    \item \(MAR\): Memory Address Register - what memory is currently being read from
    \item \(MBR\): Memory Buffer Register - what data was just read.
    \item \(IO AR\): Input/Output Address Register - what IO memory register is currently being read from
    \item \(IO BR\): Input/Output Buffer Register - what data was just read from an IO register
\end{itemize}

\subsection{Instruction Cycle}\label{sub:instruction_cycle}

\begin{highlight}{The fetch-execute cycle}
    \begin{tikzpicture}
        \draw
        node (s) {Start}
        node[right=of s] (f) {Fetch}
        node[right=of f] (e) {Execute}
        node[right=of e] (h) {Halt};
        \draw[->] (s) -- (f);
        \draw[->] (f) -- (e);
        \draw[->] (e) -- (h);
        \draw[->] (e) -- ++(0,-5mm) -| (f);
    \end{tikzpicture}
\end{highlight}
\begin{enumerate}
    \item The program fetches the instruction at the program counter
    \item We increment the program counter (unless explicitly told not to when doing branches)
    \item Load the instruction to the instruction register
\end{enumerate}
%
There are four types of instruction: processor, memory; processor, IO; data processing; and control which may be combined together.
