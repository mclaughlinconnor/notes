\section{Synchronous Circuits}\label{sec:synchronous_circuits}

A \emph{synchronous circuit} uses synchronous logic (sometimes called sequential logic -- which is usually synchronous anyway -- or synchronous sequential logic -- which is the fancy correct way of saying it).
Synchronous circuits have the benefit of being able to store memory/state.


\subsection{Why use sequential logic?}\label{sub:why_use_sequential_logic_}

Consider the process:

\begin{enumerate}
    \item Multiply one bit by another
    \item Shift the bit over by one
    \item Add the bit to another bit
    \item Repeat three times
\end{enumerate}
%
A combinational circuit cannot be used to do these \(4\) steps in a sequence, since we must store the output bit after each step, so we must use sequential logic.

\subsection{Timing and Gate Delay}\label{sub:timing_and_gate_delay}

All gates and wires have a slight delay (which is caused by the finite speed of light and the reaction time of transistors) that can add up over the course of an entire circuit.
Different paths can take different amounts of time, so a \emph{clock signal} is used to synchronise all of the logic gates together.

\subsection{The Flip Flop}\label{sub:the_flip_flop}

A \emph{flip flop} is a bi-stable component which is stable in either the on or off state and provides a way to store memory to let us sequence operations together.
The flip flop introduces a whole new design paradigm.

\subsubsection{Understanding Feedback}\label{ssub:understanding_feedback}

\begin{highlight}{A flip flop}
    \begin{minipage}{0.50\linewidth}
        \begin{circuitikz}
            \draw
            node[not port] (notone) {}
            node[not port, right=of notone] (nottwo) {}
            node[right=of nottwo] (q) {\(Q\)}

            (notone.in) -- ([xshift=-5mm]notone.in) |- ([yshift=15mm]$(nottwo.out)!.5!(q)$) -- ($(nottwo.out)!.5!(q)$)
            (nottwo.out) -- (q)
            (notone.out) -- (nottwo.in);
        \end{circuitikz}
    \end{minipage}
    \hfill
    \begin{minipage}{0.40\linewidth}
        If \(Q\) is \(1\), it will always be \(1\) and if \(Q\) is ever \(0\), it will always be \(0\).
    \end{minipage}
\end{highlight}
There is one problem with this circuit though: it cannot be set or reset.
We can switch it out with some NOR and AND gates to allow for these missing features.

\begin{highlight}{A resettable flip flop}
    \begin{circuitikz}
        \draw
        node[nor port] (norone) {}
        node[nor port, below=of norone] (nortwo) {}

        node[left=of norone.in 1] (r) {Reset (\(R\))}
        node[left=of nortwo.in 2] (s) {Set \(1\) (\(S\))}

        node[right=of norone.out] (q) {Output \(Q\)}
        node[right=of nortwo.out] (qbar) {Inverted output \(\overline{Q}\)}

        (r) -- (norone.in 1)
        (s) -- (nortwo.in 2)

        (norone.out) -- (q)
        (nortwo.out) -- (qbar)

        (norone.out) node[circ] {} -- ([yshift=2mm]$(norone.out)!.5!(nortwo.out)$) -- ++(-2,0) |- (nortwo.in 1)
        (nortwo.out) node[circ] {} -- ([yshift=-2mm]$(norone.out)!.5!(nortwo.out)$) -- ++(-1.8,0) |- (norone.in 2)
        ;
    \end{circuitikz}
\end{highlight}

\subsubsection{Synchronicity with flip flops}\label{ssub:synchronicity_with_flip_flops}

Flip flops are synchronous bi-stable components.
\emph{Synchronous} means that the output changes only when a clock signal is supplied.

\subsubsection{The delay flip flop}\label{ssub:the_delay_flip_flop}

The \emph{delay flip flop} is a brand new primitive component that stores a single bit.
A DFF has a single input, a single output and a clock signal input.

Every flip flop must be connected to the same clock in order to be synchronous and the clock signal must arrive at all components at the same time.
\textbf{No logic is ever performed on the clock signal} -- it is strictly a control signal.

\subsection{Clock ticks and cycles}\label{sub:clock_ticks_and_cycles}

\emph{Ticks} are individual events in time (tick, tock, tick, \ldots).
\emph{Cycles} are the intervals between the ticks.

A flip flop will stay stable for the entire duration of a cycle.
During a cycle, all logic circuits are updated and their outputs propagate through the circuit until all of the outputs are stable at a valid value.
Only once all outputs are stable and valid can a new tick occur.

\subsubsection{Clock speed}\label{ssub:clock_speed}

\emph{Clock speed} is the rate at which the internal clock signal ticks and is measured in Hertz (Hz).
A typical processor has a clock speed on the order of magnitude \(1\)GHz which is \(1\times 10^{9}\)Hz or \(1\times 10^{-9}\) seconds per tick.
At these very fast speeds, the speed of light becomes important.

\section{The applications of flip flops}\label{sec:the_applications_of_flip_flops}

\subsection{Registers}\label{sub:registers}

A register generally words on \(8\) bit words.
In order to create an \(8\) bit register, we must use \(8\) individual flip flops together.
Registers have a CLR (clear) signal to empty all flip flops.

A \emph{bus} is a collection of signals all coming from or going to the same place.

\subsection{Finite state machines}\label{sub:finite_state_machines}

In combinational circuits, \(1\) input means \(1\) output.
In a finite state machine, the output depends on the current state as well as the current input.
After each operation, the state is updated with the current output.

An ATM is an example of a finite state machine where the machine will do different things when you put your card in depending on what screen you are on.
