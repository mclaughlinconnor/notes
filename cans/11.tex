\section{Computer Performance}\label{sec:computer_performance}

The amount of memory and clock speed has been increasing exponentially.
Clock speed has been increasing at a faster rate than memory capacity leading to the \emph{logic memory gap} which is one of the main processor bottlenecks.

\begin{itemize}
	\item You should increase the number of bits received at one time (make DRAM wider, not deeper because deeper is slower)
	\item Reduce frequency or memory access (have a more compelx cache and have cache on the chip).
	\item Increase interconnection bandwidth (use hierarchy of buses or increase bus speed, etc.).
\end{itemize}

\subsection{IO performance}\label{sub:io_performance}

Input and output devices are \emph{very} slow (keyboards only have \(10\)s of bits per second).
Use interrupts, multi-level bus structures and allow direct memory access.

\subsection{How to increase processor speed}\label{sub:how_to_increase_processor_speed}

\begin{itemize}
	\item Increase clock frequency by shrinking logic gates.
	\item Increase size and speed of caches
	\item Change processor organisation and architecture (parallelism)
\end{itemize}
%
Ideally all components should be able to work at their maximum performance.
Try not to let one component stall others.

\subsection{Performance Assessment}\label{sub:performance_asssessment}

\subsubsection{Clock speed}\label{ssub:clock_speed_one}

Clock speed cannot be arbitrary (ie. data should be reliably availbale at all inputs on a clock edge) so clock speed is not the whole story.

\subsubsection{Instruction execution rate}\label{ssub:instruction_execution_rate}

Average clock cycles per instruction of programs.
CPI alone won't tell us the execution time for a certain program.
We can find out number of clock ticks needed for \(n\) number of instructions, but we don't know what the duration of each clock cycle is.
\begin{align*}
	T = I_c \times CPI \times \tau
\end{align*}

\subsection{MIPS and MFLOPS}\label{sub:mips_and_mflops}

Millions of instructions per second (MIPS).
\begin{highlight}{MIPS rate formula}
	\begin{align*}
		\textrm{MIPS rate} = \frac{I_c}{T \times 10^6} = \frac{f}{CPI \times 10^6}
	\end{align*}
\end{highlight}
Millions of floating point instructions per second (MFLOPS).
Heavily dependent on instruction set, compiler design, processor implantation, cache and memory hierarchy.
MIPS and MFLOPS are bad for measuring performance.

\subsubsection{Why are the not good enough?}\label{ssub:why_are_the_not_good_enough_}

Modern processors are CISC (complex instruction set) which means one instruction can do complex tasks.
Some processors are RISC (reduced insturction set) and require several instructions per task which takes longer.

\subsection{Benchmarks}\label{sub:benchmarks}

\begin{itemize}
	\item Programs designed to test performance
	\item Written in high level languages (compile for different architectures)
	\item Represents style of task (systems, numerical, commercial)
	\item Is easily measured (probably by time)
	\item Is widely distributed and probably open source
	\item eg. System Performance Evaluation Corporation (SPEC)
\end{itemize}

