\section{Introduction to Networks}\label{sec:introduction_to_networks}

\subsection{Why learn about computer networks?}\label{sub:why_learn_about_computer_networks_}

\begin{itemize}
    \item   How are systems connected together to form a network?
    \item How are networks connected?
    \item How does data go through the network to its destinations?
    \item How can we have reliable communications through unreliable links?
    \item What are the repercussions for programs that communicate over this infrastructure?
\end{itemize}

\subsection{Roles with Respect to Network Systems}\label{sub:roles_with_respect_to_network_systems}

\begin{description}
    \item[Users] The people that use a network don't have to worry about intricacies.
    \item[Manager] Manage a network
    \item[Application Designers] This is probably us. We have to worry about some details.
    \item[Designers of Systems] Have to have a deep knowledge of networks.
\end{description}

\section{Networked Systems}\label{sec:networked_systems}

Autonomous computing devices that exchange data to perform a goal.The exchange of data is visible to the application.
However, a computer system is only aware of the network (this is similar to how a computer writes to an IO device).

There is no single computer (or set of computers) that needs to organise data over a network (\emph{autonomous}).
Traffic on roads manages itself, but follows set rules.

\subsection{Key Aspects}\label{sub:key_aspects}

\begin{description}
    \item[Communication] How is information exchanged across a single link?
    \item[Networking] How are links interconnected to build a network?
    \item[Networked Systems] How do systems communicate across the network?
\end{description}

\subsection{Applications Using Networks}\label{sub:applications_using_networks}

Some applications use The Internet World Wide Web:
\begin{itemize}
    \item Email
    \item Social Networks
    \item Streaming audio and video
    \item File sharing
    \item Instant messaging
\end{itemize}
Others don't:
\begin{itemize}
    \item Digital TV
    \item Mobile voice telephony
    \item Sensor networks
    \item Controller area networks connecting sensor and other components inside vehicles or aircraft (eg. on board computers in car)
\end{itemize}

\subsection{Communication}\label{sub:communication_}

Messages are transferred from source to destination via a communication channel.
The size of messages might be bounded.

\paragraph{Information Required For Communications}\label{par:information_required_for_communications}

\begin{itemize}
    \item Communication might be \emph{simplex} (one way), \emph{half-duplex} (both ways, one at a time), \emph{full-duplex} (both ways all of the time).
    \item We want to communicate data; messages convey information.
    \item How much information is in a message? We can calculate this mathematically (not covered here).
    \item What is the information carrying capacity of a channel?
    \item Capacity of channels to convey information can be modelled using how much power they use, how much they can transfer and how quickly they can do it.
    \item Physical limits exist on the capacity of a channel.
\end{itemize}

\subsubsection{Physical Forms of Messages}\label{ssub:physical_forms_of_messages}

A physical form of a message can be a material object (carrier pigeon, CD, etc.), but is usually a wave (sound, electrical, etc.) and may be analogue or digital.

\paragraph{Analogue Signals}\label{par:analogue_signals}

The simplest analogue signal means that amplitude directly codes the value of interest: louder voice means louder signal, AM radio, analogue telephones.

These can be arbitrarily accurate, but are susceptible to noise and interference and are difficult to process with digital electronics.

\paragraph{Digital Signals}\label{par:digital_signals}

These comprise a sequence of symbols and not arbitrary values (the underlying channel can still be analogue though and modulation will be used to convert the signal).

Computer systems use binary encoding much like networked systems too.

\begin{description}
    \item[Bit Rate] The bits per second
    \item[Baud Rate] The symbols per second (one symbol is not one bit).
\end{description}

\paragraph{Analogue to Digital Conversion}\label{par:analogue_to_digital_conversion}

\begin{enumerate}
    \item Sample the signal at a suitable rate (higher is better).
    \item Quantise each value to the nearest allowable discrete value.
    \item Convert to a digital representation.
\end{enumerate}
The \emph{sampling theorem} determines that rate at which the signal must be sampled for accurate reconstruction.

\section{Channels, Links, Networks and Switching}\label{sec:channels_links_networks_and_switching}

\begin{description}
    \item[Hosts] The source and destination for messages
    \item[Signal] A signal is conveyed via a channel and may be directly conveyed on a cable, or it may be modulated onto an underlying carrier.
        There will probably be many links that the channel has to go through.
    \item[Link] A link directly connects one or more nodes together.
    \item[Network] A network comprises several links connected together.
        The devices connecting the links are called their switches or routers.
\end{description}

\subsection{Network Switching}\label{sub:network_switching}

Network switching describes how devices can share a single wire to communicate.

\subsubsection{Circuit Switched Networks}\label{ssub:circuit_switched_networks}

A dedicated circuit can be set-up between two points.
There the two devices can exchange arbitrary length messages.
There is a guaranteed capacity once the network is created, but the dedicated circuit can block other communications from other devices.

\subsubsection{Packet Switched Networks}\label{ssub:packet_switched_networks}

Messages are split into small packets (small, size constrained chunks) before transmission.
Allows more than one pair of devices to communicate.
Connectivity is guaranteed, but the average capacity varies depending on how many people are using the network.

\section{Protocols and Layers}\label{sec:protocols_and_layers}

\subsection{Network Architecture}\label{sub:network_architecture}

A network architecture is organised into layers in order to manage the complexity of network communications (encoding conversion, route to take, fiber network conversions, \dots).

\subsection{Layers}\label{sub:layers}

\begin{enumerate}
    \item Application programs
    \item Process-to-process channels (IRC to IRC)
    \item Host-to-host connectivity
    \item Hardware (what happens in the wires?)
\end{enumerate}

\subsection{Network Protocols}\label{sub:network_protocols}

Each layer in the network architecture will have its own protocol.
To be useful, the messages need to follow a set syntax and have agreed semantics.
Protocols have an agreed language for encoding message and rules for what messages mean and when they can be sent.
Protocols define the behaviour of the network too through the interactions of devices.

\subsubsection{Interfaces}\label{ssub:interfaces}

Each protocol object (an implemented protocol) has two interfaces:
\begin{description}
    \item[Service Interface] Performs the actual operations on this protocol.
    \item[Peer-to-peer Interface] Messages that are exchanged with the peer.
        This is how services are requested.
        The service interface sends messages through this.
\end{description}

\subsubsection{Protocol Data Units}\label{ssub:protocol_data_units}

A PDU defines what messages are legal to send.

\paragraph{Syntax}\label{par:syntax}

A protocol will have different types of message called Protocol Data Units (PDUs).
Each type of PDU will have a syntax, etc., that describes what information is included and whether it is textual data or binary data.

\begin{description}
    \item[Textual PDUs] These have a syntax and grammar that describes their format (similar to how programming languages have a grammar).
    \item[Numeric PDUs] These have rules too like is the data big or little endian? \(32\) or \(64\) bit? Fixed or variable length?
\end{description}

\paragraph{Semantics}\label{par:semantics}

Protocol semantics defined when PDUs can be sent and what response is needed.
\begin{itemize}
    \item Who can send PDUs
    \item Roles for hosts (client and server, peer-to-peer)
    \item What are entities and how are they named?
    \item How are errors handled?
\end{itemize}
A state transition diagram can be used to indicate the stages of a protocol operation and the transitions that occur in response from PDUs with what can be sent back.

