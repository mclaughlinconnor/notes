\section{Arrays with Pointers}\label{sec:arrays_with_pointers}

Because we know about pointers now, we can use addresses as variables (pointers) and perform arithmetic on pointers.

\subsection{Accessing an Array Element Using Pointers}\label{sub:accessing_an_array_element_using_pointers}

Create a pointer to the beginning of the array \(x\) so \mintinline{python}{R1} points to \mintinline{python}{x[0]}.

\begin{minted}[linenos,numbersep=5pt,frame=lines,framesep=2mm]{asm}
lea R1,x[R0] ; R1 = x[0]
\end{minted}

\noindent
To access the current element (eg.\ element at index \(0\)), follow the pointer in \mintinline{python}{R1}:

\begin{minted}[linenos,numbersep=5pt,frame=lines,framesep=2mm]{asm}
load R1,0[R1] ; R2 = *R1
\end{minted}

\begin{note}
	We have to create the pointer to the label first, afterwards we just use the pointer and never have to reference the label again.
	This lets us reuse a procedure for different labels.
\end{note}

\noindent
We can move to the next index by incrementing \mintinline{python}{R1}.

\begin{minted}[linenos,numbersep=5pt,frame=lines,framesep=2mm]{asm}
lea R1,1[R1] ; now R1 points to x[1] and not x[0]
\end{minted}

\begin{note}
	To increment the pointer we use \mintinline{asm}{lea R1,1[R1]} instead of \mintinline{asm}{add R1,R2,R1} because that requires there to be a constant \(1\) in another register (which uses another instruction too).
\end{note}

\subsection{Sum of an array using pointers}\label{sub:sum_of_an_array_using_pointers}

\subsubsection{High Level}\label{ssub:high_level}

\begin{minted}[linenos,numbersep=5pt,frame=lines,framesep=2mm]{python}
sum = 0
p = &x # start of array
q = &xEnd # points to a label that is at the end of the array

while p < q:
    sump = sum + *p # add the data at the current element
    p = p + 1 # Increment current element
\end{minted}

\subsubsection{Low Level}\label{ssub:low_level}

\begin{minted}[linenos,numbersep=5pt,frame=lines,framesep=2mm]{asm}
; R1 = p = pointer to current element
; R2 = q = pointer t oend of array
; R3 = sum of elements

    lea R1,x[R0] ; p = &x
    lea R2,xEnd[R0] ; q = &xEnd
    add R3,R0,R0 ; sum = 0, could also do a lea I think
sumLoop
    cmp R1,R2
    jumpge sumLoopEnd ; p >= q  (ie. !<) go to end
    load R4,0[R1] ; R4 = *p, this is current value
    add R3,R3,R4 ; R3 += R4
    lea R1,1[R1] ; p += 1, get pointer to next element
    jump sumLoop[R0] ; go back to start
sumLoopEnd
    trap R0,R0,R0

x   data 23
    data 42
    data 19
xEnd

\end{minted}

\subsection{Comparison of Approaches}\label{sub:comparison_of_approaches}

\subsubsection{Index}\label{ssub:index}

\begin{itemize}
	\item Get \mintinline{python}{x[i]} with \mintinline{asm}{load R5,x[R1]} where \mintinline{python}{R1=i}
	\item Move to next element using \mintinline{python}{i = i + 1}
	\item Determine end of loop with \mintinline{python}{i < xSize} so we need to have the total length.
	\item Know how big the array is
	\item Use a for loop
\end{itemize}

\subsubsection{Pointer}\label{ssub:pointer}

\begin{itemize}
	\item Initialise with \mintinline{asm}{lea R1,x[R0]} and initialise \mintinline{python}{q} to the end pointer.
	\item Get \mintinline{python}{x[i]} with \mintinline{asm}{load R5,0[R1]} where \mintinline{python}{R1 = p}/
	\item Move to next element by incrementing \mintinline{python}{p} like \mintinline{python}{p = p + 1}.
	\item Determine the end of the loop with \mintinline{python}{p < q} where \mintinline{python}{q} is at the end of the array.
	\item Don't need to know the number of iterations in advance.
	\item Use a while loop.
\end{itemize}

\section{Array of Records with Pointers}\label{sec:array_of_records_with_pointers}

Using pointers is more clean and simple than using the index based method.
We just loop with a while loop, but instead of incrementing by \(1\), we increment by the length of the record.
