\section{Protocol Layering}\label{sec:protocol_layering}

Communication systems are a series of layers to manage complexity, modularity, separation of concerns.
Each layer offers services to the higher layers using services from lower  layers.
The highest layer is the applications that we use, the lowest layer is the physical connections.

Peers at some layer \(i\) communication via the layer \(i\) protocol using lower layer services.

\subsection{OSI Reference Model}\label{sub:osi_reference_model}

The OSI Reference Model is a way of thinking about layered protocol design.
It is a design tool, since realistic implementations will be far more complex and more closely related to the \emph{TCP/IP (or the Internet)} model.

\begin{note}
	All nodes implement the bottom three layers (network, data link, physical) because they all need this information.
\end{note}
\subsection{Physical Layer}\label{sub:physical_layer}

Handles the transmission of raw bits over a communication link.
Defines attributes of the physical medium of communication:
\begin{itemize}
	\item Size and shape of plug
	\item Type of wire
	\item Radio frequency
	\item Modulation scheme
	\item Transmission power
	\item Electrical voltage
\end{itemize}

\subsection{Data Link Layer}\label{sub:data_link_layer}

The structure and frame of the physical layer bit stream.
This layer splits the bits in to individual \emph{frames}.
\begin{itemize}
	\item Detect and correct errors in messages using parity and ECC or acknowledgements.
	\item Access Control
	      \begin{itemize}
		      \item Assign addresses to each host on the link (MAC or IP address)
		      \item Allow access to link and allow hosts to send messages
		      \item Rate limit to ensure all hosts get fair access and avoid overwhelming hosts.
	      \end{itemize}
\end{itemize}

\subsection{Network Layer}\label{sub:network_layer}

Connects several links to form a connection between a source and a destination.
\textbf{Handles routing among nodes within a packet switched network.}

\begin{note}
	The unit of data exchanged between nodes in this layer is called a \emph{packet}.
\end{note}
\begin{note}
	Because these networks can be worldwide, the layer must be standardised (even more so than at any of the other layers).
\end{note}

\subsection{Transport Layer}\label{sub:transport_layer}

This layer handles the end-to-end transfer of data from the source process to the destination process (a \emph{process-to-process channel}).
The unit of data here is called a \emph{message}.
A transport layer provides:
\begin{itemize}
	\item Reliability
	\item Ordering
	\item Framing
	\item Congestion control
\end{itemize}
But the availability of each of these depends on the guarantees made in the network layer.

\subsection{Session and Presentation Layers}\label{sub:session_and_presentation_layers}

\subsubsection{Session Layer}\label{ssub:session_layer}

The session layer provides a place (namespace) to bunch several different transport streams together in one application.
eg., audio and video streams happening at the same time.

\subsubsection{Presentation Layer}\label{ssub:presentation_layer}

Manages the presentation, representation and conversion of data (character set, markup languages, data format conversion).

\subsection{Application Layer}\label{sub:application_layer}

Here we are talking about user application protocols (HTTP) and not the application themselves (Firefox):
\begin{itemize}
	\item HTTP
	\item FTP
	\item SSH
	\item \dots
\end{itemize}

\subsection{Protocol Standards}\label{sub:protocol_standards}

A standard is a formal definition of a network protocol to ensure interoperability.
Standards can be openly defined or closed source, restricted or free, collaborative or combative, etc.

Not all players in the standard process have the same goals:
\begin{itemize}
	\item Delaying a standard to allow a different solution to get market share
	\item Including intellectual property, patented technologies to get royalties, etc.
	\item Enhancing (or subverting) the security of a protocol.
\end{itemize}
Standards are required, but they may not always be the best solution.

\section{Physical and Data Link Layers}\label{sec:physical_and_data_link_layers}

These are the layers that interact with the physical hardware.
The physical layer defined what will happen in the hardware.
The data link layer still has to worry about the physical connections so that it can handle the data connection properly.

\begin{note}
	These layers usually differ from device to device.
\end{note}

\subsection{Physical Layer (or \(L1\))}\label{sub:physical_layer_or_mkl1_}

The physical layer is the only layer that when it speaks to its peer is actually physically connected to it.

The main focuses are to transmit a sequence of bits over an analogue channel subject to noise/hardware limitations/clock skew (time sensitive events are late), and interface to \(L2\).

\subsubsection{Examples}\label{ssub:examples}

\paragraph{Unshielded Twisted Pair}\label{par:unshielded_twisted_pair}

An electrical cable with two wires twisted together.
Each pair is one direction of signal and ground.
\begin{itemize}
	\item Twists reduce interference and noise pickup
	\item Can be hundreds of metres long and still work well
	\item Longer cables are more noisy
	\item Used in Ethernet and telephones.
\end{itemize}

\paragraph{Optical Fibre}\label{par:optical_fibre}

Glass core with shielding.
Fragile and can only transmit in one direction.
Low noise, high capacity, low cost, but requires expensive lasers to operate.

\subsubsection{Encoding Data For Wired Data Transmission}\label{ssub:encoding_data_for_wired_data_transmission}

Signal will be transported as binary.
The signal is usually put straight onto the channel without any conversion at all (unlike for a wireless data connection).
Encoding is done by changing the brightness or voltage.

\paragraph{Baseband Data Encoding}\label{par:baseband_data_encoding}

The signal is in its original frequency range (it has not been modulated to a higher frequency).
We encode the signal as a change in voltage or brightness.

If there are too many \(1\)s or \(0\)s in a row, it can be hard to tell exactly how many signals arrived.
To solve this, we can encode a \(1\) as a change in signal and a \(0\) as no change (NRZ inverted) instead of having the data encoded directly (NRZ).
This still has a problem for long lengths of \(0\)s though since the signal doesn't change.

\emph{Manchester encoding} solves this completely where a low-high transmission is encoded as \(0\) and a high-low transmission is encoded as \(1\).
Manchester increases the bandwidth requirement because the frequency is now double.

\subsubsection{Carrier Modulation}\label{ssub:carrier_modulation}

The baseband is the range of frequencies I want to transmit (the original data before modulation).
When we modulate, we convert the baseband data to become a higher frequency based on the data itself.

This allow you to have several signals on the same channel by having data on the same baseband frequency modulated to a free frequency for each signal.
Generally this is used for wireless links, but can be used for wired links too (ADSL and voice telephones use same wires).

\section{Bandwidth}\label{sec:bandwidth}

The bandwidth of a channel defines the frequency range it can transport (based on the physical limitations of the hardware itself).
The maximum transmission rate of a digital signal depends on the channel bandwidth:

\smallskip
\begin{minipage}{0.4\linewidth}
	\centering
	\scalebox{2}{
		\begin{math}
			R_{max} = 2H\log_2 V
		\end{math}
	}
\end{minipage}
\hfill
\begin{minipage}{0.55\linewidth}
	Where:
	\begin{itemize}
		\item \( R_{max} \) = maximum transmission rate of channel (bits per second)
		\item \(H\) = Bandwidth \(Hz\)
		\item \(V\) = number of possible discrete values per symbol (the size of the alphabet used).
	\end{itemize}
\end{minipage}
\begin{note}
	We assume a noise free channel here.
\end{note}

\subsubsection{What is \(V\)}\label{ssub:what_is_mk}

\(V\) is the possible discrete values per symbol (size of the alphabit).
Inside a computer system, we use binary (or \(V=2\)).
In a physical communication channel, we can have a larger alphabet so that each symbol has more data.

\subsection{Channel Capacity and Noise}\label{sub:channel_capacity_and_noise}

Real world channels have noise that alters the signal.
We can measure a channel's signal power, \(S\), and noise floor, \(N\), and compute the signal to noise ration (\(SN\) or \(SNR\)).

Bandwidth and \(SNR\) are fundamental limits on the throughput (bit rate) which might be able to be reached using careful engineering, but can never be surpassed (Shannon's Theorem).





