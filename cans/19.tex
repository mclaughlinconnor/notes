\section{Linked Lists}\label{sec:linked_lists}

A linked list is a dynamic (can change size) data structure consisting of a sequence of nodes arranged in order.
The order in a linked list is determined by a pointer in each record object

A node is a record with a total of two fields:
\begin{description}
    \item[Value] is a word containing useful information (could be anything at all).
    \item[Next] is a word containing a pointer to the next node in the list.
\end{description}
%
The last node in the list has a special \emph{nil} value in the \mintinline{python}{next} field.
Nil is represented by \(0\) since you cannot have a pointer to the memory address there.

\subsection{Basic Operations on Lists in Sigma16}\label{sub:basic_operations_on_lists_in_sigma16}

There are three key basic operations and they are implemented in Sigma16:
\begin{itemize}
    \item Is a list empty?
    \item What's the value of the node?
    \item What's the next node?
\end{itemize}

\begin{note}
    In the coming examples, we will assume that all pointer variables are in memory, but in practice these will often be kept in registers so there are less \mintinline{asm}{load}s and \mintinline{asm}{store}s.
\end{note}

\subsubsection{Is a list empty?}\label{ssub:is_a_list_empty_}

Nil is represented by \(0\), so the list \(p\) is empty iff (if and only if) \(p=0\).
It is unsafe to perform an action on a list  \(p\) unless \(p\) actually points to a node, this this test is commonly needed.

\begin{highlight}{Checking if a linked list is empty with Sigma16}
    \begin{code}{asm}
        load R1,p[R0] ; R1 = p
        cmp R1,R0 ; compare p with 0
        jumpeq pIsEmptyProcedure[R0] ; jump to pIsEmptyProcedure if p == 0
        ; No, p is not empty.

        pIsEmptyProcedure
        ; Yes, p is empty
    \end{code}
\end{highlight}

\subsubsection{Get the value in a node}\label{ssub:get_the_value_in_a_node}

Suppose \(p\) is a pointer to a node.
We want to do \mintinline{c}{x = *p.value} which is only safe to do if the list is not empty.

\begin{highlight}{Getting the value of a linked list node in Sigma16}
    \begin{code}{asm}
        load R1,p[R0] ; R1 = p
        load R2,0[R1] ; R2 = *p.value
        store R2,x[R0] ; x = *p.value
    \end{code}
\end{highlight}

\subsubsection{Get the pointer to the next node in a list}\label{ssub:get_the_pointer_to_the_next_node_in_a_list}

We want to do \mintinline{c}{p = *p.next}.
Again, this is only possible to do if \(p\) is not empty.
The \mintinline{asm}{next} field of a node is at offset \(1\) in the node record.

\begin{highlight}{Getting the pointer to the next node in a linked list}
    \begin{code}{asm}
        load R1,p[R0] ; R1 = p
        load R2,1[R1] ; R2 = *p.next
        store R2,q[R0] ; q = *p.next
    \end{code}
\end{highlight}

\section{Advanced Linked List Operations}\label{sec:advanced_linked_list_operations}

\subsection{Traversing a List}\label{sub:traversing_a_list}

A while loop is the best looping construct for traversing a list (a for loop is better for arrays).

\begin{highlight}{Traversing a linked list}
    \begin{code}{python}
        def list_sum(p):
        sum = 0
        while p != nil: # Have I reached the end of the list?
        x = (*p).value # read node value
        sum = sum + x # accumulate into sum
        p = (*p).next # update pointer to next node
    \end{code}
\end{highlight}

\subsection{Searching a list for a value}\label{sub:searching_a_list_for_a_value}

Again, we should use a while loop.

\begin{highlight}{Searching a linked list for a value}
    \begin{code}{python}
        def list_search(p, x):
        found = False
        while p != nil and not found: # loop until end or until found
        found = (x == (*p).value) # found = True if x matches the value
        p = (*p).next # move on to next node

        return found
    \end{code}
\end{highlight}

\subsection{Getting a New Node From The Heap}\label{sub:getting_a_new_node_from_the_heap}

Linked lists go onto the heap.
We have to assume:
\begin{itemize}
    \item We have a set list of available nodes on the heap
    \item It is organised as a linked list themselves
    \item With a pointer \mintinline{asm}{avail} pointing to the first node.
\end{itemize}

\begin{highlight}{Getting a new linked list node from the heap}
    \begin{code}{python}
        if avail == nil:
        raise Exception('Fatal error: out of heap')
        else:
        newnode = avail
        avail = (*avail).next
        return newnode
    \end{code}
\end{highlight}

\begin{note}
    Nodes do not move around the memory, it always stays in the same place, we just have pointers pointing to it.
\end{note}

\subsection{Inserting a node after another node}\label{sub:inserting_a_node_after_another_node}

To insert \(x\) after \(q\):
\begin{highlight}{Inserting a node into a linked list after another node}
    \begin{code}{python}
        r = newnode() # from above
        (*r).value = x # set value of current node to x
        (*r).next = (*q).next # r and q both have the same next
        (*q).next = r # q.next now points to r
    \end{code}
\end{highlight}

\subsection{Insert a node at the head}\label{sub:insert_a_node_at_the_head}

We can either create a different procedure for inserting at the head or we can make a blank header node where the value is nil and the next points to the first item.

We can then just use the same procedure to insert a node after the head.

\subsection{Deleting a node}\label{sub:deleting_a_node}

We can delete a node after \(q\).
Just change the previous node to skip over the one to be deleted.
Return the node being deleted back to \mintinline{asm}{avail}.

\begin{highlight}{Deleting a particular node from a linked list}
    \begin{code}{python}
        def delete(q):
        if q != nil: ; is q nil just in case
        r = (*p).next ; get the actual node to be deleted
        if r != nil: ; is r the last node?
        (*q).next = (*r).next ; skip over r entirely
        ; r is still using memory

        ; insert r between avail and avail[i+1]
        (*r).next = avail
        avail = r
    \end{code}
\end{highlight}

\subsubsection{Space Leaks}\label{ssub:space_leaks}

If you return a deleted node to the list of available addresses, it can be reused.
If you don't do this, the node becomes inaccessible (doesn't hold useful data, but cannot be reallocated).
This is a bug in the program and over time, many nodes may become inaccessible: a space leak.

\subsubsection{Memory Management}\label{ssub:memory_management}

It is a bug if you delete a node that contains useful data.
It is also a bug if you don't delete a node that doesn't have useful data.
With complicated data structures, this can be difficult.

A common solution is \emph{garbage collection}.
\begin{itemize}
    \item The program doesn't explicitly return nodes to the list of available locations.
    \item Periodically, the garbage collector traverses all data structures and marks the nodes it finds.
    \item Then it adds all unmarked nodes to the list of available addresses.
\end{itemize}

\section{Arrays and Linked List Comparison}\label{sec:arrays_and_linked_list_comparison}

Lists and arrays are two kinds of data structure that contain a sequence of data.

\subsection{Accessing Elements}\label{sub:accessing_elements}

\subsubsection{Direct access to an element}\label{ssub:direct_access_to_an_element}

\begin{description}
    \item[Array] Gives direct access (random access) to any element.
    \item[List] give only direct access to an element you have a pointer to.
\end{description}

\subsubsection{Traversal}\label{ssub:traversal}

\begin{description}
    \item[Array] Initialise an index to \(0\), then increment for each iteration (a for loop).
    \item[List] Repeatedly go to the next pointer.
\end{description}

\subsection{Memory Usage}\label{sub:memory_usage}

\subsubsection{Memory Per Element}\label{ssub:memory_per_element}

\begin{description}
    \item[Array] Need just the memory for each element
    \item[List] Need a node for each element with two values (requires \(2 \times n\) words).
\end{description}

\subsubsection{Flexibility}\label{ssub:flexibility}

\begin{description}
    \item[Array] An array has a fixed size and needs to be allocated fully.
    \item[List] A list has a variable size and only needs the space of the data.
\end{description}

\subsection{More Types of Linked List}\label{sub:more_types_of_linked_list}

\begin{itemize}
    \item Doubly linked list where nodes are linked both ways.
    \item Circular lists where the start is linked to the end.
    \item Trees where each node has more than one next.
\end{itemize}

