\section{Permutations and Combinations}\label{sec:permutations_and_combinations}

\subsection{Permutations}\label{sub:permutations}

The motivating problem for permutations was the travelling salesman problem.
Ie, how many different ways are there to visit all of: Glasgow, Livingston, Perth, Edinburgh, and Stirling, then return back to their base.

\paragraph{Definition}\label{par:definition}

A permutation of a set of distinct objects is an ordered arrangement of these objects.
Eg, \(\{1,2,3\}\) and \(\{3,2,1\}\) are different permutations of the set \(S=\{1,2,3\}\).
There's a general formula to calculate how many permutations there are:
\[
    n = (|S|)!
\]
If we are using the form \(P(n)\), this will be:
\[
    P(n) = n!
\]
or \(n\times (n-1) \times (n-2) \times ...\), which works since all values are possible for the first position, then after we fill that position, there will only be two choices left, and so forth.

\subsection{r-Permutation}\label{sub:r_permutation}

Again, we have a motivating problem: how many unique ways are there to visit \(r=5\) of all of the \(8\) cities?
We will keep the same \(S=\{1,2,3\}\)
\begin{itemize}
    \item \(\{1,3,2\}\) is a permutation
    \item \(\{3,2\}\) is a \(2\)-permutation
\end{itemize}
The number of \(r\)-permutations of a set with \(n\) elements is denoted by \(P(n,r)\).
Eg, the number of \(2\)-permutations of \(S\) is \(P(3,2)=6\).
This \(6\) can be calculated with the product rule:
\begin{itemize}
    \item Instead of having the three ``places'' we had before, we only have \(2\).
    \item For the first position, there are three choices
    \item For the second position, there are now only choices
    \item There is no third position, so we don't multiply by \(1\) (it's a bad example)
    \item So we just multiply \(3\times 2\) or \(n\times (n-1)\).
\end{itemize}
This means that we can calculate the \(r\)-permutation by:
\[
    P(n,r) = \frac{n!}{(n-r)!}
\]

\subsection{Combinations}\label{sub:combinations}

If we don't case about the ordering of a particular set of permutations (eg, for people on a team, we don't care the order they're on the team).
So we need to divide the number of \(r\)-permutations by the number of ways we can rearrange \(r\) elements (we count the number of values of \(P(r,r)\)).
Combinations can be denoted in one of two ways:
\[
    C(n, r) = {n\choose r}
\]
and can be calculated like:
\[
    C(3,2) = \frac{n!}{(n-r)!r!}
\]
