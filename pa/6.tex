\section{Static vs Dynamic Data Structures}\label{sec:static_vs_dynamic_data_structures}

\subsection{What is a data structure?}\label{sub:what_is_a_data_structure_}

A collection of data where there is some relationship between the data pieces that we can perform operations on.

\subsection{Static Data Structures}\label{sub:static_data_structures}

Static data structures need to have a maximum size defined when they are first defined.

\subsection{Dynamic Data Structures}\label{sub:dynamic_data_structures}

Dynamic data structures can vary in size at runtime so you can grow the structure as needed.

\subsection{Comparison of Dynamic and Static Data Structures}\label{sub:comparison_of_dynamic_and_static_data_structures}

\begin{tabular}{p{7cm}p{7cm}}
    \toprule
    Dynamic                                                          & Static                                                            \\
    \midrule
    Only use as much memory as needed                                & Memory is fixed, so there is no risk of running out               \\
    Easier to add/remove data                                        & No need to check data size at runtime (you already know the size) \\
    We don't need to know the size in advance                        & Random access is easier                                           \\
    Run the risk of ``overflowing'' if you run out of memory         & Can waste some memory if the whole structure is not used          \\
    Harder to program (you need to keep track of the data structure) & Adding/removing items is slow (part of array needs rewritten)     \\
    Data can be left ``orphaned'' filling up memory with garbage     & Array size must be estimated before use                           \\
    \bottomrule
\end{tabular}

\section{Singly Linked Lists}\label{sec:singly_linked_lists}

This is a sequence of nodes arranged in linear order.
Every node has a \mintinline{text}{key} and a pointer to the next node \mintinline{text}{next}.
For a node \mintinline{text}{x}, the next element will be \mintinline{text}{x.next}.
Where \mintinline{text}{x.next = NIL}, then there is no next element.

\subsection{Head Pointer}\label{sub:head_pointer}

We need a head pointer to point to the first element of the list, so we call that \mintinline{text}{x.head}.
If \mintinline{text}{x.head = NIL}, the list is empty.

\subsection{Insertion}\label{sub:insertiontolinkedlist}

\begin{enumerate}
    \item Start while loop with the condition set to check if the \mintinline{text}{current.next = NILL} to insert at the tail, or \mintinline{text}{current.key = target} to insert after a particular node.
    \item Update \mintinline{text}{current} to \mintinline{text}{current.next} .
    \item Allocate a new node
    \item Update \mintinline{text}{new.next} to point to \mintinline{text}{current.next}
    \item Update \mintinline{text}{current.next} to point to the new node
\end{enumerate}
\begin{note}
    The order here is important since we will ``lose'' the next element if we don't do them in the correct order.
\end{note}
\begin{note}
    Insertion at the tail is useful to implement a first-in-first-out data structure like a queue.
\end{note}
\begin{note}
    If we are already at the head of the list, then we can give a special case to insert at the head with no iteration.
\end{note}

\subsubsection{Time Complexity}\label{ssub:linkedlistinsertcomplexity}

The time complexity of inserting at the head is only \(O(1)\), since we already know where the first element is.
The time complexity of inserting at any other point is \(O(n)\), since we have to loop through the while list.

\subsection{Deletion}\label{sub:linkedlistdeletion}

\begin{enumerate}
    \item Update \mintinline{text}{prev.next} to \mintinline{text}{deleted.next}
    \item Deallocate \mintinline{text}{deleted}.
\end{enumerate}
\begin{note}
    We always have to check that \mintinline{text}{prev.next} is not \mintinline{text}{NIL} so that we know the deleted element exists.
\end{note}

\subsubsection{Time Complexity}\label{ssub:linkedlistdeletecomplexity}

The time complexity of deleting at the head is only \(O(1)\), since we already know where the first element is.

\subsection{Search}\label{sub:linkedlistsearch}

\begin{enumerate}
    \item Starting from \mintinline{text}{current}.
    \item Start while loop with the condition set to check if the \mintinline{text}{current.key = target} -- for example, something like \mintinline{text}{while current.key != target and current != NILL}.
    \item Update \mintinline{text}{current} to \mintinline{text}{current.next}.
    \item Return \mintinline{text}{current} which will either be a pointer to the correct element or \mintinline{text}{NIL} if the search result doesn't exist.
\end{enumerate}

\subsubsection{Time Complexity}\label{ssub:time_complexitysearchlinkedlist}

Because we might need to search through every element of the linked list, the complexity is \(O(n)\).

\subsection{Reducing the time complexity of tail operations}\label{sub:reducing_the_time_complexity_of_tail_operations}

We can add another pointer to the tail end of a linked list, which will allow us to insert or delete directly at the very end of the list without having to traverse all the way through it first.
This does have the downside of requiring inserting and deleting functions to update this new tail pointer.
\begin{note}
    The space complexity is slightly worse here because we're storing another variable, but we do manage to reduce the time complexity of a tail insertion or deletion to \(O(n)\)
\end{note}
\begin{note}
    If the list is empty, we can just perform a regular head insertion for a \(O(1)\) time complexity still.
\end{note}
When we want to delete a node, we need to be able to update the previous node \mintinline{text}{prev}, which can only be done if we traverse the list in order, which we don't do if we jump straight to the end.
This means that deletion of an element of a linked list is always \(O(n)\) unfortunately, unless we can traverse the list in the other direction.

\section{Doubly Linked Lists}\label{sec:doubly_linked_lists}

A doubly linked list is very similar to a regular linked list, except it has a \mintinline{text}{node.prev} pointer in addition to the existing \mintinline{text}{node.next} pointer, that points to the (you guessed it!) previous node.
This has the downside that there is a larger memory overhead to the list, but some operations are much more efficient and simple.
\begin{note}
    As with \mintinline{text}{node.next}, if \mintinline{text}{node.prev} is \mintinline{text}{NIL}, then we are at the head of the linked list.
\end{note}

\subsection{Deletion}\label{sub:deletionofelementsonadoublylinkedlist}

To perform a deletion, we need to update the next and previous nodes in the list.
We don't need to worry about

\begin{enumerate}
    \item Update \mintinline{text}{prev.next} to \mintinline{text}{delete.next}, if the previous node is \mintinline{text}{NIL}, we need to update the head instead.
    \item Update \mintinline{text}{next.prev} to \mintinline{text}{delete.prev}, if the next node is \mintinline{text}{NIL}, we don't need to do anything since \mintinline{text}{NIL} has no \mintinline{text}{prev} value.
\end{enumerate}

