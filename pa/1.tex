\section{Course Introduction}\label{sec:course_introduction}

The course will be worth 20 credits, so will be slightly easier than last semester's CANS.

We will spend the course talking about what algorithms are.
Simply an algorithm is a  step by step method for completing a task.
If we are able to specify an algorithm for a problem, we are able to automate its solution.

In general, the field of computing science is the study of mathematical, linguistic, and hardware representations of algorithms.
\begin{description}
    \item[Mathematical] We need to make sure algorithms are efficient and correct.
    \item[Hardware] We need to make machines to complete the algorithms.
    \item[Linguistic] We need to make programming languages to express algorithms in.
\end{description}

\subsection{Learning Outcomes}\label{sub:learning_outcomes}

\begin{enumerate}
    \item We will implement basic data structures
    \item We will learn to use recursion
    \item We will analyse and implement algorithms
    \item We will use mathematics to write and prove assertions about our algorithms
    \item Understand inductively generated structures and proofs
    \item Use basic combinatorics to solve mathematical problems
\end{enumerate}

In general we will be focusing on problem-based learning to complete these objectives.

\subsection{Class Times}\label{sub:class_times}

Class time will be split \(60\%\) to \(40\%\) where \(40\%\) of the time will be spent with ``online any time'' lectures, and the remaining ``60\%'' will be live workshops that will require a pen, paper, and at least one laptop.

\subsection{Assessments}\label{sub:pa_assessments}

\begin{description}
    \item[Workshop Participation] Worth \(10\%\) which is either awarded or isn't awarded.
    \item[Class Tests] There will be a total of 5, where each one is worth \(1\%\) of the total course.
    \item[First Coursework Task] This will be a programming task worth \(15\%\) of the entire course and will be due on \(2021-12-16\).
    \item[Second Coursework Task] This will be a written exercise worth \(10\%\) that is dues on \(2022-01-09\).
    \item[Final Exam] There will be a problem-based final exam on \(2022-02-09\) that will be worth the remaining \(60\%\) of the course.
\end{description}

\subsection{Applications of Algorithms}\label{sub:applications_of_algorithms}

\begin{enumerate}
    \item Game Design
    \item Film special effects
    \item Most areas of mathematics and physics
    \item Computer simulations
    \item Finding organ transplants
    \item Artificial Intelligence
    \item Natural language processing
\end{enumerate}

\section{Writing Programs to Solve Problems}\label{sec:writing_programs_to_solve_problems}

There are three phases to any algorithmic computer program:
\begin{enumerate}
    \item Input
    \item Process
    \item Output
\end{enumerate}
Data processing is the step we are most interested in.
Here we need to think computationally by avoiding intuitions, guesses and feelings that aren't available to a computer.

\subsection{Thinking Like a Computer}\label{sub:thinking_like_a_computer}

\begin{description}
    \item[Algorithmic Thinking] We need to create a logical flow of thinking for the computer to follow.
    \item[Analysis] We need to create accurate, fast solutions to our problems.
    \item[Decomposition] Problems need to be broken down.
\end{description}
Along with these more algorithmic styles of thinking, we also need to be able to \emph{abstract} and \emph{generalise} problems.

In an algorithmic program, we have access to three types of operations: \emph{sequential}, \emph{conditional}, iterative -- which are all quite self explanatory.

\subsubsection{Sequential}\label{ssub:sequential}

A sequential operation is one which performs a process.
The program flow goes straight in one side and right out the other.

\subsubsection{Conditional Operations}\label{ssub:conditional_operations}

A conditional operation asks a question, then performs an operation based on he results.
