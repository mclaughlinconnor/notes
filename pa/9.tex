\section{Sequences}\label{sec:sequencespanine}

A sequence is similar to a set, except there is an order, we care about the relationship from one element to the next, and duplicates are allowed.
When we are looking at sequences, we will say that a relationship \emph{recurs} across elements (a recurring relationship between elements).
Once we get the starting value, we can use the recurring relation to recreate the entire sequence.

The idea of recurrence relations will link closely to the idea of recursive functions later on (and how to calculate their big \(O\) time).

One of the most important uses of recurrence relations is to provide solutions to certain ``counting problems'' like counting the steps in an algorithm, for example.

\subsection{Arithmetic Progression}\label{sub:arithmetic_progression}

An arithmetic progression is where a sequence has a common difference between terms, ie:
\[
    x = (\text{difference} \times n) + \text{offset}
\]
where \(x\) is the sequence element, \(\text{difference}\) is the space between the elements, and \(\text{offset}\) is the offset from \(0\) that the first element is.
All terms are real numbers.
This is usually written as \(x=a+nd\).

\begin{note}
    A good way to remember this, is that it creates a straight line graph, so is basically \(y=mx+c\).
\end{note}

\subsection{Geometric Progression}\label{sub:geometric_progression}

Here, there is a common ration between the terms, and not a common difference, ie:
\[
    x = \text{initial} \times \text{ratio}^{n}
\]
This can also be written as \(x=ar^{n}\).

\section{Recurrence Relations}\label{sec:recurrence_relations}

Sequences don't always have to have a relationship, but we don't care about these.
We can define a sequence by defining each item in terms of previous items (a recurrence relation), or we can use a formula to give the term at any position we want.

For a recurrence relation, we need to specify the relationship, as well as the first (or first few elements) of the sequence, for example we would say:
\begin{align*}
    a_0 = 0 \quad a_n = a_{n-1} + 2
\end{align*}
\begin{note}
    This means that \(\left\{a_n\right\}=0, 2, 4, 6, \ldots\). We need to specify that it's \(a_n\) and not just \(a\) since it represents the whole sequence and not just one value.
\end{note}

\section{Solving Recurrence Relations}\label{sec:solving_recurrence_relations}

If we take the recurrence relation:
\begin{align*}
    a_0 = 0 \quad a_n = a_{n_1}+2
\end{align*}
Then to find the \(50^{\text{th}}\) step, we would have to do \(50\) calculations (\(O(n)\)), but there is a better way.

Finding a formula for the \(n^{\text{th}}\) term of a recurrence relation is called solving the recurrence relation.
Such a formula is called a \emph{closed formula}.
\begin{enumerate}
    \item Start with the initial condition
    \item Work upwards until you reach \(a_n\) in terms of \(a_0\) and constants only
    \item Deduce the formula
\end{enumerate}

\section{Summations}\label{sec:summations}

A summation is represented by the \(\Sigma\) symbol:
\[
    \sum^{n}_{i=m} a_{i} = a_m + a_{m+1} + a_{m+2} + \dots + a_{n-1} + a_{n}
\]
A product is represented by the \(\Pi\) symbol:
\[
    \prod^{n}_{i=m} a_{i} = a_m \times a_{m+1} \times a_{m+2} \times \dots \times a_{n-1} \times a_{n}
\]
where \(n\) is the upper limit, \(m\) is the lower limit and \(i\) is the index.

\begin{note}
    We can also specify conditions for our sums:
    \[
        \sum_{j\in \{2, 5, 7, 10\} } a_j = a_2 + a_5 + a_7 + a_10
    \]
\end{note}

\section{Proof By Induction}\label{sec:proof_by_induction}

To prove a condition is true for any term in a sequence, we need to show:
\begin{enumerate}
    \item We can reach the first term
    \item We can reach any term
\end{enumerate}
To prove \(P(n)\), where \(n\in \Z^{+}\), show that \(P(1)\) is true (named the \emph{basis step}), then show that if \(P(k)\) is true, then \(P(k+1)\) is also true (named the \emph{inductive step}).
\begin{note}
    Here, \(P\) is a preposition or a statement that is either true or false.
\end{note}
The implication symbol looks like \(\to\), and the for all operator looks like \(\forall\), so we get \(P(k) \to P(k+1) \forall k\)
To complete the inductive step, we:
\begin{itemize}
    \item \textbf{Assuming} the inductive hypothesis that \(P(k)\) is true for any integer \(k\)
    \item \textbf{Show} that \(P(k+1)\) must be true by equating \(P(k) + (k+1)\) and \(P(k+1)\)
\end{itemize}

\subsection{Template}\label{sub:template}

\begin{itemize}
    \item State clearly what you want to prove, ie. \(P(n)\).
    \item Basic step (show that \(P(1)\) is true, or, in general \(P(b)\) is true)
    \item Inductive step
          \begin{itemize}
              \item State clearly \(P(k)\) -- ie. What we are assuming, also called the \emph{indicative hypothesis} -- and all values of \(k\)
              \item State clearly \(P(k+1)\) -- ie. What we want to prove
              \item Show that \(P(k+1)\) is true using the inductive hypothesis (\(P(k)\)). This usually means showing that \(\text{lhs} = \text{rhs}\), always start by manipulating \(\text{lhs}\), but keep an eye on \(\text{rhs}\) since this is where we want to end up.
          \end{itemize}
\end{itemize}
