\section{Mathematical Sets}\label{sec:mathematical_sets}

Sets are a (the?) fundamental component to all mathematics.
Even the concept of functions is built upon the idea of sets, since a function just assigns the numbers from one set on to another set.

\subsection{Definition}\label{sub:definition}

A \emph{set} is a collection of objects (named \emph{members}) where the most important aspect of the set is the membership, therefore we can consider a set to be unordered.
\begin{itemize}
    \item We only care about membership, so the order or being able to sort it is irrelevant and useless.
    \item We can have duplicates, but they are useless and irrelevant too since we still only care about whether one member is or isn't in the set.
\end{itemize}
\begin{note}
    A set ``contains'' its elements.
\end{note}

\subsubsection{Symbols}\label{ssub:symbols}

To say that \(a\) is a member of \(A\), we say: \(a \in A\). Where we want to say that \(a\) is not a member of \(A\), we say: \(a \notin A\).

\subsection{Notations}\label{sub:notations_pafour}

There are several different ways of representing a set which are all useful on different occasions.

\subsubsection{Roster Method}\label{ssub:roster_method}

We include all of the elements between braces.
\[
    S = \{A, B, C, D\}
\]
We can also use ellipses (\(\dots\)) to show a continuation, but we must make sure that the patterns is unambiguous.
\[
    S = \{1, 2, 3, \dots, 99, 100\}
\]
\begin{note}
    The order and duplicicity here is still irrelevant.
\end{note}
Even though this is the clearest method, there are some limitations:
\begin{itemize}
    \item It is impossible or inconvenient to describe sets which do not follow a very set and clear pattern (like the rational numbers for example).
\end{itemize}

\subsubsection{Set Builder Notation}\label{ssub:set_builder_notation}

Here write an expression that can be expanded to a set.
\begin{align*}
    S & = \{x \mid x \text{ is a positive integer } < 100\}     \\
    O & = \{x \mid x \text{ is a positive odd integer } < 100\} \\
    P & = \{x \in \Z \mid x \text{ is odd and } < 100\}
\end{align*}
Sometimes it is not very easy to write an expression to describe a set, so we can use a predicate function instead.
\[
    S=\{x \mid \mathrm{Prime}(x)\}
\]

\subsection{Common Sets}\label{sub:common_sets}

As well as the sets we define ourselves, there are several sets which are always available and always defined the same.

\begin{description}
    \item[Natural/Counting] \(\N = \{0, 1, 2, 3, 4, \dots\}\)
    \item[Integers] \(\Z = \{\dots, -3, -2, -1, 0, 1, 2, 3, \dots\}\)
    \item[Positive Integers] \(\Z^{+} = \{1, 2, 3, \dots\}\)
    \item[Rational] \(\Q=\{\frac{p}{q} \mid p, q \in \Z, \text{and} \neq 0\}\)
    \item[Real] There is no specific definition, these are just all numbers.
\end{description}

\subsubsection{Universal Set}\label{ssub:universal_set}

The universal set contains all possible elements that are currently under considerations.

\subsubsection{Empty Set}\label{ssub:empty_set}

The empty set is pretty self explanatory -- it simply contains no elements.

\section{Set Equality and Subsets}\label{sec:set_equality_and_subsets}

For this section, we have to first define a few symbols:

\begin{tabular}{ll}
    \toprule
    Symbol           & Description                                                                                                              \\
    \midrule
    \(\forall \)     & For all members                                                                                                          \\
    \(A \implies B\) & If \(A\) is \mintinline{text}{true}, then \(B\) is too.                                                                  \\
    \(A \iff B\)     & If and only if \(A\) is \mintinline{text}{true}, \(B\) is true. \(B\) cannot be \mintinline{text}{true} unless \(A\) is. \\
    \(A \lor B\)     & Logical OR. Used to OR two conditions together.                                                                          \\
    \(A \land B\)    & Logical AND. Used to AND two conditions together.                                                                        \\
    \bottomrule
\end{tabular}

\subsection{Set Equality}\label{sub:set_equality}

Two sets are said to be equal if and only if they have the same elements.
\[
    \forall x(x \in A \iff x \in B) \quad \mathrm{\textbf{OR}} \quad A = B
\]
For all values of \(x\), if \(x\) belongs to \(A\), it also belongs to \(B\).

\subsection{Subsets}\label{sub:subsets}

It can be said that \(A\) is a subset of \(B\) if and only if all of the values in \(A\) are also in \(B\).
\[
    A \subseteq B
\]
Which can be expanded to the more verbose:
\[
    \forall x (x \in A \implies x \in B)
\]
Equal sets are also considered subsets of each other:
\[
    A=B \iff A \subseteq B \land B \subseteq A
\]
\begin{note}
    Since \(\emptyset\) is a subset of all other sets, \(\forall S(\emptyset \subseteq S)\).
    It should also be obvious that every set is a subset with itself: \(\forall S ( S \subseteq S )\)
\end{note}

\subsubsection{Proper Set}\label{ssub:proper_set}

If \(A\) is a subset of \(B\), but \(A \neq B\), then \(A\) is a ``proper subset'' of \(B\).

\section{Set Operations}\label{sec:set_operations}

Set theory is very similar to the boolean algebra covered in CANS, so we will reuse most of the same operations here.

\subsection{Union}\label{sub:unionpafour}

The union of sets \(A\) and \(B\) can be written as:
\[
    A \cup B
\]
which is basically just \(A + B\), but can be written more verbosely like:
\[
    A \cup B = \{x \mid x \in A \lor x \in B\}
\]

\subsection{Intersection}\label{sub:intersection}

The intersection operator selects members which are in \(A\) and \(B\) and can be written as:
\[
    A \cap B = \{x \mid x \in A \land x \in B\}
\]
If \(A \cap B= \emptyset\), then it can be said that \(A\) and \(B\) are disjoint sets.

\subsection{Compliment}\label{sub:compliment}

This is simply everything that is not included in a specific set and is written as:
\[
    \overline{A}=\{x \in \mathbb{U} \mid x \notin A\}
\]
This can be simplified to a much nicer:
\[
    \mathbb{U}-A
\]

\subsection{Difference}\label{sub:differencepafour}

The difference between two sets is all of the elements in \(A\), but not in \(B\) and can be written as:
\[
    A-B = \{x \mid x \in A \land x \notin B\} = A \cap \overline{B}
\]
It's worth mentioning that \(A \cap \overline{B}\) works since \(\overline{B}\) is everything that is not in \(B\), so we can get the intersection of that with \(A\) to get members in \(A\), but not in \(B\).

\section{Set Identities}\label{sec:set_identities}

Similar to boolean algebra, we have several identities which are good to understand.

\subsubsection{Identity Laws}\label{ssub:identity_laws}

\[
    A \cup \emptyset = A \quad A \cap \mathbb{U} = A
\]

\subsubsection{Domination Laws}\label{ssub:domination_laws}

\[
    A \cup \mathbb{U} = \mathbb{U} \quad A \cap \emptyset = \emptyset
\]

\subsubsection{Idempotent Laws}\label{ssub:idempotent_laws}

\[
    A \cup A = A \quad A \cap A = A
\]

\subsubsection{Complementation Laws}\label{ssub:complementation_laws}

\[
    \overline{(\overline{A})} = A
\]

\subsubsection{Commutative Laws}\label{ssub:commutativepafour}

\[
    A \cup B = B \cup A \quad A \cap B = B\cap A
\]

\subsubsection{Associative Laws}\label{ssub:associateivepafour}

\[
    A \cap (B \cap C) = (A \cap B) \cap C \quad A \cap (B \cap C) = (A \cap B) \cap C
\]

\subsubsection{Distributive Laws}\label{ssub:distributive_laws}

\[
    A \cap (B \cup C) = (A \cap B) \cup (A \cap B)
\]
\begin{note}
    This is just like breaking brackets in mathematics. Also to note, we can't move the position of the brackets since there are different operations going on.
\end{note}

\subsubsection{De Morgan's Laws}\label{ssub:de_morgan_s_laws}

\[
    \overline{(A \cup B)} = \overline{A} \cap \overline{B}
\]

\subsubsection{Absorption Laws}\label{ssub:absorption_laws}

\[
    A \cup (A \cap B) = A \quad A \cap (A \cup B) = A
\]

\subsubsection{Complement Laws}\label{ssub:complement_laws}

\[
    A \cup \overline{A} = \mathbb{U} \quad A \cap \overline{A} = 0
\]

\subsubsection{Converting Complements to Difference}\label{ssub:converting_complements_to_difference}

\[
    A \cap \overline{B} = A - B
\]

