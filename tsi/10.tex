\section{Advanced Testing}\label{sec:advanced_testing}

\subsection{Graphical User Interface Testing and User Experience Testing}\label{sub:graphical_user_interface_testing_and_user_experience_testing}

Testing the user interface,keyboard input, mouse input and touch screen.
This process is done manually and is very time consuming (the same effort has to be repeated every time).
The quality of the test can be dependent on the tester, but a human tester can find edge cases that would be missed in automated testing.

To reduce the time taken, you could capture and replay actions performed to compare to the next time to improve reproducibility.

We can use \emph{model view controller} to separate the fronted automatically because it is completely separate from the other code (inputs and outputs).

Automated testing is a thing, but there is a lot of effort for initial set up and maintain.
You need to find a balance between the effort and the return (you don't always get edge cases too).

\subsection{Manual Testing}\label{sub:manual_testing}

We can have test packs which is just a big document of test cases.
There are tools that can be used to record the results (what has and should happen)

\subsection{Integration Testing}\label{sub:integration_testing}

Integration testing is testing everything works well together.
We used mocks and fakes before to eliminate the need of integration testing, but running an overnight test with a real database (possibly just a copy of a live database or created synthetic data) could raise some issues.

\emph{Chaos engineering} is only usable with load balancing.
What you do is turn off some servers and make sure things still work or fail like they are supposed to.
You should do these during quiet times (ie.\ not at peak times).
Does disaster recovery kick in and work correctly.

\subsection{User Acceptance Testing}\label{sub:user_acceptance_testing}

The end users test their requirements.
This is often the very last test before the release to production.
Each user acceptance test should be mapped back to the original requirements (so you can see which requirements have or have not been met. Is the requirement ambiguous?).

Using a stable environment (that has already passed a number of test stages) is important because some applications could have a knock-on effect if they fail.

The hardware should be the same as production, sometimes the clone is used for disaster recovery.

\subsection{Behaviour Driven Development}\label{sub:behaviour_driven_development}

Rather than capturing requirements, capture behaviours, so tests are created from the behaviours.

\emph{Cucumber} is a tool that can be used to automatically create tests from the behaviours after the developer has written ``glue'' between the behaviours and the code.

\begin{quote}
	As a \dots, I want \dots, so that \dots
\end{quote}

\subsection{Non functional testing}\label{sub:non_functional_testing}

\begin{itemize}
	\item \textbf{Penetration testing}: What happens when someone tries to bring down a site?
	\item \textbf{Capacity testing}: Can you handle all of the requests being made?
	\item \textbf{Usability testing}: How is the screen working? Do people have to test too many buttons?
	\item \textbf{Portability Testing}: does it work on many different operating systems?
\end{itemize}

\subsection{A/B testing}\label{sub:a_b_testing}

\begin{enumerate}
	\item Create a hypothesis or idea
	\item Create an experiment environment to test out changes
	\item Create a base default cell to compare against (control case)
	\item Create two variations of the default to assign to certain users.
	\item Compare the two variations with the control to see if there was any improvements and which one was best.
\end{enumerate}
