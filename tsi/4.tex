\section{Test Doubling}\label{sec:test_doubling}

\subsection{Why?}\label{sub:why_}

We double our tests so that we can avoid the IO operations that slow down our code (which gives us early feedback) and also limits the expensive connection to the database.

\subsection{How?}\label{sub:how_}

When doubling, it is easier to work with individual functions where we can just test the return values.
We should only ever test with raw data (ie.\ don't pre-manipulate anything), then test the manipulation separately.

It is easier to do this if the data parsing is done separately to any of the processing so do it separately.

\subsection{Single Responsibility}\label{sub:single_responsibility}

A class should do exactly one thing.
A function should do exactly one thing.
This makes code easier to read.

\subsection{Adapters}\label{sub:adapters}

\begin{enumerate}
	\item Create an interface with a common public function to get data.
	\item Create one class that implements the interface and function to get data from a file.
	\item Create another class that implements the function stub with function stubs.
\end{enumerate}
%
You should create a different class for each adapter and each adaptee must implement all of the methods on the interface (because it is an interface).

The client should have a variable for the adapter and should have a getter and a setter to allow the interface methods to be changed.

\subsection{Command Query Separation}\label{sub:command_query_separation}

A command performs an action.
A query only returns data.
Never do both at the same (don't query and change state in the same function).
An interface function will just return hard coded data.
