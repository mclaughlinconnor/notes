\section{Code Review and Retrospective}\label{sec:code_review_and_retrospective}

\subsection{Code Review}\label{sub:code_review}

\begin{itemize}
	\item Ownership
	\item Readability and maintainability of code
	\item Find bugs
	\item Consistent style (style guide)
	\item Adequate tests
	\item Find any security issues (bugs)
\end{itemize}

\subsubsection{Ownership}\label{ssub:ownership_one}

Lets a senior engineer understand a particular process.
Make sure they understand the indention and they also will pick up things.

Gives a person ownership and pride in what section they created.
Having clear ownership means there is one person to ask about one particular piece of the codebase.

In smaller teams there may be a shared ownership through everyone.

\subsubsection{Maintainability}\label{ssub:maintainability}

Can future software engineers easily write code for a class, etc.
Being readable means more than one person can understand the code which eliminated key man depenedancies and solves problem where people go on holiday or quit.

\subsubsection{Why review?}\label{ssub:why_review_}

\begin{itemize}
	\item \textbf{Educational}: teaching and learning from looking at and getting your code looked at.
	\item \textbf{Maintaining Standards}: several people looking at a change will catch any bad code.
	\item \textbf{Gatekeeping}: a senior developer can highlight any potentially bad decisions.
	\item \textbf{Accident Prevention}: can catch bugs and defects.
\end{itemize}

\subsection{Retrospective}\label{sub:retrospective}

Regularly, the team reflects on how to become more effective and adjusts their processes accordingly.
This helps the you to improve as a team.

Boosts shared ownership where more than one person are responsible for a section.
Collaboration where the team all works together and discusses improvements.
A chance to reflect on how the team works (what works, what doesn't what was too fast/slow).

Any measurements made are specific to your own team and will mean nothing to anyone else.
Improvement is an ongoing activity.
Requires the whole team so can be expensive but is worth it.

You are not trying to make the team perfect, you are just trying to make your best effort.
Always look for the route cause and not the symptoms.
Simply working harder will not improve the process and blaming people is pointless and counter-productive.

Always create follow up actions and assign owners to all of them.

\subsubsection{General method}\label{ssub:general_method}

\begin{itemize}
	\item Use icebreakers to let people get to know each other a bit to let people become more relaxed together.
	\item Team members should populate a template board independently using sticky notes. Having an independant person can mean that the person in charge has no previous agenda.
	\item Similar items should be grouped together so that they can be talked about at the same time.
	\item Team votes on priority issues for the discussion (maybe like the beans in the office)
	\item Conduct a discussion to reveal the route cause for any issues encountered
	\item Select actions for improvements with owners (and discuss what actions were completed since last time).
\end{itemize}

\subsubsection{Templates}\label{ssub:templates}

\paragraph{Stop, start, continue}\label{par:stop_start_continue}

What should we stop doing? What should we start doing? What should we keep on doing?

\paragraph{Mad, glad, sad}\label{par:mad_glad_sad}

What is mad and we should stop? What are we glad we are doing and? What are we sad about and should reduce?

\paragraph{Sailboat}\label{par:sailboat}

The wind pushes you forward, the anchor is slowing you down and the rocks indicate risks.

\subsubsection{Getting to a route cause}\label{ssub:getting_to_a_route_cause}

Keep on asking why something did or didn't happen until there is a route cause (``the five whys'').

\subsubsection{Monitoring and measuring}\label{ssub:monitoring_and_measuring}

Use a issue ticket to track and assign to somebody.
You can include previous discussions in future retrospective.
Review the process review process.

\subsection{Improving process improvement}\label{sub:improving_process_improvement}

Reflect on the retrospective itself.
Vary the structure to see what does and doesn't work.
Experiment with different frequencies so that issues don't get forgotten about, but not that people stop caring about it being so often.
Don't always have the same people, get somebody else to lead it (maybe someone external).
