\section{Clean code and refactoring}\label{sec:clean_code_and_refactoring}

\subsection{Why have clean code?}\label{sub:why_have_clean_code_}

\begin{itemize}
    \item It is easier to start and finish working on something.
    \item The code is easier to follow.
    \item Team members are on boarded faster.
    \item Most of you time is spent updating old code so having clean code will make this easier.
    \item You can avoid code duplication since if you understand the code you will be more likely to use it somewhere else.
    \item If only you understand code you will be stuck with it forever and won't ever be promoted.
\end{itemize}

\subsection{Clean Code Rules}\label{sub:clean_code_rules}

\begin{itemize}
    \item A method should only do one thing.
    \item Have a maximum of four or five variables or method calls total (most people can only keep track of this many things at once).
    \item Have good identifier names
    \item Must be quick to read (not too long)
    \item Try to avoid using identifiers with a similar word shape (length, start/end characters) since these are easily confused.
\end{itemize}

\subsection{Why refactor code?}\label{sub:why_refactor_code_}

\begin{itemize}
    \item Clean code is better
    \item Avoids duplication
    \item If only you understand code you will be stuck with it forever and won't ever be promoted.
\end{itemize}

\subsection{Refactoring Rules}\label{sub:refactoring_rules}

\begin{itemize}
    \item A method should do one thing.
    \item There should be no duplication (problems have to be fixed in many places)
    \item Style consistency
    \item ``Later'' tends to be never (do refactoring now since it is familiar now).
    \item Comment rarely (comment and code can end up different).
    \item Remove commented out code (this is what a version control system is for).
    \item Is the code readable and clean?
\end{itemize}

\subsection{How to improve?}\label{sub:how_to_improve_}

\begin{itemize}
    \item Read and write code
    \item Always get your code reviewed
    \item Look at examples of well written code
\end{itemize}

\section{Testing}\label{sec:testing}

\subsection{Why test?}\label{sub:why_test_}

\begin{itemize}
    \item Lets you catch errors early.
    \item Prevent production failures (so the customer doesn't experience the breakages).
    \item Makes code more maintainable (there is regression testing).
    \item To make sure hotfixes don't break what's already working.
\end{itemize}

\subsection{What is testing?}\label{sub:what_is_testing_}

Creating, executing and maintaining tests.
The purpose of testing is:
\begin{itemize}
    \item Prevent the introduction of defects
    \item Demonstrate that an implementation meets the required requirements.
    \item Document typical usecases of the code
\end{itemize}

\subsection{AAA/Triple A/Amahl Ali Akbar}\label{sub:aaa_triple_a_amahl_ali_akbar}

\begin{itemize}
    \item \textbf{A}rrange variables.
    \item \textbf{A}ct: do a thing.
    \item \textbf{A}ssert: check if the thing you did worked (matches an expected value).
\end{itemize}
