\section{Over and under testing}\label{sec:over_and_under_testing}

\subsection{Over Testing Why?}\label{sub:over_testing_why_}

\begin{itemize}
	\item Unnecessary tests can break
	\item Confusion and lost time to determine faults are with tests
	\item Excessive time managing test and reworking them
	\item People can just be making tests to bump coverage.
\end{itemize}

\subsection{Over Testing What?}\label{sub:over_testing_what_}

\begin{itemize}
	\item Testing for position of elements in arrays is bad (unless sorted).
	\item In JSON, look for the presence of attributes and not the position in case the old data changes.
	\item Looking for specific data can be bad.
	\item Testing for assumptions (eg.\ specific currency or timezone) and not requirements is bad.
	\item The developer could guess that there should always be a middle name, but in reality not every one has a middle name.
\end{itemize}

\subsection{Under Testing Why?}\label{sub:under_testing_why_}

\begin{itemize}
	\item Missed tests may break once in production
	\item Lack of regression testing, so there's a lack of confidence to maintain and improve
	\item Technical debt to resolve missing tests (just don't do work for a few weeks while you add tests?)
\end{itemize}

\subsection{Under Testing What?}\label{sub:under_testing_what_}

\begin{itemize}
	\item Are you missing edge cases?
	\item Are you missing tests for some requirements.
	\item You might only be testing for positive paths through the software (are you testing for failures and no responses?).
	\item Are you testing for exceptions?
\end{itemize}
