\section{Improving Software Quality}\label{sec:improving_software_quality}

\subsection{Why is source code quality important?}\label{sub:why_is_source_code_quality_important_}

\begin{itemize}
    \item Code is easier to read and maintain if it is in a consistent style.
    \item Avoids redundant code
    \item Can help to identify issues (even more in interpretted languages because the compiler won't complain here)
\end{itemize}

\subsubsection{Linting}\label{ssub:linting}

Linters are pieces of code that statically analyse your codebase to identify issues and can be run in the IDE or before committing.

\begin{itemize}
    \item Spell checking
    \item Grammar checking (syntax)
    \item Detects undeclared variables
    \item Unused variables
    \item Deprecated functions
    \item Spacing and formatting conventions
    \item Misuse of scope
    \item Implicit fall though (on case and switch statements)
    \item Dangerous language features
\end{itemize}

\subsection{Why is codebase security important?}\label{sub:why_is_codebase_security_important_}

\begin{itemize}
    \item Hackers are always looking for vulnerabilities
    \item Libraries are always improving so updating can improve security (or also new versions can have vulnerabilities)
          \begin{itemize}
              \item Older versions will not always be supported so any new bugs detected might allow attackers in.
              \item Security scanning of library versions
              \item Highlight level of warning.
              \item Suggest resolutions and maybe automatic fixing (make sure you test).
          \end{itemize}
\end{itemize}

\subsubsection{Composition Analysis}\label{ssub:composition_analysis}

\begin{itemize}
    \item Dependency analysis
    \item Library and component analysis
    \item Snippet analysis (does any of your code itself have any vulnerabilities?).
    \item Vulnerability analysis
    \item Remediation guidance
    \item Automation on fix
\end{itemize}

\subsection{Why are metrics important?}\label{sub:why_are_metrics_important_}

\begin{itemize}
    \item Helps to evaluate cost and schedule of future projects
    \item Evaluate the productivity impacts of new tools and techniques
    \item Establish productivity over time (to improve).
    \item Improve software quality
    \item Forecast future staffing needs
    \item Anticipate and reduce future maintenance needs
\end{itemize}

\subsubsection{Types of metrics}\label{ssub:types_of_metrics}

\begin{itemize}
    \item Test coverage
    \item Minimum, maximum and average time to deployment
    \item Minimum, maximum and average number of fixes to a requirement
    \item Correction rate from code review
    \item Code quality
\end{itemize}

\begin{note}
    Your metric should add value. ie.\ requiring a high test coverage may result in having lots of useless tests.
\end{note}

\subsubsection{Metrics for code quality}\label{ssub:metrics_for_code_quality}

\begin{itemize}
    \item Complexity of methods
    \item Depth of inheritance
    \item Number of child classes
    \item Coupling between object classes
    \item Number of responsibilities for a class
    \item Lack of cohesion
\end{itemize}

\paragraph{Complexity of a method}\label{par:complexity_of_a_method}

The complexity of a method is the sum of the complexities of methods in a class.
Complexity is measured using the McCabe complexity.
Less complex code is better.

We can use the Weight Method Per Class (WMC) to predict how much time and effort is required to develop and maintain the class (lowe is better).
CK Metrics Implementation assigns a complexity score of \(1\) to each method

\paragraph{Depth of Inheritance (DIT)}\label{par:depth_of_inheritance_dit_}

The length of the path from the node to the root of the tree.
For greater DITs:
\begin{itemize}
    \item The greater the number of methods it is likely to inherit making more complex to predict its behaviour
    \item The greater potential reuse of methods
    \item The objective is to achieve a trade-off in DIT
    \item Small values of DIT in more of the system's classes may be an indicator that designers are forsaking re-usability for simplicity of understanding.
\end{itemize}

\paragraph{Number of children}\label{par:number_of_children}

For any class in the inheritance tree, the number of children is the number of immediate children of the class (the number of direct subclasses).
For greater values of NOC:
\begin{itemize}
    \item The greater the reuse
    \item The greater the probability of improper abstraction of the parent class
    \item The greater the requirement of method's testing in that class
    \item Small numbers of NOC may indicate lack of communication between class designers.
    \item Aim for a trade-off
\end{itemize}

\paragraph{Coupling between classes}\label{par:coupling_between_classes}

\begin{itemize}
    \item CBO is the number of other classes the class is coupled to.
    \item A class is coupled if it operates on or affects another class or is required by another class.
    \item Improve modularity and promote encapsulation
    \item Indicates independence in the class, making its reuse easier
    \item Makes easier to maintain and to test a class
    \item Try to get low coupling
\end{itemize}

\paragraph{Responses for Class}\label{par:responses_for_class}

This is the number of methods of the class plus the number of methods called by any of those methods.
Normally ROC is calculated to only one level.
Smaller numbers are better and larger numbers indicate increased complexity and debugging complexity making testing and maintenance harder.
Aim for low.

\paragraph{Lack of Cohesion}\label{par:lack_of_cohesion}

Counts the sets of methods in a class that are not related through the sharing of some of the class's fields
\begin{enumerate}
    \item Consider all Paris of a class's methods
    \item Check if the pair share common fields
    \item The lack of cohesion in methods is then calculated by subtracting the number of methods pairs that don't share afield access and the number of pairs that do.
\end{enumerate}

\begin{itemize}
    \item A measure of the tightness of the code
    \item Lower values of cohesion increase complexity and do not promote encapsulation (suggests classes should be split into more than one)
    \item Helps to identify low quality designs
    \item Aim for high cohesion
\end{itemize}
