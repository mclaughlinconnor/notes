\section{Mocking - Testing}\label{sec:mocking_testing}

\subsection{Why?}\label{sub:why_two}

\begin{itemize}
    \item Faster continuous integration because there are no IO operations, database queries or VPNs needed, which means there is less time between finishing code and getting feedback.
    \item There is no adapter design pattern needed making it a bit simpler to implement.
    \item Removes dependencies on external services
    \item Reading from a file will be harder to test
    \item Will still work when a database or REST service is down.
    \item Querying a webpage on a slow internet connection will slow down tests.
    \item Using a 3\textsuperscript{rd} party service might be expensive.
\end{itemize}

\subsection{Mocks in Python}\label{sub:mocks_in_python}

Mocks can be used to specify a return value for all subsequent calls for a function.

\subsubsection{Magic Mock}\label{ssub:magic_mock}

Lets you set a return value, return several values or throw exceptions (we should ideally do all three in our group project).

\subsection{Mocking in Java}\label{sub:mocking_in_java}

In Java you can specify one return value, but inside a class you must use a Spy (which copies a class and alters one method) which requires the use of an adapter design pattern.

\subsection{Challenges}\label{sub:challenges}

You need to make sure you don't get lost in mocking (make sure you aren't only testing mocked functions, don't over do it) -- you should limit your use to only within parameter passing.
``Only test real code''.

Mocking can be a sign of high coupling and in some cases you may not need to test something (like a database connection).

\emph{Blackbox testing} is where only the API endpoints are tested so mocking is probably not needed.
\emph{Whitebox testing} is where the internals of a method are tested so mocking probably will be needed.

Code coverage is how much of your code is tested -- it is good to keep this number high, but beware that there are diminishing returns as you approach \(100\%\).
Case coverage is a count of how many of your user use cases have been covered by the code.

You should try to test for \mintinline{python}{null}s wherever possible, though this might be difficult and any \mintinline{python}{null}s may be caught by a validation layer.
