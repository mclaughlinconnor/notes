\section{Overview}\label{sec:a_overview}

Overall the course will have:

\begin{description}
    \item[\(5\%\)] in class quizzes at the end of each week
    \item[\(15\%\)] for a class test on string and test algorithms
    \item[\(20\%\)] for an assessed exercise
    \item[\(60\%\)] for an exam
\end{description}

Throughout the course we will mostly be using Java, and we will need these things from Practical Algorithms last semester:
\begin{itemize}
    \item Graphs
    \item Sets
    \item Proofs
    \item Abstract data types
    \item Data structures
    \item Big \(O\) notation
    \item Searching
    \item Sorting
\end{itemize}

\section{Big O notation recap}\label{sec:big_o_recap}

The time complexity of a function is a function of the input size, where to worst case is called ``big \(O\)'' and lets us easily determine an algorithm's performance.
We only really care about asymptotic behaviour (increasing to infinity, like an asymptote).

Occasionally, we care about space efficiency too.

\subsection{Big O}\label{sub:big_o}

\(f(n)\) is \(O(g(n))\), where \(f\) is of order \(g\).
Informally, \(f\) grows no faster than \(g\), but formally we can say that:
\[
    | f(n) | \leq | \text{real constant} \times g(n) | \text{for all \(n \geq \text{integer constant}\)}
\]
usually, \(f\) is a complex function which is not known precisely, but \(g\) is is known to be \(O(n)\), \(O(n^2)\) or similar.

We will always use the ``tightest'' \(g\) that we can, eg., for a selection sort, we will use \(O(n^2)\) and not \(O(n)\) (that's too small) or \(O(n^{3})\) (that's too big).

\section{Logs Recap}\label{sec:logs_recap}

The three most important log roles are:
\begin{align*}
    \log n \times m  & = \log m + \log n \\
    \log \frac{n}{m} & = \log m - \log n \\
    \log n^{m}       & =m \log n
\end{align*}
\begin{note}
    Another thing to note is that, since a log of a constant evaluates to a constant, we can consider them constants.
\end{note}

\section{Sorting Recap}\label{sec:sorting_recap}

\begin{itemize}
    \item Selection sort, insertion sort, and bubble sort are naive and slow, with time complexity \(O(n^2)\).
    \item Merge and heap sorts are slightly better with a worst/average case of \(O(n\log n)\).
    \item Quick sort is the fastest with \(O(n \log n)\) on average, but no better than \(O(n^2)\) in the worst case.
\end{itemize}

No pair-wise comparison algorithm can be better than \(O(n \log n)\) because we effectively make a decision tree out of the array, meaning there are \(h\) operations in the worst case, which is less than or equal to \(2^{h+1}-1\), we can combine these to get:
\[
    n! \leq 2^{n+1}-1 \leq 2^{n+1}
\]
When we take the log of both sides of \(2^{h+1} \geq n!\), we get:
\begin{align*}
    h+1 & \geq \log_2(n!)                             \\
    h+1 & > \log_2 (\frac{n}{2})^{\frac{n}{2}}        \\
    h+1 & = \frac{n}{2}\log_2(\frac{n}{2})            \\
    h+1 & = \frac{n}{2}\log_2 n - \frac{n}{2}\log_2 2 \\
    h+1 & = \frac{n}{2}\log_2 n - \frac{n}{2}
\end{align*}
So the time complexity is at least \(O (n log n)\).

\section{Sorting}\label{sec:sorting_a}

To be fast, we can't use pair-wise comparisons, Radix sort uses the binary structure to be \(O(n)\).
We have to assume that the items can be bit strings of length \(m\).
We will let \(b\) be a chosen factor of \(m\) so that \(b\) and \(m\) are constant.
\begin{itemize}
    \item \(b=1\) leads to \(m\) iterations and \(2\) buckets
    \item \(b=m\) leads to \(1\) iteration and \(2^{m}\) buckets.
\end{itemize}

We will label each bit position in a binary number from \(0\) to \(m-1\) starting from the least significant digit (on the right).
We will label our buckets from \(0\) to \(2^{b-1}\).

Using a \(b=4\) and a current iteration (i=2), we will use bits \((b \times i)-1...b \times (i-1)\), so \(7, 6, 5, 4\).
We evaluate the number we get  \(0101 = 5\), so we put the whole value in bucket \(5\).

At the end of each iteration, we concatenate the buckets, then repeat from the start.

\subsection{Correctness}\label{sub:correctness}

For \(x\) and \(y\) where \(x<y\), the relevant bits of \(x\) will be smaller than those in \(y\) and \(x\) will go into an earlier bucket than \(y\).
In all later iterations the order of \(x\) and \(y\) is preserved, so they will be sorted.

\subsection{Complexity}\label{sub:complexity}

The number of iterations is \(\frac{m}{b}\)and the number of buckets is \(2^{b}\).
\begin{itemize}
    \item Allocating each bucket takes \(O(n)\)
    \item Concatenating takes \(O(2^{b})\)
    \item Complexity is \(O(\frac{m}{b} \times (n + 2^{b}))\) or \(O(n)\).
\end{itemize}

\begin{note}
    The larger \(b\) is , the smaller \(\frac{m}{b}\) is making the sort faster, however an internal array of size \(2^{b}\) is needed.
\end{note}

\section{Tries}\label{sec:tries}

A binary search tree is comparison based, a trie uses a similar system to how radix sort works.

Each items is a sequence of bits and each node in the tree has a multi-way branch where each branch has a symbol and no \(2\) siblings have the same symbol.
The branch taken at a level \(i\) is the \(i^{\text{th}}\) bit of the original value.
Tracing a path to a node spells out the value.

We can implement a trie as a linked list or an array:
\begin{description}
    \item[array] The array is full of pointers to the node's children.
    \item[list] The linked list is basically the same as how a tree is represented with a linked list.
\end{description}

\section{String Compression}\label{sec:string_compression}

We are trying to make string smaller.
Text compression is a special case of data compression, where we must ensure lossless encoding (a pixel can change, but a letter cannot).

There are two approaches: statistical and dictionary.

\subsection{Definitions}\label{sub:definitions}

The compression ratio is \(\frac{\text{compressed size}}{\text{original size}}\), the percentage saved is \(1-\text{compression ratio} \times 100\%\).
The usual savings for a compressions are \(40\%\) to \(60\%\).

\subsection{Huffman Encoding}\label{sub:huffman_encoding}

Huffman encoding is generally less effective than modern dictionary compression (this is a statistical approach).
In Huffman, a fixed ASCII code is replaced by a variable length code for each character, where each code is not a prefix of any other to give us unambiguous decompression.

Huffman encoding is represented on a tree where each leaf node is a character, and the route to that character is the code.
The unique prefixes rule follows from this.

The most common letters have the shortest codes because they are placed at the top of the tree.

\begin{note}
    At each tree fork, going left gives \(0\), going right gives \(1\).
\end{note}

\subsubsection{Optimality}\label{ssub:optimality}

The weighted path length (WPL) of a tree, \(T\), is:
\[
    \Sigma (\text{weight}) \times (\text{distance from root})
\]
A Huffman tree always has the minimum WPL of any binary tree given the same leaf weights.
It should be noted, though, that Huffman trees are not unique, but that Huffman trees minimise the WPL (the number of bits in the compressed file), so a Huffman tree is optimal.

\subsubsection{Complexity}\label{ssub:complexity_ht}

If \(m\) is the number of distinct characters, and \(n\) is the length of the text, then
\begin{itemize}
    \item The complexity to calculate the frequencies is \(O(n)\).
    \item To build the tree, we have to:
          \begin{itemize}
              \item Create nodes \(O(n)\).
              \item Insert a node \(O(\log m)\)
              \item \(m-1\) total iterations to insert all
          \end{itemize}
\end{itemize}
So the complexity is \(O(n \times \log m)\), but since \(m\) is constant, we say \(O(n)\).

\subsubsection{Compression}\label{ssub:compression}

Compression uses a table of codes indexed by character, so we need \(O(m \log m)\) to build the table as there are \(m\) characters, so \(m\) paths of length \(\leq \log m\).
Then we need \(O(n)\) to actually compress, for \(O(n \log m) + O(n)\) overall.

\subsubsection{Decompression}\label{ssub:decompression}

We are basically just tracing paths in the tree, so the complexity is \(O(n \log m)\) since there are \(n\) characters, so \(n\) paths of length \(\leq \log m\).

\subsubsection{Algorithmic Requirements}\label{ssub:algorithmic_requirements}

We must store some representation of the tree in the compressed file so that we can decompress later.

Alternatively, we can use the same set of frequencies, but this would reduce the efficacy of the algorithm.

Alternatively alternatively, we could build the same tree, but adapt it in the compressor and the decompresser as characters are encoded or decoded, but this will slow down the algorithm.

\subsection{LZW Compression}\label{sub:lzw_compression}

LZW is dictionary based, where the dictionary stores a collection of strings.

Each dictionary value has a fixed length (or a length that increases as the algorithm continues depending on the algorithm) codeword which is a bit pattern that can be interpreted as a number if needed.
The bit pattern is written to file.

The dictionary is built up as the algorithm progresses, all that is known at the start is the codewords to represent every single character (both the compressor and decompresser have the same starting dictionary).
Throughout the execution, the dictionary is ``closed under prefixes'' meaning that, if we have a string \(s\), we will have every prefix of \(s\).

A trie is an ideal representation where each node is a codeword the dictionary.

\subsubsection{Variable Length Codewords}\label{ssub:variable_length_codewords}

At any time during compression and decompression, there is a current codeword length, \(k\).
There are \(2^{k}\) codewords possible with a length of \(k\).

In order to have more than \(2^{k}\) codewords, we can increase \(k\) to double the total number.
The initial value for \(k\) should be large enough to capture all single character strings.
