\section{XML and JSON}\label{sec:xml_and_json}

\subsection{Markup Languages}\label{sub:markup_languages}

All markup languages used in modern software are descended from the Standard General Mark-up Language.
A mark-up language defines the content of a document using \emph{tags}.

\subsection{XML}\label{sub:xml}

Stands for the extensible markup language and is developed by the W3C.
It is designed to transport and store data.
XML is \textbf{everywhere}.

\subsubsection{XML Design Goals}\label{ssub:xml_design_goals}

The W3C standardised XML because each web browser was supporting different HTML features in inconsistent ways and SGML was far too complex.
HTML was too limited and mixed the format of a document with the structure of the document (separation of concerns).

XML aimed to provide a simple mark-up language (better than SGML), separate the format from the structure, be extensible (be useful in lots of different applications), and transport and store data.

\subsubsection{The Role of XML}\label{ssub:the_role_of_xml}

\begin{itemize}
    \item Describe the structure of semi-structured documents.
    \item To provide principle mechanisms for charring, transporting and storing annotated data.
    \item Be general purpose for data description and interchange.
\end{itemize}
As a result XML has:
\begin{itemize}
    \item Emerged as the dominant standard
    \item Developed lots of vocabularies for different usages.
    \item Additional tools for additional processing layers like: querying, formatting and transforming.
\end{itemize}

\subsubsection{XML Extensions}\label{ssub:xml_extensions}

\begin{description}
    \item[XHTML] Web pages
    \item[WML] Special XML for old mobile websites
    \item[MathML] The language of mathematics
    \item[SOAP] describes method parameters between web services (a bit like REST).
\end{description}

\subsubsection{XML vs HTML}\label{ssub:xml_vs_html}

\begin{itemize}
    \item HTML was designed to display data, as a result, HTML combines data and the presentation of the data together. XML provides only the data.
    \item The structure of XML is tightly controlled, HTML is very must not like that.
          \begin{itemize}
              \item Tags are case sensitive
              \item Values must be quoted
              \item If there is a start tag, there must be an end tag.
              \item There is a strict hierarchical structure
          \end{itemize}
    \item XML is flexible and new tags can be created whenever you want, you cannot with HTML.
\end{itemize}

\subsubsection{XML Document Structure}\label{ssub:xml_document_structure}

\begin{enumerate}
    \item An optional prologue: \mintinline{xml}{<?xml version="1.0" encoding="UTF-8"?>}.
          \mintinline{xml}{version} is either \(1.0\) or \(1.1\).
          We also have \mintinline{xml}{standalone} which if \mintinline{xml}{yes} (default)it means the document is entirely self-contained, otherwise it has an external schema.
    \item The body which contains the actual data.
    \item An optional epilogue containing processing instructions and other comments.
\end{enumerate}

\subsubsection{XML Elements}\label{ssub:xml_elements}

Elements are the basic building blocks of XML and included \emph{everything} in the entire document.
An element can contain text, attributes, other elements, or a mixture.

\subsubsection{XML Attributes and Values}\label{ssub:xml_attributes_and_values}

Attributes are characteristics of elements which are case sensitive and whose values must be quoted.
Some special characters are difficult to include in a value so we use \mintinline{xml}{&lt;}, etc. instead.

\subsection{Well-formed XML}\label{sub:well_formed_xml}

An XML document is ``well formed'' if:
\begin{itemize}
    \item Tags are case sensitive.
    \item There must be start and end tags.
    \item Tags are closed in the order they are opened (hierarchical).
    \item Values are quoted.
    \item There must be a single root element.
\end{itemize}

\subsection{Valid XML}\label{sub:valid_xml}

Valid XML is XML that is well formed and follows a pre-defined structure (has the correct tags, etc.).
We can define these using Document Type Definitions (DTD) or the more complex XML Schemas and XML Name spaces.

XML can be checked and validated against the definitions.
The structures can be external or internal.
Name spaces are used to avoid element name conflicts for elements in different contexts.

\begin{note}
    Name spaces are actually URLs since these are guaranteed to be unique.
\end{note}

\subsection{DTDs vs Schemas}\label{sub:dtds_vs_schemas}

Why are XML schemas better than DTDs?
\begin{itemize}
    \item Schemas are written in XML
    \item XML schemas are extensible additions (because they are XML)
    \item XML schemas support data types
    \item XML schemas support name spaces
\end{itemize}
Why use an XML schema?
\begin{itemize}
    \item With schemas, XML files can carry a description of its own format.
    \item XML schemas can be used by many people to agree on a common interface.
    \item You can verify your data.
\end{itemize}

\subsection{Formatting XML}\label{sub:formatting_xml}

We have XSL (extensible style sheet language) to style XML.
We start with the raw XML data, create the XSL style sheet, then link the two together.

\begin{note}
    XSLT also exists which lets you transform XML documents into other formats (like HTML).
\end{note}

\subsection{Processing XML}\label{sub:processing_xml}

There are two ways of using XML in a program:
\begin{itemize}
    \item The document object model (DOM) can build an in-memory hierarchical model of the XML elements which is useful if you need the whole document.
    \item Simple API for XML (SAX) provides an event driven parser for XML which is better to just get small parts of data (doesn't load it all to memory).
\end{itemize}

\subsubsection{DOM Parsing}\label{ssub:dom_parsing}

\begin{enumerate}
    \item Load the XML document
    \item Find the element we want by searching or traversing the tree.
    \item For the element, extract, modify, or remove attributes, values, or data.
    \item Repeat until done
\end{enumerate}

\begin{minipage}[t]{0.45\linewidth}
    Pros:
    \begin{itemize}
        \item Can handle the entire document at once
        \item Easier to program
    \end{itemize}
\end{minipage}
\hfill
\begin{minipage}[t]{0.45\linewidth}
    Cons:
    \begin{itemize}
        \item Uses more memory
        \item Usually slower
        \item Can process large files through slow disk caching
    \end{itemize}
\end{minipage}

\subsubsection{SAX Parsing}\label{ssub:sax_parsing}

SAX parsing has event handlers which are triggered for open and closing tags of specific types.

\begin{enumerate}
    \item Create a custom object model (like a class) to represent each element.
    \item Create a SAX parser.
    \item Create a DocumentHandler to turn the document into instances of the custom object model.
\end{enumerate}

\begin{minipage}[t]{0.45\linewidth}
    Pros:
    \begin{itemize}
        \item Uses less memory
        \item Usually faster
        \item Can process files larger than memory
    \end{itemize}
\end{minipage}
\hfill
\begin{minipage}[t]{0.45\linewidth}
    Cons:
    \begin{itemize}
        \item Cannot handle all parsing tasks directly (to validate entire document, several parses needed).
        \item Requires more programmer effort
    \end{itemize}
\end{minipage}

