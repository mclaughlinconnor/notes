\section{CSS Box Model}\label{sec:css_box_model}

\begin{description}
	\item[Padding] adds space around an element's content inside any borders
	\item[Margins] add space around elements outside of borders
	\item[Borders] sit between padding an margins (these are only visible if a \mintinline{css}{border-style} has been set).
\end{description}

\section{Entities, Relationships and Django Models}\label{sec:entities_relationships_and_django_models}

\subsection{Entity Relationship Model}\label{sub:entity_relationship_model}

An \emph{entity} is a thing and is similar to a class in OOP.
Provides an abstract representation of the data and how they are related to each other.
We have relationships between the entities like ``belongs'', ``owns'', etc.
\emph{Attributes} is the information about an entity.

\subsubsection{Notations}\label{ssub:notations}

There are many different notations.
We will be using the \emph{Compressed Chen Notation} which uses rectangles to represent entities, diamonds to represent relationships and \(1\), \(N\) and \(M\) to notate cardinality.

\begin{center}
	\begin{tikzpicture}[auto,node distance=1.5cm]
		\node[entity] (person) {Person};
		\node[relationship] (has) [below right = of person] {Has} edge node {\(M\)} (person);
		\node[entity] (house) [above right = of has] {House} edge node[midway] {\(1\)} (has);
	\end{tikzpicture}
\end{center}

\subsubsection{Many to Many}\label{ssub:many_to_many}

For many to many, we use \(M\) to \(N\) so that it is clear that the two numbers can be different.
\begin{center}
	\begin{tikzpicture}[auto,node distance=1.5cm]
		\node[entity] (student) {Student};
		\node[relationship] (enrolls) [right = of student] {Has} edge node {\(N\)} (student);
		\node[entity] (course) [right = of enrolls] {Course} edge node {\(M\)} (enrolls);
	\end{tikzpicture}
\end{center}
Because it can be hard to represent a many to many relationship in a database we can also do something like:

\begin{center}
	\begin{tikzpicture}[auto,node distance=1.5cm]
		\node[entity] (student) {Student};
		\node[relationship] (has) [below = of person] {Has} edge node {\(1\)} (student);
		\node[entity] (enrollment) [right = of has] {Enrollment} edge node {\(N\)} (has);
		\node[relationship] (in) [right = of enrollment] {In} edge node {\(N\)} (enrollment);
		\node[entity] (course) [above = of in] {Course} edge node {\(1\)} (in);
	\end{tikzpicture}
\end{center}

\subsubsection{Keys}\label{ssub:keys}

The \emph{primary key} is a field (or fields) that uniquely identify each record.
Primary keys must contain unique values and cannot contain null values.
A table can only have one primary key.

A \emph{foreign key} is used to link two tables together.
A foreign key should point to a primary key on another table.

\subsubsection{Compressed Chen Notation}\label{ssub:compressed_chen_notation}

The difference between Chen and Compressed Chen notation is that with regular Chen, the attributes are all around the entities and with Compressed Chen, we just use a table.

\paragraph{Chen}\label{par:chen}

Chen with the combined entities and attributes:

\begin{center}
	\begin{tikzpicture}[auto,node distance=1.5cm]
		\node[entity] (student) {Student}
		[grow=up,sibling distance=3cm]
		child {node[attribute] {Age}}
		child {node[attribute] {Name}}
		child {node[attribute] {Student ID}};
		\node[relationship] (enrolls) [right = of student] {Has} edge node {\(N\)} (student);
		\node[entity] (course) [right = of enrolls] {Course} edge node {\(M\)} (enrolls);
	\end{tikzpicture}
\end{center}

\paragraph{Compressed Chen}\label{par:compressed_chen}

Compressed Chen with the separate table:

\medskip
\begin{minipage}{0.50\linewidth}
	\centering
	\begin{tikzpicture}[auto,node distance=1.5cm]
		\node[entity] (student) {Student};
		\node[relationship] (enrolls) [right = of student] {Has} edge node {\(N\)} (student);
		\node[entity] (course) [right = of enrolls] {Course} edge node {\(M\)} (enrolls);
	\end{tikzpicture}
\end{minipage}
\hfill
\begin{minipage}{0.40\linewidth}
	\centering
	\begin{tabular}{cc}
		\toprule
		Field      & Type    \\
		\midrule
		Age        & Integer \\
		Name       & String  \\
		Student ID & String  \\
		\bottomrule
	\end{tabular}
\end{minipage}



\subsection{Creating Tables}\label{sub:creating_tables}

To create tables we can use:
\begin{itemize}
	\item SQL \mintinline{sql}{CREATE} statements
	\item phpMyAdmin
\end{itemize}

\subsubsection{Converting to Django Models}\label{ssub:converting_to_django_models}

In Django, instead of creating tables manually, we can just create a few classes.
In Django, every model is automatically assigned an id.
To create relationships, we can just refer to the model instead of the id itself.

\begin{minted}[linenos,numbersep=5pt,frame=lines,framesep=2mm]{python}
class House(models.Model):
    ...

class Person(models.Model):
    # Every person must live in a house
    house = models.ForeignKey(House)
\end{minted}

\section{Presentations}\label{sec:presentations}

A presentation should be given in class on 2021-04-23 for a total of \(5\%\) of your total grade.

Each group will be asked to present the design specification for a web application which you won't have to actually implement.
The presentation should include:
\begin{itemize}
	\item Overview
	\item Personas
	\item A specification
	\item System architecture design
	\item ER diagram
	\item User interface wireframes
	\item A list of technologies to use
\end{itemize}
