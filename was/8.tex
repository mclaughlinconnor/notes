\section{jQuery}\label{sec:jquery}

JavaScript has very few built in tools (standard library) so we have to write lots of boilerplate code to get anything working.
We should use an extra standard library that focuses on a specific task (user interaction, animation, etc.)
We have:
\begin{itemize}
	\item jQuery
	\item Prototype
	\item Yahoo UI
	\item Moot Tools
\end{itemize}

\subsection{jQuery}\label{sub:jquery}

One of the most popular JavaScript libraries that simplifies client-side scripting.
\begin{itemize}
	\item Selecting DOM elements
	\item Creating UI animations and elements
	\item Handling events
	\item Developing AJAX applications
\end{itemize}

\subsection{Cross Browser Compatibility}\label{sub:cross_browser_compatibility}

Since lots of browsers support different code, we might have to use different JavaScript for different browsers (using browser sniffing).
jQuery fixes this by acting as a layer of abstraction over the browser.

\subsection{Plugins}\label{sub:plugins}

jQuery creases a foundation for additional functionality to be added.
There are lots of added features for images, animations and lots more.

\subsection{Syntax}\label{sub:syntax}

jQuery uses a patterns of selecting and acting on a particular DOM element and altering its parameters.
Selectors can be reused.
\begin{highlight}{jQuery basic syntax}
	\begin{code}{javascript}
		$('p').css('color', 'blue');
	\end{code}
\end{highlight}
jQuery uses the pattern, making use of anonymous functions and chaining functions together.
We can use \mintinline{javascript}{(document).ready} to make sure the document is loaded before we to anything to it.

\begin{note}
	Instead of using \mintinline{javascript}{$(document).ready()}, we can use \mintinline{javascript}{$()} as shorthand.
\end{note}

\subsection{Attaching Event Handlers}\label{sub:attaching_event_handlers}

\begin{highlight}{jQuery event handlers}
    \begin{code}{javascript}
		$(document).ready(function() {
			$("#button").click(function() {
				$("#message").css("color", "blue");
				if ($("#message").is(":visible")) {
					$("#message").hide();
				} else {
                    $("#message").show();
				}
			})
		})
	\end{code}
\end{highlight}
