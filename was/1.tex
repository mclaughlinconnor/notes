\section{Overview}\label{sec:was_overview}

All Wednesdays will be lab sessions.

The course is about web applications in the broadest sense, but does acknowledge that most modern web development is done using application frameworks.
We will also cover networked and distributed systems with respect to the communications and protocols involved in a web application's design.

This is a \(10\) credit course so this will take \(\frac{1}{3}\) of the effort of HTLANL

We will be using Python with Django, but little actual Python will be written (we will use lots of HTML, JS, Ajax, etc.).

\subsection{Course Aims}\label{sub:course_aims}

\begin{itemize}
    \item Explore the tools and technologies used
    \item The strengthen understanding of the context and rationale of distributed systems (like many servers and many users).
          We will discuss the ``separation of concerns'' extensively (MVC architecture).
    \item To promote the disciplined design and development of distributed web applications (use things like wireframes and UML diagrams).
    \item To understand the messaging and protocols used as communication mechanism in web applications.
    \item To develop the ability to implement and deploy distributed web applications
    \item \textbf{There will be a heavy focus on principled design mastery of languages and excellence in coding.}
\end{itemize}

\subsection{Assessments}\label{sub:was_assessments}

\begin{description}
    \item[Practical Test] Designed to test ability to set up a simple Django web application during class.
    \item[Presentation] Group presentation given during class to present a design specification for a web application (there will be deltas). Teams are based on the amount of prior web development experience.
    \item[In-class quiz] A timed online Moodle-based multiple choice quiz with \(10\) questions (correct answers give you \(3\) points, incorrect points subtract \(1\) point).
    \item[Exam] Is worth \(70\%\) of your overall course grade. Takes place on 2021-05-19 at 09:30. Will be open for \(24\) hours, but you may only work on it for probably \(3\) hours.
    \item[Coursework] Will be the Rango web application to be developed individually (with support from teams).
        The application will be based on the \emph{Tango with Django 2} book.
        You should have a GitHub repository with commits for each chapter and the final application should be hosted on PythonAnywhere.
\end{description}

\section{Web Applications}\label{sec:web_applications}

\subsection{What is a web application}\label{sub:what_is_a_web_app}

A web application is a Distributed Information Management system (DIM).
Most large companies are web applications (Facebook, Amazon, Expedia, Youtube) which are geographically distributed to be closer to all of the many customers.
A web application allows for the management, sharing, finding, modification and presentation of information.
Ideally a DIM would allow users to access relevant data in a timely fashion.

\subsection{Types of Architecture}\label{sub:types_of_architecture}

\begin{description}
    \item[System Architecture] You build the infrastructure for the system and users interact with that.
    \item[Information Architecture (user focused)] You create an environment for users and clients to interact with the system effectively and efficiently, then tailoring the system to their needs
\end{description}

\subsubsection{Design Elements}\label{ssub:design_elements}

For a information architecture system we can use:
\begin{itemize}
    \item User Personas
    \item User needs matrix
    \item Site / URL design
    \item Wire frames and walk-throughs
    \item Sequence diagrams to show data transfer
    \item ER diagrams for database structures.
\end{itemize}
For a system architecture we use a slightly different set of design elements.
\begin{itemize}
    \item Specifications and requirements
    \item High level system architecture
    \item Sequence diagrams
    \item Entity relationship diagrams
\end{itemize}

\subsection{System Components}\label{sub:system_components}

We use a diagram called an n-tier structure to show the different stages of a particular application.
We will generally be working on the middleware or logic tier.
\begin{center}
    \begin{tikzpicture}
        \draw
        node (u) {User}
        node[right=of u] (c) {Client}
        node[right=of c] (m) {Middleware}
        node[right=of m] (p) {Database};
        \draw[<->] (u) -- (c);
        \draw[<->] (c) -- (m);
        \draw[<->] (m) -- (p);
    \end{tikzpicture}
\end{center}

\subsection{User}\label{sub:user}

A user could be a human or machine which initiates contact and interacts with the client.
There will be a range of skills and abilities.
Each user will have a specific set of requirements to be satisfied.

\subsection{Client}\label{sub:client}

The client is a program witting on a client device which accepts response messages and either communicates to the user or affects the environment in some way in response.
A client should be able to request messages.

\subsubsection{Messaging}\label{ssub:messaging}

The request message is sent to the server from a client to ask for information or send some information to be stored (user input of data from a device).

The response message is sent from a server to a client to return the requested information of to affect the environment in some way.

\paragraph{Messaging Protocols}\label{par:messaging_protocols}

The request message protocol is usually an HTTP request which embeds any data to be sent.

The response message protocol is also an HTTP response, but with content what is usually XML (XHTML, WML, etc.).

\subsection{Middleware}\label{sub:middleware}

The middleware of application server is the central components which sits between the hard data (database) and the client.
The middleware should accept request messages and returns response messages and co-ordinates the application components.
Many different databases and operating systems can be used because the middleware will take care in the integration.

\subsection{Backend}\label{sub:backend}

The backend or data base is typically on a separate node and will store and provide data.
A backend needs to be scalable and reliable so could be a database, index or just a flat file.

\subsection{Web Development Complexity}\label{sub:web_development_complexity}

\begin{description}
    \item[Collision of Languages] Markup, programming, database query languages.
    \item[Shifting standards] Document object model, XML/JSON.
    \item[HTTP is a stateless protocol] However, most applications require the persistence of state (use cookies).
\end{description}

\subsubsection{It's getting better}\label{ssub:it_s_getting_better}

\begin{itemize}
    \item Web development makes lots of money now.
    \item We now use classical software engineering best practices like APIs, libraries, frameworks, tools and software.
\end{itemize}

\subsection{Web Development Tools}\label{sub:web_development_tools}

\subsubsection{IDE}\label{ssub:ide}

The nature of web development is disjoint.
A developer must be familiar with a set of distinct an typically non-integrated tools.
Web development tool support is not yet as advanced as with classic software development.

\subsubsection{Checking Code}\label{ssub:checking_code}

Interpreted languages lack the compilation stage where errors and warnings can be raised.
Some scripts will run and just fall over when they error out.
We use tools like PyLint, ESLint and JSLint to perform static analysis to spot errors.

\subsection{Understanding W3C Standards}\label{sub:understanding_w3c_standards}

Most aspects of software that underpins the web is specified by the World Wide Web Consortium that create working drafts, create candidate recommendations (publiched to gather implementation experience and feedback), proposed implementation (sent to advisory committee), recommended by W3C.

\subsubsection{Complying With Standards}\label{ssub:complying_with_standards}

Browsers will try to compensate for bad web code, which leads to poor software.
You should check your code against the standards.
