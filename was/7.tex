\section{Client Side Scripting - JavaScript}\label{sec:client_side_scripting}

\textbf{JavaScript is not Java}.
The name is probably a marketing idea and was originally called \emph{LiveScript} which was created by the folks at \emph{Netscape} for \emph{Navigator}.

JavaScript looks like a procedural language, but it is probably closer to being a functional language since functions are first class objects.
The support of anonymous functions is heavily used by jQuery.

Earlier versions of JavaScript lacked object oriented programming support and excepting handling.
Newer versions (those after Netcape) are handled by ECMA.
ECMA are not very good at writing standards so JavaScript has lots of issues (it's a bit better now).

\subsubsection{Design Errors}\label{ssub:design_errors}

\begin{itemize}
	\item There are very few useful error messages (also helped by there being no compiler since it is interpretted).
	\item Semi-colons are required, but the programmer is not required to put them in since the interpreter just guesses where to put them.
	\item Type coercion is super funky with numbers and strings being coerced to each other with the \mintinline{javascript}{+} operator, for example.
\end{itemize}

\subsubsection{Object Oriented Programming}\label{ssub:object_oriented_programming}

JavaScript has objects with encapsulate data and methods, but all properties are public.
Inheritance is super broken.

\subsubsection{Broken Implementations}\label{ssub:broken_implementations}

Each browser can implement its own engine.
These engines could be extra buggy.
The JavaScript performance was helped make this better though.

\subsubsection{Accessibility}\label{ssub:accessibility}

JavaScript is very easy to write, so lots of amateur programmers can write terrible code.

\subsection{Core Features}\label{sub:core_features}

\begin{itemize}
	\item Interpreted so no compiler is needed.
	\item Some typing (of the dynamic nature) with some basic primitive types.
	\item Syntactically similar to Java or C.
	\item Kind of object oriented.
	\item Has first class anonymous functions.
	\item Scripts can be used inside of HTML which is good for experimentation, but bad for separation of concerns.
\end{itemize}

\subsection{Ways of Including JavaScript}\label{sub:ways_of_including_javascript}

\subsubsection{Embedded}\label{ssub:embedded}

Having all of the code in the \mintinline{html}{<head>} section of the code.
This is bad for the separation of concerns.

\subsubsection{External JavaScript}\label{ssub:external_javascript}

We have an external JavaScript file that is then included like a style sheet in the \mintinline{html}{<head>} section.

\subsection{DOM Integration}\label{sub:dom_integration}

The intent behind JavaScript is to dynamically manipulate documents.
HTML documents are modelled using the DOM.
DOM methods and properties can be modified using JavaScript.

\subsection{Regular Expressions}\label{sub:regular_expressions}

A regular expression is an object that describes a pattern of text that can be matched against and replaced (if needed).
RegExps can be considered a program within a program which is probably why they are quite tricky.

\subsubsection{RegEx in JavaScript}\label{ssub:regex_in_javascript}

In JavaScript, regular expressions are represented as RegExp objects.
\begin{minted}[linenos,numbersep=5pt,frame=lines,framesep=2mm]{javascript}
let pattern = /colou?r/i; // Matches "colour" and "color" case-insensitivly
\end{minted}
The JavaScript \mintinline{javascript}{search} function can be called on a string with a RegExp as the parameter to search (returns the character index of the match).
The JavaScript \mintinline{javascript}{exec} function can be called on a RegExp object with a string as the parameter returns the matching text, use \mintinline{javascript}{test} to return a boolean.
