\section{Threading}\label{sec:threading}

\begin{code}{c}
#include <stdlib.h>
#include <stdio.h>
#include <pthread.h>
#include <unistd.h>

typedef struct account {
    int amount;
    pthread_mutex_t mut;
} acc_t;

typedef struct instruction {
    acc_t * source;
    acc_t * destination;
    int toTransfer;
} ins_t;
 
void * transfer(void * args) {
    ins_t * t = (ins_t *)arg;

    // Start a lock on the resources
    // This will make other threads wait until it is unlocked again.
    // This can cause a deadlock if the source is reserved as a destination
    // on another thread, because now source has been reserved on this
    // thread, so the other thread cannot continue.

    // Note it is still possible to update t->source, you just can't
    // lock the mutex again.

    // Note it is also possible to just unlock here instead of on
    // another thread.
    // pthread_mutex_lock(t->source->mut);
    // pthread_mutex_lock(t->destination->mut);

    int waittime = 1;

    pthread_mutex_lock(t->source->mut);
    // Returns non-zero if we can't lock
    while (pthread_mutex_trylock(t->destination->mut) != 0) {
        pthread_mutex_unlock(&t->source->mut);

        // We can just use 1 since there is already some randomness
        // from the operating system's scheduler, but we can do this
        // if we really care...
        sleep(waittime); // From unistd.h, or windows.h

        waittime = waittime * 2

        pthread_mutex_lock(&t->source->mutex);
    }

    int src_amt = t->source->amount;
    int dest_amt = t->destination->amount;

    printf(
        "Transferring %d from source (%d) to destination (%d).\n",
        t->toTransfer,
        src_amt,
        dest_amt
    );

    t->source->amount = src_amt - t->toTransfer;
    t->destination->amount = dest_amt + t->toTransfer;

    // Release the lock again
    pthread_mutex_unlock(&t->source->mut);
    pthread_mutex_unlock(&t->destination->mut);

    pthread_exit(NULL);
}

int main() {
    acc_t connor = {50};
    acc_t tom = {200};

    pthread_mutex_init(&connor.mut, NULL);
    pthread_mutex_init(&tom.mut, NULL);

    ins_t firstTransfer = {&connor, &tom, 20};
    ins_t secondTransfer = {&connor, &tom, 15};
    ins_t thirdTransfer = {&connor, &tom, 25};
    ins_t fourthTransfer = {&tom, &connor, 80};
    ins_t fifthTransfer = {&tom, &connor, 10};

    pthread_t transfer1, transfer2, transfer3, transfer4, transfer5;

    pthread_create(&transfer1, NULL, transfer, &firstTransfer);
    pthread_create(&transfer2, NULL, transfer, &secondTransfer);
    pthread_create(&transfer3, NULL, transfer, &thirdTransfer);
    pthread_create(&transfer4, NULL, transfer, &fourthTransfer);
    pthread_create(&transfer5, NULL, transfer, &fifthTransfer);
}

\end{code}

