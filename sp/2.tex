\section{Pointers}\label{sec:c_pointers}

A memory address points to a location in memory where data can be stored.
When you pass a parameter to a function, the parameter data is copied to the new function (basically passing by value).

\begin{code}{c}
    void increment(int *n) { // Take an int pointer
        // The value at n's memory address is the value at n's memory address + 1
        *n = *n + 1;
    }

    int main() {
        int number = 6;
        increment(&number); // Pass the memory address of number
    }
\end{code}
\begin{note}
    \begin{description}
        \item[\&] Gets the address of a variable
        \item[*] Gets the value at an address
    \end{description}
    \begin{code}{c}
        *(&number) == number;
    \end{code}
\end{note}

\section{Pointers to Pointers}\label{sec:pointers_to_pointers}

\begin{code}{c}
    void increment_two(int *n) { // Take an int pointer
        // The value at n's memory address is the value at n's memory address + 1
        *n = *n + 1;
    }

    void increment_one(int **n) { // Pointer to a pointer to an int
        increment_two(*n); // Pointer to an int
    }

    int main() {
        int number = 6;
        int *pointer = &number; // Pointer to number
        increment(&pointer); // Pass pointer to pointer (double layer pointer)
    }
\end{code}

\section{Pointer Arithmetic}\label{sec:pointer_arithmetic}

\begin{code}{c}
    int main() {
        int a = 5;
        int b = 6;

        int *pointer_to_a = &a;
        *pointer_to_a = *pointer_to_a + 1; // Increments a

        // Increments the memory address after a, happens to be b here
        *(pointer_to_a + 1) = *(pointer_to_a + 1) + 1 
    }
\end{code}

\section{Arrays}\label{sec:c_arrays}

\begin{code}{c}
    int main() {
        // In memory, this is 5 consecutive memory locations
        int arr[5] = {1, 2, 3, 4, 5};

        // An array variable is a pointer to the start of a section of memory       
        // These are both "syntactic sugar" for each other
        int a = arr[2] // 3
        int b = *(arr+2) // 3
    }
\end{code}

\subsection{Appending To Arrays}\label{sub:c_appending_to_arrays}

\begin{code}{c}
    // An array variable is a pointer to a memory block.
    // We can also use arr[] instead of *arr too.
    void append(char *arr, char element, int index) {
        arr[index] = element
    }

    int main() {
        char arr[5] = {'a', 'b', 'c'}

        append(arr, 'd', 3)
        // Accessing data beyond the bounds of an array is undefined
        // behaviour and may do different things on different operating
        // systems
    }
\end{code}

\section{Nested Arrays}\label{sec:c_nested_arrays}

\begin{code}{c}
    int main() {
        int *a_value; // Pointer to an int
        int **some_value; // Pointer to a pointer to an int

        int arr_one[2] = {6, 5};
        int arr_two[2] = {2, 1};

        int **arrays[2] = {arr_one, arr_two};
        // arrays points to a memory location that points to
        // the start of arr_one
\end{code}

\section{Strings}\label{sec:c_strings}

\begin{code}{c}
    int main() {
        // C doesn't have strings, so we use character arrays instead
        char one_string[5] = {'a', 'b', 'c', 'd'};
        char another_string[5] = "abcd"; // Notice the double quotes

        printf("%s\n", another_string); 
        // How does the compiler know the length is 4 and not 5?
        // ASCII formatting has a NULL terminator that signifies the end, so:
        "abcd" == {'a', 'b', 'c', 'd', \0}.
        // You should almost never define a string as an array of characters.

        // Remember that trying to assign past the end of the array, eg:
        another_string = "abcde";
        // The behaviour is undefined, so ~anything~ could happen.
        
        // If you have pointers to data you are not meaning to access, 
        // you will get a segmentation fault
    }
\end{code}

\section{Segmentation Faults}\label{sec:segmentation_faults}

\begin{code}{c}
    int main() {
        int number = 6;
        *number;
        // We almost definitely can't access memory address 6, and it probably
        // has operating system code, so the operating system will just kill it.
        // This is a segmentation fault.
    }
\end{code}

\section{User Input}\label{sec:c_user_input}

\subsection{During Execution}\label{sub:during_execution}

\begin{code}{c}
    int main() {
        int age;

        // We want to update the variable, and not the value, so use a pointer.
        scanf("%d", &age); 
    }
\end{code}

\subsection{Prior To Execution}\label{sub:prior_to_execution}

\begin{code}{c}
    // argc is a conventional name that is the argument count.
    // argv is a conventional name that is an array of the argument texts.
    int main(int argc, char *argv[]) {
        int i;
        for (i = 0; i < argc; i++) {
            printf("%s\n", argv[i])
        }
    }
\end{code}

\section{Structures}\label{sec:c_structures}

\begin{code}{c}
    struct Address {
        int housenumber;
        char postcode[8];
    };

    int main() {
        struct Address connor = {124, "PA3 3BT"};
    };
\end{code}

\subsection{Type Definitions}\label{sub:type_definitions}

\begin{code}{c}
    typedef struct Address {
        int housenumber;
        char postcode[8];
    } address_t;

    typedef int age; // Just an alias really

    int main() {
        address_t connor = {124, "PA3 3BT"};

        connor.housenumber; // Value of housenumber
        &(connor.housenumber); // Memory address of housenumber
    }

    void getAddress(address_t *addressToSet) {
        addressToSet->housenumber; // -> follows a pointer, a . just gets the value.
    }
\end{code}
