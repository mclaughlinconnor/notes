\section{Concurrency In Golang}\label{sec:concurrency_in_golang}

Golang lets you create several threads (named Go Routines) with an input and an output ``channel'' (\mintinline{go}{chan}s) like:
\begin{code}{go}
    func worker(input chan string, output chan bool) {
        // The infinite loop here means that each worker can do several
        // things one after the other
        for true {
            var data string

            data <- input // Wait until we get new data

            // We need to "return" something so that the main thread
            // Properly waits on this thread
            output <- true
        }
    }

    func main() {
        input := make(chan string)
        output := make(chan bool)

        // Create 6 threads
        // If we were to create only 2 threads, we would get hung up
        // trying to add "of" to the input channel. This is because
        // Golang will only try to put data on a channel if there's
        // something waiting to pick it up, there isn't (workers
        // are trying to put data on the output channel which isn't
        // being waited on. Golang will kill the program when it
        // encounters a deadlock.
        // This can cause something similar to a memory leak, but
        // with threads instead.
        for i = 0; i < 6; i++ {
            // Creates go routines, these are not threads. Creating
            // operating system threads is slow and expensive and
            // takes the control away from us. Golang asks the
            // OS for many, many threads, then handles concurrency
            // in its own runtime using the threads it got from
            // the operating system.
            go worker(input, output)
        }

        // Add a string to channel which gets given to each thread
        input <- "The"
        input <- "path"
        input <- "of"
        input <- "the"
        input <- "righteous"
        input <- "man"

        // Get data back out, but do nothing with it
        // If we do not do this, the main thread will end before
        // the workers do
        for i = 0; i < 6; i++ {
            <- signal
        }

        other_function(data);

        // Original worker threads are still running, we have
        // a ``thread leak''.
    }
\end{code}
This process of handling threads in a program runtime, is called Green Threading.
