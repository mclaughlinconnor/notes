\section{Session One}\label{sec:session_one_one}

\subsection{General Overview}\label{sub:general_overview}

\begin{itemize}
	\item Derek travels a lot
	\item Anything in any of the pre-reading or slides or things he talks about can be assessed.
	      \begin{itemize}
		      \item There are \emph{summative assessments} that count toward our grades.
		      \item And \emph{formative assessments} that are for our learning only.
		      \item There is no grade curving (I score an \(A\), I get an  \(A\))
	      \end{itemize}
\end{itemize}

\subsection{Property}\label{sub:property}

\begin{itemize}
	\item \emph{Tangible Property} is an actual thing that can be held or given to someone else.
	\item \emph{Intellectual Property} is an idea or thing that can be duplicated easily (drugs can be reformulated, books can be rewritten, code can be copied).
\end{itemize}

\subsection{Copyright}\label{sub:copyright}

Copyright protects databases, art and literary works from being distributed.
The designs that go into creating these pieces can also be covered by copyright since they can take considerable effort to create.
Even an original curated playlist of music is covered by copyright law.

\subsubsection{Ownership}\label{ssub:ownership}

\begin{itemize}
	\item \textbf{Joint Ownership} is where more than one author's work is too closely intertwined to be separated from each other.
	\item \textbf{Co-Authorship} is where more than one pieces of work are created separately to be used together (chapters in a book).
	\item \textbf{First Ownership} is where a piece of work was created during the course of employment at a company and the work's first owner is automatically the employer.
	\item \textbf{Independent Contractor} - As a contractor all work is originally your own, but you will usually have to sign your ownership over to you employer before your employment begins.
\end{itemize}

\subsubsection{Timeframe}\label{ssub:timeframe}

Generally copyright lasts \(70\) years from the end of the year of the original author's death, but there are exceptions:

\begin{itemize}
	\item \textbf{Unknown Author:} 70 years from when the work was first made.
	\item \textbf{Outside of the European Economic Area:} Each country decides for itself.
	\item \textbf{Computer Generated Works:} 50 years from the end of the year the work was first created.
	\item \textbf{Joint Authorship:} 70 from the end of the year of the last author's death.
\end{itemize}

\subsubsection{Techniques}\label{ssub:techniques}

Patented techniques cannot be copied.
Unpatented techniques can be copied provided the techniques are not source code or any other finished product.

\subsection{Non-Disclosure Agreement}\label{sub:non_disclosure_agreement}

By even having a contract with a company means you cannot speak to a competitor about its practices.
Where the company you have the contract with might or has done something illegal, something that affects someone's health or affects the environment, you are legally obliged to disclose this.

\subsection{Patents}\label{sub:patents}

For something to be eligible for a patent, it must be:

\begin{itemize}
	\item A new idea.
	\item Must have an inventive step somewhere in the process.
	\item Capable of industrial application.
	\item Outside of a specifically excluded area.
\end{itemize}

\subsection{Trademark}\label{sub:trademark}

A trademark may consist of words, phrases, designs, letters, numerals, colours, sounds, etc.

\begin{note}
	\begin{itemize}
		\item[Note:] Internet domains are not covered by trademark law since domains are controlled by ICANN (an international committee).
	\end{itemize}
\end{note}
