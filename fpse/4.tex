\section{Session Four}\label{sec:fpse_session_four}

\textbf{GDPR breaches can be very expensive}

\subsection{Personal Data}\label{sub:personal_data}

\emph{Personal data} is data that relates to an identifiable or identified \textbf{person} (not a company) or allows a person to be identified - even if all unique identifiers are removed, since other details can identify a person and even if some of the data is incorrect.
\emph{Identifiable information} covers:

\begin{itemize}
    \item Names
    \item Addresses
    \item Unique IDs
    \item IP addresses
    \item Cookies
    \item Postcode
\end{itemize}

\subsubsection{Sensitive Personal Data}\label{ssub:sensitive_personal_data}

\emph{Sensitive personal data} is sensitive because of discrimination.

\begin{itemize}
    \item Trade unions can fire striking employees
    \item Health can be used to give worse services (health insurance)
    \item Address/neighbourhood can give worse services too (car insurance)
    \item People with bad genetics can have worse health, so cost more for insurers
\end{itemize}

\noindent
Covers: Race, genetics, religion, politics, trade union membership, biometrics

\subsubsection{Personal Data Value}\label{ssub:personal_data_value}

Generally, things that are free need to harvest you data to sell to make money.
This data is used to target advertisements at you specifically and often more specifically the elderly to take advantage of them.
Also sometimes, people with certain political views may be targeted with the opposing political party.
Developers sometimes use harvested data for production like test data.

\subsubsection{Lawfulness}\label{ssub:lawfulness}

\textbf{You must have valid grounds for collecting and using personal data} and you must not breach any other data protection laws.
The data harvested must not be detrimental or misleading to individuals and the individuals must be informed and consent to the collection.

In collecting data, there must be a clear purpose for collection from the very beginning and data must be sufficient, relevant and adequate for your purpose.

\medskip
\noindent
In summary:

\begin{itemize}
    \item You must have a contract with the harvestees
    \item You must use the data legally
    \item You must only store and use the necessary data
    \item You must only give access to those who need it
    \item Individuals have the right to be informed, access, rectify, erase, restrict the processing of and restrict any automated decision making (including profiling) pertaining to their data
    \item You must have data protection officers and adhere to data protection laws by design and by default
\end{itemize}

\subsubsection{Accuracy}\label{ssub:acuracy}

\begin{itemize}
    \item You must take reasonable steps to ensure the data collected is correct.
    \item You must take reasonable steps to ensure data is kept up to date.
    \item You must take reasonable steps to correct data if it becomes wrong.
    \item You must carefully consider any challenges to the accuracy of your data.
\end{itemize}

\subsubsection{Storage Limitations}\label{ssub:storage_limitations}

\begin{itemize}
    \item \textbf{Store data only as long as is needed.}
    \item Have a policy outlining the standard retention periods.
    \item Periodically review, erase and anonymise data when not needed.
    \item Carefully consider any challenges to the accuracy of your data.
    \item Personal data can be stored for longer if it is stored for the public interest (scientific, statistical or archival purposes)
\end{itemize}

\subsubsection{Accountability Principle}\label{ssub:accountability_principle}

\textbf{Be responsible for your data}

\subsubsection{Engineering Concerns}\label{ssub:engineering_concerns}
\begin{itemize}
    \item Only use the needed data (for example, use \mintinline{sql}{SELECT name FROM ...} instead of \mintinline{sql}{SELECT * FROM ...}).
    \item Where is data stored? (AWS, other hosting? Is this allowed?)
    \item Can data be deleted in a timely manner?
    \item Can a person be easily identified from their data?
    \item Is live data being used for test systems?
    \item Is there a roles/privileges system to restrict data access?
    \item Is any of the data protected? (sexual orientation, religion, etc.)
\end{itemize}

