\section{Session Three}\label{sec:session_three}

\subsection{Code Base Navigation}\label{sub:code_base_navigation}

\begin{itemize}
	\item Helps us to understand the code base
	\item Helps to find/reproduce issues and find a solution
	\item Helps to avoid code reuse
\end{itemize}

\subsection{Changes}\label{sub:changes}

\subsubsection{What starts a code change?}\label{ssub:what_starts_a_code_change_}

\begin{itemize}
	\item A bug
	\item ``Change control''/``User story'': someone higher up wants a feature.
	\item JIRA, Gitlab, work item tracking
\end{itemize}

\subsubsection{Process}\label{ssub:process}

\begin{enumerate}
	\item Change issue raise
	\item Change is prioritised
	\item Change is assigned (either by a manager or by yourself)
	\item Change is made:
	      \begin{enumerate}
		      \item Developed locally and tested locally
		      \item Committed and pushed to a branch
		      \item Deployed to a test environment and tested
		      \item Deployed to live/production:
		            \begin{enumerate}
			            \item Hand over to support team (write FAQs, etc.) to avoid overwhelming the development team
			            \item Handle any post release problems
			            \item Roll back if needed
		            \end{enumerate}
	      \end{enumerate}
\end{enumerate}

\subsection{Deployment}\label{sub:deployment}

Deploy frequently with high quality testing to ensure successful releases and minimal user impact.
Always have as few main branches as possible so that merging branch happens as little as possible to avoid manual conflict resolution.
Possibly use configuration options to line up features with a marketing campaign or to disable the feature again if it ends up being broken.
Deploy to individual servers so that only certain users will be affected if there's a broken version --- these users can be sent to another server with load balancing, keeping uptime high.

\subsection{Commits}\label{sub:commits}

Commits should be self-container (\emph{atomic}) to allow easy rollbacks (commit doesn't have to be split).

\subsubsection{What makes a good commit?}\label{ssub:what_makes_a_good_commit_}

\begin{itemize}
	\item Links back to original work item on GitLab or Jira
	\item Is a singular standalone change
	\item Is simple, small and easy to review (but not too small)
	\item Can span several files
	\item Has limited impact on other parts of the code base
	\item Should consider other people still using old versions (REST services can return different data depending on version)
\end{itemize}

\subsection{Branching}\label{sub:branching}

\begin{itemize}
	\item Master, Trunk, Mainline, Default
	\item Release branches (v1, v2, v3)
	\item Feature branches (blogs, emails, forums)
	\item Personal branches (Connor, Kyle, Callum)
\end{itemize}

Branches should always aim to merge back into the main branch.
