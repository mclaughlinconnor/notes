\section{Session Five}\label{sec:fpse_session_five}

\subsection{Contracts}\label{sub:contracts}

A \emph{contract} is an agreement between one and at least one other person or organisation.
A contract may be either verbal or written.
To be valid, both parties must be competent (not drunk or under-age) and must both give and receive something.

Until recently, existing contract law was sufficient to cover computer hardware and software suppliers.
E-commerce sites need new legal provisions because people can be in various different countries where laws may differ.

\subsubsection{Software Licenses}\label{ssub:software_licenses}

Buying software is not like buying something tangible since you do not buy something tangible, but instead you buy a copy and a license to use it.

\paragraph{Retail licenses}\label{par:retail_licenses}

\begin{itemize}
    \item Have a modest price (less than £500)
    \item A license for just one computer
    \item No maintenance support
\end{itemize}

\paragraph{Organisation licenses}\label{par:organisation_licenses}

\begin{itemize}
    \item Have a slightly higher price
    \item Usually for many computers (up to a limit)
    \item Usually has some maintenance support
\end{itemize}

\paragraph{Corporate licenses}\label{par:corporate_licenses}

\begin{itemize}
    \item There are thousands of generic software solutions, hundreds of slightly specialised software solutions and only tens of highly specialised software solutions.
    \item Very high up front costs
    \item Maintenance is almost always included
    \item Usually follows a subscription model
\end{itemize}

\paragraph{Open Source Licenses}\label{par:open_source_licenses}

\begin{itemize}
    \item Open source software is distributed as source code
    \item The price is very low and often £0
    \item Open source licenses often have certain conditions attached to them.
        \begin{itemize}
            \item Author's name is kept in code
            \item License must be kept the same when distributed
            \item Code may (or may not) be reused in another application
        \end{itemize}
\end{itemize}

\paragraph{Free Software}\label{par:free_software}

\begin{itemize}
    \item No cost (``free'')
    \item Richard Stallman is a prominent supporter and started the GNU (\emph{GNU's not Unix}) Project. GNU is:
        \begin{itemize}
            \item A version of Unix
            \item A set of standard compilers
            \item A set of standard libraries
            \item A set of standard utilities
        \end{itemize}
    \item Linux is the \textbf{superstar} of the free software community
\end{itemize}

\subsubsection{Open Source Java}\label{ssub:open_source_java}

Java was originally made by Sun Microsystems in 1990, then the source was made free in 1995.
Oracle bought Java from Sun in 2010, then promptly sued Google for their use of Java in Android - in 2016, jurors ruled that Google's use was ``fair use'' and the subsequent federal appeal was overruled.
The supreme court is still debating the outcome.

\subsubsection{Bespoke Contracts}\label{ssub:bespoke_contracts}

\emph{Bespoke software} is software developed for just one company.
The contract should outline how the project will be managed, all deliverables (code, compiled programs, support, training) and also account for changing requirements.

Any surcharges should be covered too in case any requirements are changed or the customer is late getting back to you.
Any test cases or performance requirements needed for the project to be considered a success will need to be mentioned here as well.
The contract might specify who retains the copyright once finished.

\subsection{Fixed Price vs Cost Plus}\label{sub:fixed_price_vs_cost_plus}

\paragraph{Fixed Price}\label{par:fixed_price}

This is where a contract will specify exactly how much the customer will pay.
A fixed price contract increases the cost if the customer is at fault (changing requirements) and decreases if the supplier is at fault (unmet requirements, late delivery)
The customer will have to pay any contingency fees regardless.

\paragraph{Cost Plus}\label{par:cost_plus}

The customer will pay the developer's cost with a profit margin.
Cost plus might be used if the supplier thanks the requirements are unclear and may change later.

\subsubsection{Consultancy and Contract Hire}\label{ssub:consultancy_and_contract_hire}

\paragraph{Contract Hire}\label{par:contract_hire}

A supplier provides a certain number of people's services for a time at a role.
The customer manages staff, whereas the supplier is responsible for providing staff with certain competencies.
\emph{Freelancers} are individuals who contract their own services - this is a special case for contract hire.
\emph{Consultants} are experts who are contracted to advice customers on their operations or projects.

Contractors are useful if you don't need a permanent IT departments or just want a short term project completed.
Companies can fire contractors immediately to cut costs in a team.

\medskip
\noindent
Contract hire contracts are far simpler:

\begin{itemize}
    \item Should address intellectual property rights (copyrights, confidentiality)
    \item Terms of reference
    \item Liability (is company or contract hire agency at fault for losses)
\end{itemize}

\subsubsection{Liability for Defective Software}\label{ssub:liability_for_defective_software}

\textbf{All software has bugs.}
Suppliers try to limit their liability for any bugs or faults that may come up.
Typically if software is unusable, the supplier will offer a refund

Terms limiting liability can be overwritten by law.
By law, liability only counts to the point where it is reasonable.
A person who is injured can sue regardless of liability terms and even non-safety critical software can still cause damage (emails losing messages, map misses locations).
