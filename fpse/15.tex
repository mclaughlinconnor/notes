\section{Session 17}\label{sec:session_17}

\subsection{Models}\label{sub:models}

\begin{itemize}
	\item Class diagram: high level overview of code
	\item Interaction diagrams: how other classes interact with each other
	\item Domain model: ?
\end{itemize}

\subsection{Object}\label{sub:object}

\begin{itemize}
	\item Data and properties
	\item Operations
	\item Discrete entities with well defined boundaries
\end{itemize}

\subsection{Lifecycle}\label{sub:lifecycle}

\begin{itemize}
	\item Construction: when created
	\item Running: calling functions, etc. (State is the name given to all of the variables).
	\item Deconstruct: for memory management when the object is killed.
\end{itemize}

\subsection{Universal Modelling Language (UML)}\label{sub:universal_modelling_language_uml_}

\subsubsection{Syntax}\label{ssub:syntax}

\paragraph{Class}\label{par:class}

\begin{itemize}
	\item Name: capitalised single noun (Book, Library).
	\item Attribute: show visibility (public (+) or private (-)), show typing, named with camelCase, include getters and setters.
	\item Operator: Public or private, camelCase, have names and types.
\end{itemize}

\begin{center}
	\begin{tikzpicture}
		\begin{class}[text  width=8cm]{Book}{0,0}
			\attribute{name : string}
			\attribute{author : string}
			\operation{buy(quantity : integer) : boolean}
		\end{class}
	\end{tikzpicture}
\end{center}

\subsubsection{Relationships}\label{ssub:relationships}

\paragraph{Generalisations}\label{par:generalisations}

``\(a\) is a \(b\)'', basically a subclass instance of a superclass (inheritance).

\begin{center}
	\begin{tikzpicture}
		\begin{class}[text width=6cm]{Book}{0,0}
			\attribute{name : string}
			\attribute{author : string}
			\operation{buy(quantity : integer) : boolean}
		\end{class}
		\begin{class}[text width=6cm]{ComicBook}{0,-3}
			\inherit{Book}
			\attribute{name : string}
			\attribute{author : string}
			\operation{buy(quantity : integer) : boolean}
		\end{class}
	\end{tikzpicture}
\end{center}

\paragraph{Aggregate}\label{par:aggregate}

``\(a\) has a \(b\)''. Compound class is made of components. There is no inheritance. Each class is independent of each other.

\begin{center}
	\begin{tikzpicture}
		\begin{class}[text width=6cm]{Library}{0,0}
			\attribute{name : string}
			\attribute{author : string}
			\operation{buy(quantity : integer) : boolean}
		\end{class}
		\begin{class}[text width=6cm]{Book}{0,-3}
			\aggregation{Book}{}{}{ComicBook}
			\attribute{name : string}
			\attribute{author : string}
			\operation{buy(quantity : integer) : boolean}
		\end{class}
	\end{tikzpicture}
\end{center}

\paragraph{Composition}\label{par:composition}

The same as aggregation, but with a dependency.

\begin{center}
	\begin{tikzpicture}
		\begin{class}[text width=6cm]{Library}{0,0}
			\attribute{name : string}
			\attribute{author : string}
			\operation{buy(quantity : integer) : boolean}
		\end{class}
		\begin{class}[text width=6cm]{Book}{0,-3}
			\composition{Library}{}{}{Book}
			\attribute{name : string}
			\attribute{author : string}
			\operation{buy(quantity : integer) : boolean}
		\end{class}
	\end{tikzpicture}
\end{center}

\paragraph{Association}\label{par:association}

Role name and multiplicity.
\(1\) to \(M\), \(1\) to \(1\), \(0\) to \(M\).

\begin{center}
	\begin{tikzpicture}
		\begin{class}[text width=6cm]{Library}{0,0}
			\attribute{name : string}
			\attribute{author : string}
			\operation{buy(quantity : integer) : boolean}
		\end{class}
		\begin{class}[text width=6cm]{Book}{0,-5}
			\attribute{name : string}
			\attribute{author : string}
			\operation{buy(quantity : integer) : boolean}
		\end{class}
		\association{Library}{Owns}{\(1\)}{Book}{\(M\)}{Belongs to}
	\end{tikzpicture}
\end{center}

\paragraph{Class}\label{par:class}

Name, attribute, operations (just a recipe for an object).

\paragraph{Relationships}\label{par:relationships}

Generalisation, aggregation, composition, association.

\bigskip
\noindent
\textbf{You will likely have to draw one of these in the exam.}

