\section{Session 12}\label{sec:session_12}

\subsection{Waterfall}\label{sub:waterfall}

Steps:
\begin{enumerate}
	\item Analysis.
	\item Design.
	\item Implementation.
	\item Testing.
	\item Deployment.
	\item Release and maintenance.
\end{enumerate}

\noindent
The waterfall method has lots of extra steps that can make even small changes take a long time.
Waterfall is very good for wall known technologies where the requirements and tests are well known.

\subsection{Agile Manifesto}\label{sub:agile_manifesto}

\begin{enumerate}
	\item Satisfying the customer is the main priority.
	\item Changing requirements are good because they allow the client to stay ahead of competition.
	\item Release working software as frequently as possible.
	\item Business and developers must communicate frequently.
	\item Give people all of the equipment and support they need.
	\item Face-to-face conversations are prefered where possible.
	\item Working software is the primary measure of success.
	\item Agile promotes sustainability: sponsors, users and developers must move at a constant pace for ever.
	\item Keep it simple. Do as little work as possible.
	\item Attention should be paid to technical excellence and good design patterns.
	\item Self-organising teams give the best requirements and architectures.
	\item Have regular reflections to be more effective in the workplace.
\end{enumerate}

\subsection{Agile Ceremonies}\label{sub:agile_ceremonies}

\subsubsection{Sprint Planning}\label{ssub:sprint_planning}

You should plan to deliver new software every \(x\) weeks or after a specific percentage of backlog issues have been covered.

You should decide on how long things should take, what order/priority different tasks have and how long the entire sprint should last.

\subsubsection{Daily Scrum/Stand Up}\label{ssub:daily_scrum_stand_up}

The meeting should be used to understand what people have done and what they will do today.
You should list any blockers (client data missing, someone else's work is late).
They give people an opportunity to ask or give help.

A scrum should ideally only last about \(20\) minutes and have only \(5\) to \(9\) people.

\subsubsection{Iteration review/Demo}\label{ssub:iteration_review_demo}

These are a chance to review everything that went right and wrong over the course of a sprint.
You should decide what was done well and what should be done differently.
Was everything fast enough?
Were there any extra requirements?
Does the client like it?

You should demo what is working to get early feedback if something needs changed.
The release/deployment should be scoped now.

\subsubsection{Retrospective}\label{ssub:retrospective}

What worked?
What didn't work?
What risks are there?
What slowed us down?
What sped us up?
Always create a list of actions.
Always assign a person for checking up on actions.

\subsubsection{Manifesto recap}\label{ssub:manifesto_recap}

\paragraph{Sprint Planning}\label{par:sprint_planning}

Agile \(1\), \(2\), \(3\), \(9\).

\paragraph{Stand up/Scrum}\label{par:stand_up_scrum}

Agile \(4\), \(6\).

\paragraph{Iteration review/Demo}\label{par:iteration_review_demo}

Agile \(6\), \(7\), \(2\), \(3\).

\paragraph{Retrospective}\label{par:retrospective}

Agile \(12\), \(5\), \(8\), \(10\), \(11\).

\subsection{Project Initiation}\label{sub:project_initiation}

\paragraph{Governance}\label{par:governance}

A leader should be decided upon as well as anyone else in charge.

\paragraph{Project brief}\label{par:project_brief}

Gather the scope and benefits to be delivered to any stakeholders for funding, etc.

\paragraph{Project initiation}\label{par:project_initiation}

Create a high level requirements document, find what infrastructure is needed and what people would be best for what (or who would be needed).

\paragraph{Quality assurance}\label{par:quality_assurance}

What tests are needed for success?

\subsubsection{Requirements}\label{sub:requirements}

Should aim for iterative delivery.
Make sure everything has a priority.
Focus on delivering a minimum viable product (include the things that are most valuable to the client).

\subsubsection{Approach}\label{sub:approach}

In the waterfall process, features and time (cost) may increase.
In agile, costs (time) are fixed, but some features may be missed.


\paragraph{MoSCoW Requirements}\label{par:moscow_requirements}

\begin{itemize}
	\item \textbf{M}ust have: the project cannot do without these.
	\item \textbf{S}hould have: these must be done long-term if not now.
	\item \textbf{C}ould have: these are generally low-cost tweaks.
	\item \textbf{W}on't have: we might get back to these at a later date.
\end{itemize}

\subsection{Sprints}\label{sub:sprints}

You should do work, then deploy straight away because deploying quickly makes customers trust you.

\medskip
\noindent
\textbf{Sprints are not a good idea when requirements are known.}

\medskip
\noindent
Don't overpromise or people will be disappointed with how long things take.
Don't underpromise or people will be disappointed with how long they think things will take.
Make sure all requirements, usecases and features can be completed.
Beware of distractions (meetings, helping others) because these are wasted time.

\subsubsection{Burndown}\label{sub:burndown}

At the start of a sprint measure the velocity of progress (issues per day).
Later on decide if this is too much and adjust later on.
You should account for the fact what people may leave early on a Friday and work late other days.

\subsubsection{Simplicity and Tech Debt}\label{sub:simplicity_and_tech_debt}

\emph{Tech debt} is any unfinished or broken features.
Never guess at requirements (``do as little work as possible'').
Be mindful of distractions.

\subsubsection{Fail Fast and Fail Forward}\label{sub:fail_fast_and_fail_forward}

Always do harder tasks first so you don't fail to finish them after a year of work has gone into the project.
Assign a single person to investigate whether a hard task is even possible.

