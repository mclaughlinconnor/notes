\section{Session 8}\label{sec:session_8}

\subsection{Requirement Gathering}\label{sub:requirement_gathering}

\subsubsection{Scope}\label{ssub:scope}

Scope is a high level description of the project and is useful to get funding on a project.
It should cover what the project will achieve and any business benefits of it.

A stakeholder is a anyone who is involved that will be hurt or benefit from a project; you should have a full list of these.

\subsubsection{Prototype}\label{ssub:prototype}

Useful for when requirements are unknown or unclear to show off to clients.
Can be used to check something is technologically possible with the technology.
Is a new team needed; if yes, this will get them up to speed.
Lets you fail early if this is wrong, but can set expectations on completion dates too early.

\subsubsection{Requirements}\label{ssub:requirements}

\begin{enumerate}
	\item Get an \emph{absolutely complete} list of stakeholders.
	\item Identify decision makers.
	\item Identify who has knowledge.
	\item Get a detailed list of what is needed, wanted, liked and out of scope.
	\item Create tests to tie work back to the requirements.
\end{enumerate}

\noindent
There are different types of questions that would be useful to ask to gather requirements.

\begin{itemize}
	\item Likert [sic] scale (rating from \(1\) to \(5\) or from very to least likely).
	\item Categorical questions like nationality or age range.
	\item True and false questions.
	\item Open ended extended response questions.
	\item A structured interview (a set list of questions or survey)

	      \begin{itemize}
		      \item Allows easy comparisons.
		      \item Creates consistent responses.
		      \item Is difficult to ask for more information on responses because of the set list.
		      \item Don't need skilled interviewers.
	      \end{itemize}
	\item Surveys don't need an interviewer at all and can be used to generate graphs of statistics when there are no open questions.
	\item Unstructured interview

	      \begin{itemize}
		      \item There is no set list of questions.
		      \item Requires a skilled interviewer to ask good questions.
		      \item Can be difficult to make comparisons.
		      \item Follow-up questions can be asked.
	      \end{itemize}
	\item Semi-structured interviews

	      \begin{itemize}
		      \item Has both structured and unstructured sections.
		      \item More complete and useful answers.
		      \item More flexibility
		      \item Unskilled interviewers can cause bad results.
		      \item No consistency.
	      \end{itemize}
	\item Focus groups

	      \begin{itemize}
		      \item Will usually need an icebreaker.
		      \item Always ask open questions (never lead people)
		      \item Have moderators to guide the conversation.
		      \item Include \(6\) to \(10\) people that represent the target group.
		      \item Make sure any key decision makers are included.
		      \item A scribe who understands what's going on needs to take notes.
		      \item Have a clear agenda and time frame.
		      \item Agree on a clear list of actions at the end.
	      \end{itemize}

	\item User stories

	      \begin{itemize}
		      \item ``As a user, I want to do a thing so that something happens''
		      \item Define what is meant by done so progress can be tracked.
	      \end{itemize}
\end{itemize}
