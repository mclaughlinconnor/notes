\section{Enforcing Integrity in the Relational Data Model}\label{sec:enforcing_integrity_in_the_relational_data_model}

\subsection{Relational Data Model Recap}\label{sub:relational_data_model_recap}

Once the header of a table (a relation) is defined, it cannot be changed because this is actually considered creating a new database altogether.
\begin{description}
    \item[Degree] The number of attributes
    \item[Cardinality] The number of rows in the relations where. Remember that the rows are unordered tuples.
\end{description}
\begin{note}
    Don't confuse the cardinality here with the cardinality in an ER diagram.
\end{note}

\subsection{Characteristics of a Relational Data Model}\label{sub:characteristics_of_a_relational_data_model}

A relation is a \emph{set} of tuples, eg.\ \(\{<1, 2, ``\text{Dave}''>, <2, 3, ``\text{Connor}''>, \dots\}\).
There can be no duplicate tuples because the tuples form a set.
This must be checked any time a record is created, modified, or when ``a new tuple is created as a restriction of an old one.''
\begin{note}
    That last sentence is really unclear, he said he would provide an update.
\end{note}
This implies that a primary key must always exist (even if it is just a primary key of all the attributes).

\subsubsection{Tuples}\label{ssub:tuples}

Tuples are unordered meaning that:
\begin{itemize}
    \item A table is only one representation of a table (you can sort the values any way you want).
    \item We can use any order we want
    \item We still have to store the tuple in an order on the computer itself.
\end{itemize}

\subsubsection{Attributes}\label{ssub:attributes}

Attributes are also unordered meaning that:
\begin{itemize}
    \item We can't infer any information from how the names are sorted on a table since the ordering has no meaning.
    \item We cannot get the \(n^\text{th}\) attribute or even the first attribute since there is no order to the attributes.
\end{itemize}

\subsubsection{Atomic Values}\label{ssub:atomic_values}

All values are atomic meaning that an attribute cannot be a set (or otherwise decomposable), so we cannot have multi-valued attributes without creating several tables.
This simplifies the relational database model lots and is called the \emph{First Normal Form Assumption}

\subsubsection{Relations}\label{ssub:relations}

Each relation in the database represents either:
\begin{itemize}
    \item An entity type attributes
    \item A relationship
    \item A set of values (instead of an array).
\end{itemize}

\subsubsection{Null Values}\label{ssub:null_values}

Even when there are unknown values, we must represent them somehow, so we use \mintinline{text}{NULL}.
Of course a primary key cannot be \mintinline{text}{NULL}.

\subsection{Integrity}\label{sub:integrity}

There are two types of integrity that can be enforced on a database:
\begin{description}
    \item[Inherent Integrity Constraints] These hold for all relational database and are generally enforced by the DBMS itself.
    \item[Enterprise Constraints] These are specific to an application and are written by the developers themselves.
\end{description}

\subsubsection{Inherent Integrity Constraints}\label{ssub:integrity_constraints}

These a built in to most DBMSs, some examples include:
\begin{itemize}
    \item Unique primary keys
    \item Non-\mintinline{text}{NULL} primary keys
    \item Foreign keys must exist if there's mandatory participation
    \item Foreign keys must exist in the other table.
\end{itemize}

\subsubsection{Enterprise Constraints}\label{ssub:enterprise_constratints}

New constraints can be added to a database to make sure some condition is always/never met (something like that start date never being after the end date).

\subsubsection{Referential Integrity}\label{ssub:referential_integrity}

Referential integrity concerns foreign keys and guarantees that the relationships between tuples are coherent (ie.\ that non-\mintinline{text}{NULL} foreign keys exist).

\paragraph{Enforcing Referential Integrity}\label{par:enforcing_referential_integrity}

A DBMS enforces each foreign key independently of all other foreign keys.
If we fail to enforce or create relationships properly, we can cause \emph{many} data modelling problems.

\paragraph{Strategies}\label{par:strategies}

\begin{description}
    \item[Restrict] Bans any alterations to primary keys if there are references to them already.
    \item[Cascade] Cascades updates to affect all of the related tuples.
    \item[Set \mintinline{text}{NULL}] Allows the update, but just wipes the foreign key for all related tuples.
\end{description}

\begin{note}
    These only apply when a primary key is changed somehow.
\end{note}

