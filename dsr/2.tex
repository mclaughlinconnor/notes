\section{Database Management Systems}\label{sec:database_management_systems}

\subsection{What is a DBMS?}\label{sub:what_is_a_dbms_}

A DBMS is a suite of services to manage several databases all together.
\begin{itemize}
    \item Enables simple access
    \item Allows multiple users of the same database
    \item Allows for the manipulation of a database
    \item Controls the security and integrity of a database
    \item Supports various forms of database access
\end{itemize}

\subsubsection{Forms of Database Access}\label{ssub:forms_of_database_access}

\paragraph{Local Database}\label{par:local_interface_database}

This is where there is an interface sitting in between the end-user and a database stored on the device itself.
Here we have the benefit of not having to complicate our program with servers of anything else complex, since the database is simply on the device.
However, this means that it is much harder to synchronise the database across devices.

\paragraph{Client-Server Database}\label{par:client_server_database}

In a client-server configuration, instead of the database interface connecting to a local database, it instead connects to a remote server where the database resides.
This server is invisible to the end user, but allows the developer to distribute the servers around the world to help with system failures and reducing connection latencies.
Because we're connected to a single remote server, we can have many devices connected to it at the same time.

\paragraph{Web-Based Database}\label{par:web_based_database}

Here we have a system very similar to the client-server system, except the client isn't on the remote device, it's instead hosted on a remote server that individual browsers each connect to.

\begin{note}
    The important bit here is that several users can access a DBMS at once without the data becoming unsynchronised.
\end{note}

\begin{note}
    It's also important to note that raw file system access is not available through a DBMS since this is abstracted away.
\end{note}

\section{Designing Database Systems}\label{sec:designing_database_system}

When we are considering creating a database system, there are several things we must first consider:
\begin{itemize}
    \item Who will use the database?
    \item What will they see when they use the database?
    \item What data do we need to store in the database?
\end{itemize}

Now that we have captured everything that the database needs to do, we need to begin designing it.
There are always three schemas for every database.
\begin{description}
    \item[External] This is where we consider how external users will view the database.
    \item[Conceptual] In the conceptual stage, we decide how programmers will model and implement the database.
    \item[Internal] The internal schema is how the DBMS stores the data internally -- we never really have to worry about this, but we do need to be aware it exists.
\end{description}

When we design a database, we take the external schema and design a corresponding conceptual schema which the DBMS then turns into an internal schema with the designer's help.

\subsection{Database Design Lifecycle}\label{sub:database_design_lifecycle}

\subsubsection{Gather Requirements}\label{ssub:dsr_db_gather_requirements}

\begin{enumerate}
    \item Gather requirements to find out what the end user needs from the final database system.
    \item Create a conceptual design with high level descriptions (usually in an ER diagram).
    \item Create a logical design which details the relational schema of the final database and is based on the ER digram created earlier.
    \item Refine this stage again, and again, and again, until it is perfect.
    \item Design and tune the physical design to cater to typical workloads and make further optimisations.
\end{enumerate}

\subsubsection{Some Terminology}\label{ssub:some_terminology}

\begin{description}
    \item[Data Model] Descriptions of real world objects that will be created in the database later.
    \item[Schema] A collection of table columns with names, data types, sizes, relationships, etc.
    \item[Database] This is an instance of the schema that has been populated with data.
\end{description}

\begin{note}
    When we design a database, we \emph{always} create this in order.
\end{note}

\subsubsection{Creating a database}\label{ssub:creating_a_database}

\begin{enumerate}
    \item Capture requirements as in \cref{ssub:dsr_db_gather_requirements}.
    \item Represent the data in a model.
    \item Convert the database model to a database schema.
    \item Implement the database in a DBMS.
\end{enumerate}

\begin{note}
    There are lots of different tools available to complete these tasks.
\end{note}

\subsubsection{The People Involved}\label{ssub:the_people_involved}

\begin{description}
    \item[User] Each individual user can be either a casual computer user, or an expert, both need to have an effective way of accessing database data.
    \item[Database Designer] A database designer needs to interact with the database to specify the schema and add content.
    \item[Web Application Designer] The person who is designing .
    \item[Database Administrators] Database administrators need to interact with the database to maintain the accuracy and speed of the system.
\end{description}

\begin{note}
    We always have to consider all of these people when designing a database system since they will all interact with the system at some point
\end{note}

We should firstly always speak to all users to identify each user's individual requirements.
After, we should identify their \emph{data requirements} which are the things they need stored as well as their attributes (types, sizes, etc. too) and relationships.

\subsection{Data Modelling}\label{sub:data_modelling}

We always need to capture all of the data represented in the database in the data model in order to:
\begin{itemize}
    \item Help communicate ideas with other developers
    \item Help the design process of the database by providing a good visualisation
    \item Help with the database implementation process by giving a good reference of the design
\end{itemize}

The type of data model depends on the database types (database types are detailed in \cref{ssub:types_of_database}), since we are mostly using relational databases, we will use ER diagrams.

\begin{note}
    Although you can use almost any diagram for a data model, they should always map easily to a conceptual schema.
\end{note}

\subsubsection{ER Diagrams}\label{ssub:er_diagrams}

Creating ER diagrams is a simple process that has three distinct steps:
\begin{enumerate}
    \item Find your entities (things that you want to store)
    \item Find that data you want to store about your entities
    \item Identify the relationships between the entities
\end{enumerate}

