\section{SQL}\label{sec:sql}

By using the \emph{Structured Query Language} we are able to create data/tables, update data/tables, delete data/tables, and query data/tables in a non-procedural way (meaning that we specify what to get, and not how to get it).

SQL is an ISO standard meaning it is standardised, but as with any standard, every implementation is still slightly different.

The basic steps of writing any SQL query are:
\begin{enumerate}
    \item Find what data you want
    \item Find the constraints you need on that data
    \item Find the relations you want to get the data from
\end{enumerate}

\subsection{SQL SELECT}\label{sub:sql_select}

SQL \mintinline{sql}{SELECT} is the basic mechanism to to retrieve data from a database.
It works by taking a set of relations, then describing the set of constraints that must be put on them to return a set of rows.
\begin{code}{sql}
    SELECT [DISTINCT] target-list -- project attributes
    FROM relations-list -- Cartesian product of relations
    WHERE qualification  -- Selection with boolean expression
\end{code}
\begin{note}
    There are some terminology changes here: the relational algebra projection (\(\Pi\)) has become the SQL \mintinline{sql}{SELECT}, while the relational algebra selection (\(\sigma\)) has become the SQL \mintinline{sql}{WHERE}.
\end{note}
We can also add calculations to a \mintinline{sql}{SELECT} (we should always name them to improve readability).
\begin{code}{sql}
    SELECT a.age * a.age FROM a
\end{code}

\subsubsection{target-list}\label{ssub:target_list}

The target-list can either select all columns or just some:
\begin{code}{sql}
    SELECT column, column, column
    SELECT *
\end{code}

\subsection{UNION}\label{sub:uniondsrseven}

The purpose of the SQL \mintinline{sql}{UNION} is to join two queries together and is performed with the following basic syntax:
\begin{code}{sql}
    (SELECT name FROM x WHERE y)
    UNION
    (SELECT name FROM a WHERE b)
\end{code}
\begin{note}
    This isn't really that useful seeing as you can just add an extra \mintinline{sql}{OR} to the original query.
\end{note}
\begin{note}
    We have to be wary of union compatibility.
\end{note}

\subsection{Cartesian Product}\label{sub:caartesianproductdsrseven}

This is simply expressed as:
\begin{code}{sql}
    SELECT * FROM a, b
    SELECT a.name, b.name FROM a, b
\end{code}

\subsection{Equi-Join}\label{sub:equijoindsrseven}

The equi-join is useful to join the primary and foreign keys of two tables together in a query.
\begin{code}{sql}
    SELECT a.name
    FROM a, b
    WHERE a.id = b.id
\end{code}
\begin{note}
    Remember that \mintinline{sql}{NULL} doesn't equate to anything.
\end{note}
\begin{note}
    This results in some redundant columns from both sides of the relationships, so it can be better to use:
    \begin{code}{sql}
        SELECT a.name
        FROM a
        NATURAL JOIN b
        \end{code}
\end{note}

\subsection{Renaming/Aliases}\label{sub:renaming_aliases}

Defining aliases is a good idea to aid clarity, and is a necessity when we are trying to access attributes from linked tables where the attributes share a name (or the same table is used twice)
\begin{code}{sql}
    SELECT a.name AS name
    FROM a
\end{code}
\begin{note}
    The scope of the alias is throughout the whole program.
\end{note}

\subsection{Semantics of SQL Queries}\label{sub:semantics_of_sql_queries}

\begin{enumerate}
    \item Compute the Cartesian product of the \mintinline{sql}{relation-list}
    \item Remove the tuples that fail \mintinline{sql}{qualification}
    \item Remove the attributes that don't occur in \mintinline{sql}{target-list}
    \item If \mintinline{sql}{DISTINCT} is set, remove all of the duplicates in the relation.
\end{enumerate}

\subsection{SQL WHERE}\label{sub:wheredsrseven}

Using SQL \mintinline{sql}{WHERE}, we can set conditions for the query.
\begin{code}{sql}
    SELECT *
    FROM a, b
    WHERE a.name = 2
        AND a.id <> 4
        AND a.age BETWEEN 10 AND 20
        AND b.id IS NOT NULL
        AND b.name LIKE 'J\%'
        AND b.name LIKE 'J__'
\end{code}
\begin{note}
    SQL supports pattern matching with \mintinline{sql}{LIKE}, but instead of regular RegExps, we use \mintinline{text}{%} in place of \mintinline{text}{.*}, and \mintinline{text}{_} in place of \mintinline{text}{.}
\end{note}

\subsection{Precedence}\label{sub:precedencedsrseven}

Generally the order of operations follows the standard BODMAS, but there is one exception (or addition maybe) where \mintinline{text}{AND} has a higher priority than \mintinline{text}{OR}.
Although you can make use of the order of operations, you should generally just use brackets to make it super clear what the order should be.

\subsection{Ordering}\label{sub:ordering}

Attributes are always unordered, but we can order the resultant rows using \mintinline{sql}{ORDER BY}.
\begin{code}{sql}
    SELECT * FROM a ORDER BY a.age DESC, a.name
\end{code}
\begin{note}
    When you include two attributes to sort by, they are used in order to break any ties.
\end{note}

\section{SQL Design Patterns}\label{sec:sql_design_patterns}

A design pattern is generally an unfinished, but reusable code design for a common problem that can represent best practices.

\subsection{Basic Design Pattern}\label{sub:basic_design_pattern}

The basic design pattern is used when we are only interested in one table:
\begin{code}{sql}
    SELECT x FROM y WHERE z
\end{code}

\subsection{Equi-Join Design Pattern}\label{sub:equi_join_design_pattern}

When there is information is several tables, and rows need filtered based on data in other tables, we should use an equi-join.
\begin{code}{sql}
    SELECT x,  y FROM v, w WHERE x.id = y.id
\end{code}
\begin{note}
    We can also use a \mintinline{sql}{NATURAL JOIN} where the attributes have the same name.
\end{note}
\begin{note}
    If there are several included relations, we should include all of them in the join condition.
\end{note}

\subsection{Self Join Design Pattern}\label{sub:self_join_design_pattern}

Where you are comparing values in different rows of the same tables, you should use a self join which effectively creates a copy of the table.
This is useful for recursive relationships.
\begin{code}{sql}
    SELECT x, y FROM z AS a, z AS b WHERE q
\end{code}

\section{SQL Aggregations and Groups}\label{sec:sql_aggregations_and_groups}

\subsection{Aggregation}\label{sub:aggregationdsrseven}

We can have functions in the \mintinline{sql}{SELECT} clause like:
\begin{code}{sql}
    SELECT AVG(age)
\end{code}
The total list we will cover is:
\begin{itemize}
    \item \mintinline{sql}{SUM}
    \item \mintinline{sql}{AVG}
    \item \mintinline{sql}{COUNT}
    \item \mintinline{sql}{MAX}
    \item \mintinline{sql}{MIN}
\end{itemize}
All of these functions operate on a whole column and return a single value.
\begin{note}
    \mintinline{sql}{NULL} is not included anywhere here.
\end{note}

\subsection{Grouping}\label{sub:grouping}

The SQL \mintinline{sql}{GROUP BY} clause allows us to create several aggregate groupings in one query depending on how many unique field values there are.
\begin{code}{sql}
    SELECT x, y FROM a WHERE x GROUP BY x.type
\end{code}
\begin{note}
    Aggregated or grouped columns must be projected.
\end{note}
\begin{note}
    There is no relational algebra equivalent to the \mintinline{sql}{GROUP BY} clause.
\end{note}
The use of this clause is for when looking for a description of a group and only one value per group is needed.
Seeing ``each'' and a statistic is usually a good clue that we need a \mintinline{sql}{GROUP BY} somewhere.

\section{SQL Order of Execution}\label{sec:sql_order_of_execution}

\begin{enumerate}
    \item Calculate the Cartesian product of the tables.
    \item Eliminate rows not included in the \mintinline{sql}{FROM} clause.
    \item Group rows using the \mintinline{sql}{GROUP BY} clause.
    \item Evaluate any expressions in the \mintinline{sql}{SELECT} clause.
    \item Sort rows with the \mintinline{sql}{ORDER BY} clause.
    \item Add column names from \mintinline{sql}{AS} cases.
    \item Remove any duplicates if \mintinline{sql}{DISTINCT} is set.
\end{enumerate}

