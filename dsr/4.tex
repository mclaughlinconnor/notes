\section{Converting ER Diagrams to Tables}\label{sec:converting_er_diagrams_to_tables}

In general, we create a table for each item of interest in a database.
A ``relation'' is roughly the same as an entity type or some sort of relationship.

\begin{highlight}{How a relation should be presented}
    \begin{tabular}{cccc}
        \cmidrule{2-4}
        Relation Name & \(A_1\) & \(A_2\) & \(A_3\) \\
        \cmidrule{2-4}
                      & Value   & Value   & Value   \\
        \cmidrule{2-4}
                      & Value   & Value   & Value   \\
        \cmidrule{2-4}
                      & Value   & Value   & Value   \\
        \cmidrule{2-4}
                      & Value   & Value   & Value   \\
        \cmidrule{2-4}
    \end{tabular}
\end{highlight}


\subsection{The Heading}\label{sub:the_heading}

All relations must have a heading (the heading is the schema) which contains the name of the relation and the names of the columns (attributes).
The \emph{number of attributes} is the \emph{degree} of the relation.

\subsection{Schemas}\label{sub:schemas}

A \emph{relation schema} is a set of attributes written as \(Relation(A_1, A_2, A_3)\) where \(A_1\), \(A_2\), \(A_3\) are attributes and their types. Eg. \mintinline{text}{name: Text}. The types of the attributes are called \emph{domains}.

\begin{note}
    A database schema is a collection of relation schemas.
\end{note}

\subsection{Domains}\label{sub:domains}

The domain determines the set of values that can be assigned to an attribute  and any operations that can be performed on that attribute (eg. Add, subtract, concatenate).

For every domain, there are two aspects to consider:
\begin{description}
    \item[Meaning] The set of matriculation numbers
    \item[Format] An integer in the range \(1\) to \(100\)
\end{description}

\begin{note}
    Different DBMSs offer different domains.
\end{note}

\subsection{Strong Entity Types}\label{sub:strong_entity_types}

Create a column for each strong attribute, then create a primary key from one or more of the key candidates.

Where you have a composite attribute, they should be split up into separate attributes that are prefixed with the name of the parent attribute.

\subsection{Primary Keys}\label{sub:primary_keys_four}

A primary key is an attribute that uniquely identifies a particular record in a relation that allows access to all of the other data on the record.

Sometimes there isn't one ideal key, so we can combine \(\geq 2\) attributes into a \emph{composite key}.

\subsection{Foreign Keys}\label{sub:foreign_keys}

Since primary keys uniquely identify a relation, we can include a reference to a primary key in another relation to link the two together.
A foreign key is an attribute (or set of attributes) that exist in at least one table as a primary key to link those tables together.
\begin{note}
    Although foreign keys don't need to have the same name on all tables, they do need to have the same data type.
\end{note}
Two pieces of data can be associated when:
\begin{itemize}
    \item They have the same primary key (they exist on the same table row)
    \item They are linked by a foreign key
\end{itemize}

\subsection{Weak Entities}\label{sub:weak_entities}

Create a primary key from either:
\begin{itemize}
    \item Partial keys of weak entities
    \item Primary keys from the identifying relationship
\end{itemize}

\subsubsection{Steps}\label{ssub:steps}

\begin{enumerate}
    \item Create the new relation schema
    \item Add primary keys from related entities as foreign keys
    \item Declare composite primary keys for the weak entities from the foreign keys
    \item Add any other attributes for the entity
\end{enumerate}

\subsection{Relationships}\label{sub:relationships_four}

\subsubsection{One to many relationship}\label{ssub:one_to_many_relationship}

In a one to many relationship, the primary key of the singular side should be moved to the side of the relationship with multiple entities as a foreign key.

We should also add any of the attributes from the relationship itself to the side with multiple entities.

\subsubsection{One to one relationship}\label{ssub:one_to_one_relationship}

Here we can add the foreign key attributes from either side to the other side.
You should identify one side as the \emph{subject} and the other as the \emph{target}, where the target will receive key attributes, and the subject will give them away.

\begin{note}
    If one of the entities have mandatory participation in the relationship, make this the target, and mark the foreign keys as \mintinline{text}{REQUIRED} and \mintinline{text}{NOT NULL}.
\end{note}

\subsubsection{Many to many relationship}\label{ssub:many_to_many_relationship}

You need to create a new relation (a table) with the primary keys of both sides as foreign keys, as well as any attributes from the many to many relationship itself.
Use the foreign keys on the new relation to create a compound primary key for the new relation.

\subsection{Subtypes and Supertypes}\label{sub:subtypes_and_supertypes}

This style of inheritance isn't really a part of the relational model, and therefore, isn't supported by most standard relational DBMSs.
However we will still sometimes want to create entities that inherit from other entities, so we have two choices:
\begin{enumerate}
    \item Create a single table for all subtypes with \mintinline{text}{NULLABLE} attributes (only some subtypes will need some attributes).
    \item Create several tables where they each have a one to one relationship with the supertype.
\end{enumerate}

\begin{note}
    We should generally prefer to create several tables as far as possible.
\end{note}

\subsection{Composite Attributes}\label{sub:composite_attributes}

Where we have attributes and these attributes have their own attributes, we should flatten these into single layer attributes that have been prefixed with the parent attribute's name for readability.

\subsection{Multi-Valued Attributes}\label{sub:multi_valued_attributes}

It's really bad to be storing several values in one database field, so we should create a new weak entity which has a primary key from the origin table, and another field containing the array element.

\subsection{Final Step Summary}\label{sub:final_step_summary}

\subsubsection{Strong Entities}\label{ssub:strong_entities}

\begin{enumerate}
    \item Build a table with columns for each attribute
\end{enumerate}
\subsubsection{Weak Entities}\label{ssub:weak_entities}

\begin{enumerate}
    \item Build a table with columns for each attribute
    \item Add the primary key of the owner entity as a foreign key
\end{enumerate}
\subsubsection{Relationships}\label{ssub:relationships_dsr_four}

\begin{description}
    \item[One to many] Add the primary key as a foreign key to the many side
    \item[Many to many] Create a new relation with all primary keys
    \item[One to one] Add the primary key of any side to the foreign key of the other side
\end{description}
\subsubsection{Subtypes}\label{ssub:subtypes}

\begin{itemize}
    \item Collapse to large super type relation
    \item[] \textbf{OR}
    \item Create as individual relations with one to one relationship back to super type.
\end{itemize}

\subsection{Tips}\label{sub:tips}

\begin{itemize}
    \item Follow the guide in \cref{sub:final_step_summary}.
    \item Create a schema first, then afterwards, go to the DBMS to add the tables.
    \item Add the entities own attributes first, then add foreign keys.
    \item Be careful to select good data types so that they match when connecting primary and foreign keys.
\end{itemize}
