\section{Sets and Set Theory}\label{sec:sets_and_set_theory}

\begin{note}
    \textbf{This lecture is mostly an extension to \cref{sec:mathematical_sets}, so I will only be including the new content here.}
\end{note}
Whenever we see \(\{\) and \(\}\) they \emph{always} mean a set, never anything else.

Two sets are said to be equal if and only if they have the same elements.
Again, order and repetition still doesn't matter.

Most of the time, it is clear how elements are related, but not always.
You are allowed to have a seemingly random selection of values in a set, but you have to be able to define the relationship still.

\subsection{Set Builder}\label{sub:set_builderdsrsix}

\[
    O=\{x  \mid x \text{ is an odd integer less than \(10\) } \}
\]
\[
    O = \{ x  \mid x \in  N \land x < 10 \land x \% 2 == 1 \}
\]
We can also use a predicate still, which is still a boolean function that defines what can or can't be in a set.

\begin{note}
    Sometimes we will use \(\{\}\) instead of \(\emptyset\) to denote an empty set.
\end{note}

\subsection{Graphical Representation}\label{sub:graphical_representation}

We will mostly use Venn diagrams, where the \(\U\) set has a special rectangle instead of a circle.

\begin{note}
    One peculiar difference on this course versus Practical Algorithms is that, \(A \subseteq B\) signifies that \(A\) is either a subset or equal to \(B\), while \(A \subset B\) signifies that \(A\) is only a subset of \(B\).
\end{note}

\subsection{Power Sets}\label{sub:power_sets}

The power set of a set is denoted as \(\mathbb{P}(S)\) where \(S\) is the target set.
The power set of a set is a set that contains the empty set, the target set, and every other possible subset of the target set.

\begin{note}
    There will always be \(2^{n}\) elements in a power set, where \(n\) is the number of elements in the target set.
\end{note}

\subsection{Set Operations}\label{sub:set_opeationsdsrsix}

We have most of the same operations as in Practical Algorithms (\cref{sec:set_operations}), with the exception that \(\overline{A}\) is the complement of \(A\), and the addition of the symmetric difference which is basically an XOR and is defined as:
\[
    A \oplus = (A-B) \cup (B-A)
\]
\begin{note}
    Sometimes the symmetric difference is called the Disjunctive Union.
\end{note}

\subsection{How Does All of This Relate To Databases}\label{sub:how_does_all_of_this_relate_to_databases}

There are two ways to define a database query.
\begin{description}
    \item[Procedural] A sequence of operations where the output of one step is input to the next.
    \item[Declarative] Describes the desired results in terms of various conditions, then the database figures out the individual procedural operations.
\end{description}
\begin{note}
    Remember that a relation is a set of tuples.
\end{note}

\subsubsection{\(n\)-tuples}\label{ssub:n_tuples}

An \(n\)-tuple is a tuple where \(n\) is the degree of the relation.
N-tuples are not sets, so are shown as \(\left<Connor, 2002, 85\right>\).
This notation is very important since:
\[
    \{1, 2, 3\} = \{3, 2, 1\}
\]
but
\[
    \left<1, 2, 3\right> \neq \left<3, 2, 1 \right>
\]

\subsubsection{Cartesian Product}\label{ssub:cartesian_product}

The Cartesian product is used in database to get all of the possible tuples for  a relation, that we can then filter down later on.

\section{Relations and Relational Algebra}\label{sec:relations_and_relational_algebra}

The Cartesian Product is the basic mechanism to make relations in a database where each relation is a subset of the Cartesian product of the attribute domains.

Relationships between elements of sets are represented in relations

\begin{quote}
    ``A relation is a subset of the Cartesian product of the domains defining the relation.''
\end{quote}

\subsection{Relations}\label{sub:relationsdsr61}

If we say that \(A\) and \(B\) are sets, then the binary relation from \(A\) to \(B\) is a subset of the Cartesian product of \(A \times B\).
We can use \(a R b\) to denote that \(a\) and \(b\) are related in \(R\) like \(\left<a, b\right> \in R\).

\subsection{Foreign Keys in Domains}\label{sub:foreign_keys_in_domains}

By adding a foreign key to an attribute, we limit the domain to also only having foreign keys that exist on the primary key.

\section{Relational Algebra}\label{sec:relational_algebra}

Sets and relations help to understand how data is stored in a database.
Relational algebra helps us to understand how a DBMS stores the data underneath.

When a query is performed, the database gets the Cartesian product, then filters out a subset of those tuples.
Individual data extraction ``programs'' can retrieve data as a subset of a relations or combine two relations (or subsets of relations) together.

\subsection{Preliminaries}\label{sub:preliminaries}

A query is a list of operations applied to relation instances returning another, different relation instance.
We use relational algebra to create these operations where the format is either:
\[
    R_{output}=\text{ Operator }_{parameters} (R_{input})
\]
or
\[
    R_{output}=R_{input} \text{ Operator }_{parameters} (R_{input})
\]
where the parameters can be column names, conditions, etc.

\subsection{Select}\label{sub:selectdsrsix}

For selection in relational algebra, we use the \(\sigma\) symbol to extract specific tuples from a relation by condition.
\[
    R=\sigma_{\text{ city }=\text{ ``Glasgow'' }}(\text{ Employee })
\]
This selects the employees from Glasgow.

\subsection{Project}\label{sub:subsectiproject_label_sub_sub}

For projection, we use \(\Pi\) to extract specific columns from relations given their names.
\[
    R=\Pi_{\text{ gender }, \text{ salary }}(\text{Employee})
\]
This chooses only the gender and salary columns from the employees.

\subsection{Combinations}\label{sub:combinationsdsrsix}

\[
    R=\Pi_{\text{ house }, \text{ street }}(\sigma_{\text{ city }=\text{ ``Glasgow'' }}(\text{ Employee }))
\]
\begin{note}
    We can swap the order of the operations here and the DBMS sometimes actually does these to improve the performance of the query.
\end{note}

\subsection{Union}\label{sub:uniondsrsix}

The union operator combines two relations, and removes any duplicates, but the two relations being joined must have the same number of columns with the same domains.
\[
    R=\sigma_{\text{ city }=\text{ ``Glasgow'' }}(\text{ Employee }) \cup \sigma_{\text{ city }=\text{ ``Stirling'' }}(\text{ Employee })
\]

\subsection{Intersection}\label{sub:intersectiondsrsix}

Intersection is very similar to the union operator, except it selects only when a value is in both relations (and uses \(\\cap\)).

\subsection{Difference}\label{sub:differencedsrsix}

Difference basically subtracts the values from one relation that are in another, it uses the column names from the first relation.

\subsection{Union Compatibility}\label{sub:union_compatibility}

Some operations don't require union compatibility, these operations indicate that an operation is a joining one.

\subsection{Join}\label{sub:join}

We can do a join on a foreign key with the Cartesian product using the \(\bowtie\) symbol:
\[
    R=R_1 \bowtie_{\text{ fk }=\text{ pk }} R_2
\]
which can also be expressed as
\[
    R=\sigma_{\text{ fk }=\text{ pk }}(R_1 \times R_2)
\]

\subsubsection{Inner Join}\label{ssub:inner_join}

We also have the inner join (or ``natural join'' or ``equi-join'') which is also symbolised by \(\bowtie\) which is where we skip out on the condition and the link is made between attributes of the same name:
\[
    R=R_1 \bowtie R_2
\]

\subsubsection{Cartesian Product}\label{ssub:cartesian_productdsrsixx}

Cartesian product has already been covered in Practical Algorithms and here previously. It basically joins two relations together by generating all of the possible tuples for the two relations.
It is signified by the \(\times\) symbol.
There can be no conditions here.
This is rarely actually used since it returns lots of useless and redundant data.
